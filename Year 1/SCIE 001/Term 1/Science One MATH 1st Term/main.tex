\documentclass{article}

\usepackage{amsmath, amsthm, amssymb, amsfonts}
\usepackage{thmtools}
\usepackage{graphicx}
\usepackage{setspace}
\usepackage{geometry}
\usepackage{float}
\usepackage{hyperref}
\usepackage[utf8]{inputenc}
\usepackage[english]{babel}
\usepackage{framed}
\usepackage[dvipsnames]{xcolor}
\usepackage{tcolorbox}
\usepackage{xeCJK}
\usepackage{physics}
\usepackage{tikz-cd}
\usepackage[mathscr]{euscript}

\colorlet{LightGray}{White!90!Periwinkle}
\colorlet{LightOrange}{Orange!15}
\colorlet{LightGreen}{Green!15}
\colorlet{LightBlue}{Blue!15}

\newcommand{\HRule}[1]{\rule{\linewidth}{#1}}
\newcommand{\R}{\mathbb{R}}
\newcommand{\C}{\mathbb{C}}
\newcommand{\Q}{\mathbb{Q}}
\newcommand{\N}{\mathbb{N}}
\newcommand{\U}{\mathbb{U}}
\newcommand{\lima}{\lim_{x \to a}}
\newcommand{\dps}{\displaystyle}

\declaretheoremstyle[name=Theorem,]{thmsty}
\declaretheorem[style=thmsty,numberwithin=section]{theorem}
\tcolorboxenvironment{theorem}{colback=LightBlue}

\declaretheoremstyle[name=Proposition,]{prosty}
\declaretheorem[style=prosty,numberwithin=section]{proposition}
\tcolorboxenvironment{proposition}{colback=LightGreen}

\declaretheoremstyle[name=Principle,]{prcpsty}
\declaretheorem[style=prcpsty,numberlike=theorem]{principle}
\tcolorboxenvironment{principle}{colback=LightGray}

\declaretheoremstyle[name=Definition,]{defnsty}
\declaretheorem[style=defnsty,numberwithin=section]{definition}
\tcolorboxenvironment{definition}{colback=LightOrange}

\setstretch{1.2}
\geometry{
    textheight=9in,
    textwidth=5.5in,
    top=1in,
    headheight=12pt,
    headsep=25pt,
    footskip=30pt
}

% ------------------------------------------------------------------------------

\begin{document}

% ------------------------------------------------------------------------------
% Cover Page and ToC
% ------------------------------------------------------------------------------

\title{ \normalsize \textsc{}
		\\ [2.0cm]
		\HRule{1.5pt} \\
		\LARGE \textbf{\uppercase{Basic Multivariable Calculus - MATH 226 Notes}
		\HRule{2.0pt} \\ [0.6cm] \LARGE{基本多元微积分——MATH 226 笔记} \vspace*{10\baselineskip}}
		}
\date{}
\author{\textbf{Author} \\ 
		Wenyou (Tobias) Tian \\ 田文友 \\
		University of British Columbia \\ 英属哥伦比亚大学 \\
		2023}

\maketitle
\newpage

\tableofcontents
\newpage
\section{Limits and Derivatives, 极限与导数}
\subsection{Limits and One-sided limits, 极限与单向极限}
\begin{definition}
    We denote the "limit of $f(x)$ as $x$ approaches $a$" to be
    $$\lima f(x) = L$$
\end{definition}
Note that if $\mathscr{D}_f = \{x \in \R: x \ne a\}$, the limit of $f(x)$ as $x$ approaches $a$ can still exist. In this case, we take the limit from both sides where $x \to a^-$ and $x \to a^+$. \\
$f(a)$ is irrelevant to $\displaystyle \lima f(x)$. \\
If we cannot assign a value to such a limit, we say this limit \textbf{\textit{does not exist}}, denoted as
$$\lima f(x) = DNE$$
Fundamentally, 
$$f(x) = c \implies \lima f(x) = c$$
$$f(x) = x \implies \lima f(x) = a$$
To compute basic limits, we have the following laws:
\begin{quote}
    If $\displaystyle \lima f(x)$ and $\displaystyle \lima g(x)$ exist, then
    $$\lima (f(x) \pm g(x)) = \lima f(x) \pm \lima g(x)$$
    $$\lima (f(x)g(x)) = \lima f(x) \lima g(x)$$
    additionally, if $\displaystyle \lima g(x) \ne 0$, then 
    $$\lima \frac{f(x)}{g(x)} = \frac{\displaystyle \lima f(x)}{\displaystyle \lima g(x)}$$
\end{quote}
When encountering the form of $\frac{0}{0}$ after directly plugging in the value, we can try the following strategies
\begin{quote}
    1. Factor and cancel \\
    2. Multiply by the conjugate when encountering radicals \\
    3. Expand the brackets and simplify \\
    4. Manipulate to reduce to well-known limits
\end{quote}
Note that a limit of the form $\frac{0}{0}$ can still $DNE$. However, if the form takes $\frac{a}{0}$, where $a \ne 0$, it is concrete that such a limit will not exist. The limit diverges to $\pm \infty$. \\
So what does it mean for a limit to $DNE$?
\begin{quote}
    1. $\displaystyle \lim_{x \to a^-} f(x) \ne \lim_{x \to a^+} f(x)$, that is the limit approaches different values from two sides of the point of interest. \\
    2. $f(x)$ keeps increasing as $x \to a$, for example, a vertical asymptote. \\
    3. $f(x)$ oscillates as $x \to a$, for example, $f(x) = \sin{\frac{1}{x}}$ oscillates as $x \to 0$.
\end{quote}
Consequently, 
\begin{theorem}
    \textbf{The limit theorem} states that \\
    $\dps \lima f(x)$ exists \textbf{if and only if}
    $$\dps \lim_{x \to a^-} f(x) = \lim_{x \to a^+}f(x) = L$$
    then
    $$\lima f(x) = L$$
\end{theorem}

\subsection{Infinite limit and Squeeze Theorem, 无限极限值与夹逼定理}
We then discuss the $2^{nd}$ case as indicated in the previous section about limit $DNE$: $f(x)$ keeps increasing/decreasing as $x \to a$. \\
We say the function has an infinite limit as $x \to a$, for example, in general
$$\lim_{x \to 0} \frac{1}{x^n} = \infty$$
if $n$ is an even number. \\
Note that $\infty$ is \textbf{NOT} a number, therefore, if a limit diverges to infinity, the limit still does not exist. We just use the notation
$$\lima f(x) = \pm \infty$$
as it is more precise to describe the function's behaviour. \\
\\
Now we discuss the $3^{rd}$ case, where the function oscillates. Consider the following limit
$$\lim_{x \to 0} \sin{\frac{1}{x^2}}$$
This limit does not exist, because no matter how close $x$ approaches $0$, the function $f(x) = \sin{\frac{1}{x^2}}$ will oscillate. \\
However, similar limits will exist, for example,
$$\lim_{x \to 0} x\sin{\frac{1}{x^2}} = 0$$
$$\lim_{x \to 0} x^2\sin{\frac{1}{x^2}} = 0$$
Why?
\begin{theorem}
    \textbf{The Squeeze Theorem} states that \\
    if we have $f(x) \le g(x) \le h(x)$ for $x \to a$ on some $\mathscr{D}$, and $\dps \lima f(x) = \lima h(x) = L$, then
    $$\lima g(x) = L$$
\end{theorem}


\subsection{Continuity, 函数连续性}
A function is continuous at $x=a$ if and only if:
\begin{quote}
    1. $f(a)$ exists; \\
    2. $\dps \lima f(x)$ exists; \\
    3. $f(a) = \dps \lima f(x)$
\end{quote}
All elementary functions are continuous, for example, polynomials, rational functions, exponentials, logarithms, trigonometric functions, and radical functions on their $\mathscr{D}$. \\
Examples of discontinuities include:
\begin{quote}
    1. Removable discontinuity (hole); \\
    2. A jump; \\
    3. Vertical asymptote
\end{quote}
We also consider how open and close intervals affect a function's continuity.
\begin{quote}
    $f(x)$ is continuous on $(a, b)$ if it is continuous on all interior points; \\
    $f(x)$ is continuous on $[a, b]$ if it also has the properties:
    $$\lim_{x \to a^+} f(x) = f(a)$$
    $$\lim_{x \to b^-} f(x) = f(b)$$
\end{quote}

\subsection{Intermediate Value Theorem, 介值定理}
\begin{theorem}
    \textbf{Intermediate Value Theorem} (IVT) states that \\
    if $f$ is continuous on the \textbf{closed interval} $[a, b]$, and $N$ is value between $f(a)$ and $f(b)$, then
    $$\exists c \in [a, b], f(c) = N$$
\end{theorem}
Application-wise, we can use the IVT to determine whether or not an equation has a root within a certain interval.

\subsection{Derivatives, 导数与导函数}
\begin{definition}
    A function $f(x)$ is \textbf{differentiable}(可微) at $x = a$ if
    $$f'(a) := \lima \frac{f(x) - f(a)}{x - a}$$
    $$f'(a) := \lim_{h \to 0} \frac{f(a + h) - f(a)}{h}$$
    the limit exists and is finite.
\end{definition}
We have assumed that $f(x)$ is defined on $(x_1, x_2)$, such that $a \in (x_1, x_2)$. \\
We have $3$ different interpretations of \textbf{differentiability}:
\begin{quote}
    1. $f$ has an instantaneous rate of change at $x = a$ \\
    2. Slope of the chord (secant) from $(x, f(x))$ to $(a, f(a))$ converges to the slope of the tangent line at $(a, f(a))$ \\
    3. We have a "good" linear approximation $L(x)$ "near" $x=a$, that is
    $$\lima \frac{f(x) - L(x)}{x - a} = 0$$
    where $L(x) = f(a) + f'(a)(x-a)$
\end{quote}
There are $3$ general ways for a function to be non-differentiable. \\
Since a function is differentiable at $x = a$ if
$$\lim_{h \to 0^-} \frac{f(a + h)-f(a)}{h} = \lim_{h \to 0^+} \frac{f(a + h)-f(a)}{h}$$
Therefore, we can have these to be non-differentiable:
\begin{quote}
    1. Corner/Not smooth: e.g. $f(x)=|x|$ at $x=0$ \\
    2. Vertical tangent line: e.g. $f(x) = \sqrt[3]{x}$ at $x=0$ \\
    3. Discontinuity: e.g. $f(x) = \sin{\frac{1}{x^2}}$ at $x=0$
\end{quote}
\begin{theorem}
    Differentiability and Continuity: \\
    If a function is differentiable at $x=a$, then the function is continuous at $x=a$. \\
    If a function is discontinuous at $x=a$, then the function is non-differentiable at $x=a$.
\end{theorem}

\subsection{Rules of Derivatives, 导数运算规则}
We first consider a function $y = f(x)$, then its derivative function can be expressed in various notations:
$$y' = f'(x) = \dv{y}{x} = \dv{f(x)}{x} = \dv{x} f(x) = D_xf(x)$$
Let $f$ and $g$ be differentiable at $x$, then we have the following rules when computing derivatives:
$$[f(x) \pm g(x)]' = f'(x) \pm g'(x)$$
$$[f(x)g(x)]' = f'(x)g(x) + f(x)g'(x)$$
$$[\frac{f(x)}{g(x)}]' = \frac{f'(x)g(x)-f(x)g'(x)}{(g(x))^2}$$
$$\dv{x}x^n = nx^{n-1}$$
$$[(f \circ g)(x)]' = (f(g(x)))' = f'(g(x))g'(x)$$
\begin{definition}
    For a function $f$, \\
    if $\exists I = [a, b], \forall x \in I, f$ is differentiable at $x$, we define
    $$f'(a) := \lim_{x \to a^+} \frac{f(x)-f(a)}{x-a}$$
    $$f'(b) := \lim_{x \to b^-} \frac{f(x)-f(b)}{x-b}$$
\end{definition}
Notation-wise,
$$f'(a) = \dv{x}|_{x=a} f(x)$$
For higher-order derivatives, like an $n$-order derivative, we have the notation
$$f^{(n)}(x) = \dv[n]{x}f(x) = \dv[n]{y}{x}$$
By the power rule, if a polynomial has an order of $n$, then
$$\forall m > n, f^{(m)}(x) = 0$$

\subsection{Linear Approximation and Mean Value Theorem, 线性近似与中值定理}
Consider a function $f$, for $x$ near $a$, we have a linear approximation of
$$f(x) \approx f(a) + f'(a)(x-a)$$
\begin{theorem}
    3 versions of the \textbf{Mean Value Theorem} (MVT): \\
    \textbf{Rolle's Theorem} states that: Let $f(x)$ be a function continuous on $[a, b]$ and differentiable on $(a, b)$, if $f(a) = f(b)$, then there exists $c \in (a, b)$ such that $f'(c) = 0$ \\
    \\
    \textbf{Lagrange's Theorem} states that: Let $f(x)$ be a function continuous on $[a, b]$ and differentiable on $(a, b)$, there exists $c \in (a, b)$ such that
    $$f'(c) = \frac{f(b)-f(a)}{b-a}$$
    \textbf{Cauchy's Theorem} states that: Let $f(x)$ and $g(x)$ be functions continuous on $[a, b]$ and differentiable on $(a, b)$, there exists $c \in (a, b)$ such that
    $$\frac{f'(c)}{g'(c)}=\frac{f(b)-f(a)}{g(b)-g(a)}$$
\end{theorem}
Rolle's Theorem is a special case of MVT, and MVT is a general case of Rolle's Theorem.

\subsection{Increase and Decrease of Functions, 函数的增减性}
\begin{definition}
    For a function $f$ to \textbf{strictly increase}(单调递增)or \textbf{strictly decrease}(单调递减)on an interval $I$: \\
    If $\forall x_1, x_2 \in I, x_1 < x_2 \implies f(x_1) < f(x_2)$, then $f$ is strictly increasing on the interval $I$. \\
    If $\forall x_1, x_2 \in I, x_1 < x_2 \implies f(x_1) > f(x_2)$, then $f$ is strictly decreasing on the interval $I$.
\end{definition}
Then with the use of MVT, we can show the following theorem:
\begin{theorem}
    The \textbf{Increasing/Decreasing Function Theorem} states that: \\
    Given $f$ is continuous on $[a, b]$ and differentiable on $(a, b)$, then \\
    if $\forall x \in (a, b), f'(x) > 0$, then $f$ is strictly increasing on $(a, b)$ \\
    if $\forall x \in (a, b), f'(x) < 0$, then $f$ is strictly decreasing on $(a, b)$
\end{theorem}
\newpage
\section{Ordinary Differential Equations, 常微分方程}

\subsection{Related Rates and Implicit Differentiation, 相关的变化率与隐函数求导}
\subsubsection{Related Rates}
Related rates is a type of problem where we are given $y = y(x)$, $z = z(x)$, some relationship of $f(y) = g(z)$, and $\dv{z}{x}$, so that we can find $\dv{y}{x}$ by differentiating with respect to $x$ on both sides of the relationship.
\begin{align*}
    f(y) &= g(z) \\
    \dv{f(y)}{x} &= \dv{g(z)}{x} \\
    \dv{f}{y} \cdot \dv{y}{x} &= \dv{g}{z} \cdot \dv{z}{x}
\end{align*}
\subsubsection{Implicit Differentiation}
Implicit differentiation is used to determine $\dv{y}{x}$ when we know that $y = y(x)$ (that is, $y$ is a dependent variable with respect to x) but we are only given a relationship with both $x$ and $y$ in it altogether. \\
Consider the following example
\begin{align*}
    e^y\sin{x} &= x^2 + \ln{y} \\
    \dv{(e^y\sin{x})}{x} &= \dv{(x^2 + \ln{y})}{x} \\
    e^y \cdot \dv{y}{x} \cdot \sin{x} + e^y \cdot \cos{x} &= 2x + \frac{1}{y} \cdot \dv{y}{x}
\end{align*}
We can then find $\dv{y}{x}$ by moving terms around, and express the derivative in terms of both $x$ and $y$.

\subsection{Antiderivatives and ODE, 不定积分与常微分方程}
\begin{definition}
    An \textbf{anti-derivative}(不定积分)of a function $f(x)$, \\
    is a function $F(x)$ such that
    $$F'(x) = \dv{F}{x} = f(x)$$
\end{definition}
Antiderivatives of some common functions can be summarized as the following
\begin{center}
    \begin{tabular}{||c|c|c|c||}
        \hline
        $f(x)$ & $F(x)$ & $f(x)$ & $F(x)$ \\
        \hline
        $x^n$ & $\frac{x^{n+1}}{n+1}$ & $\frac{1}{x}$ & $\ln{|x|}$ \\
        \hline
        $e^x$ & $e^x$ & $\cos{x}$ & $\sin{x}$ \\
        \hline
        $e^{\sin{x}}\cos{x}$ & $e^{\sin{x}}$ & $\ln{x}$ & $x \ln{x} - x$ \\
        \hline
    \end{tabular}
\end{center}
Since the antiderivative of a function is not unique, therefore
\begin{theorem}
    Any $2$ anti-derivatives of the same function differ by a constant, \\
    that is, if $F_1^{'}(x) = F_2^{'}(x) = f(x)$, then
    $$F_1^{'}(x) - F_2^{'}(x) = C$$
\end{theorem}
With the basic concepts of anti-derivatives, we can now investigate $1^{st}$-order ordinary differential equations.
\begin{definition}
    A $1^{st}$-order ordinary differential equation is that, \\
    given we know $y=y(x)$, we have a relationship in the form of
    $$y' = f(y, x)$$
    $$y'(x) = f(y(x), x)$$
    such that the highest order derivative is a $1^{st}$-order derivative, and this form is an ODE.
\end{definition}
Some practical and common $1^{st}$-order ODEs include
\begin{quote}
    1. Radioactive decay
    $$y'(t) = -ky(t), k > 0 \to y = Ce^{-kt}$$
    2. Population growth
    $$y'(t) = ky(t), k>0$$
    3. Population growth with scarce resources
    $$y'(t) = ay(t)(1-\frac{y(t)}{b}), a,b>0$$
    4. Falling object with air resistance (define downwards to be positive)
    $$v'(t) = g - \frac{\alpha}{m}v(t), v(t) > 0$$
    5. Simplified Model of Climate Change
    $$\dv{C}{t} = E - Q - \gamma PC$$
\end{quote}
\begin{definition}
    An ODE is \textbf{\textit{autonomous}} if we have a relationship of
    $$\dv{y}{t} = f(y)$$
    such that there is no explicit $t$ term.
\end{definition}
In this case, an Initial Value Problem (IVP)(初值问题) is a type of problem where we are given an ODE, and some initial condition, such that we can find a unique solution to the ODE. \\
\begin{theorem}
    The \textbf{Existence/Uniqueness Theorem} of ODE (in Brian's words) states: \\
    If $f(y, t)$ is "really, really, really" nice, then the initial value problem has a unique solution
\end{theorem}
Notation-wise, we use
$$\int f(x) \dd x$$
to denote a general anti-derivative or an indefinite integral of $f(x)$.

\subsection{Variable Separable, 分离变量法}
A variable separable ODE is in the form of
$$\dv{y}{t} = f(y)g(t)$$
if we implicit define $y=y(t)$. \\
In general, we can solve this type of ODE with the following process
\begin{align*}
    \dv{y}{t} &= f(y)g(t) \\
    \int \frac{\dd y}{f(y)} &= \int g(t) \dd t \\
    F(y) &= G(t) + C
\end{align*}
\begin{theorem}
    The graph of $2$ different solutions to an ODE, \\
    $y_1(t)$ and $y_2(t)$ cannot intersect, \\
    that is,
    $$\forall t \in \mathscr{D}_y, y_1(t) \ne y_2(t)$$
    if $y_1(t)$ and $y_2(t)$ are two different solutions to an ODE.
\end{theorem}
\newpage
\subsection{Euler's Method, 欧拉法}
Euler's method is used to numerically approximate the solution of an ODE using small steps with an initial value. It is mostly used for ODEs that do not have a direct and easy to solve. \\
Consider the following process: \\
Let $\dv{y}{t} = f(y, t)$ be an ODE of interest, with an initial value $(t_0, y_0)$. Let a step size to be $\Delta t$. Then we can have the following approximation process:
\begin{align*}
    y_1 &= y_0^{'} \Delta t + y_0 = f(y_0, t_0) \Delta t + y_0 \\
    y_2 &= y_1^{'} \Delta t + y_1 = f(y_1, t_1) \Delta t + y_1 \\
    \vdots \\
    t_i &= t_0 + i \Delta t
\end{align*}
\subsubsection{Error Analysis, 误差分析}
Since in general, Euler's method has
$$y(t_{i+1}) = y(t_i + \Delta t) \approx y(t_i) + y'(t_i)\Delta t$$
Then, the error of this approximation is
$$\epsilon = \frac{1}{2}y''(t_i)\Delta t^2$$
For each step, the error is of order $2$, however, if we consider all the steps taken to approximate to a certain point would be of order $1$, as
$$\epsilon \cdot \frac{t-t_0}{\Delta t} \approx \Delta t$$

\subsection{Concavity, 函数的凹凸性}
We have previously discussed differentiability and what it means to a function. Here, we continue to investigate about twice differentiability. \\
Let $f$ be a twice differentiable function, we then have the $4$ following interpretations:
\begin{quote}
    1. $f''(x)$ is the instantaneous rate of change of $f'(x)$ \\
    2. $f''(x)$ is the rate of change of the slope of the tangent line to the graph of $f$ at $(x, f(x))$ \\
    3. If $f'' > 0$ on $(a, b)$, then $f'(x)$ is strictly increasing on $(a, b)$ \\
    4. If $f'' < 0$ on $(a, b)$, then $f'(x)$ is strictly decreasing on $(a, b)$
\end{quote}
\begin{theorem}
    Let $f(x)$ be twice differentiable on $(a, b)$, \\
    if $f'(x)$ is strictly increasing on $(a, b)$, then the graph $y=f(x)$ lies above its tangent lines; \\
    the inverse also holds.
\end{theorem}
\begin{definition}
    A twice differentiable function $f$ is: \\
    \textbf{concave up}(向上凹、凸函数) on an interval if $f'$ is strictly increasing on that interval; \\
    \textbf{concave down}(向下凹,凹函数) on an interval if $f'$ is strictly decreasing on that interval.
\end{definition}
\begin{theorem}
    The concavity of a function has the following properties: \\
    $f''(x)>0 \implies$ concave up $\implies$ graph of $f$ lies above its tangent lines; \\
    $f''(x)<0 \implies$ concave down $\implies$ graph of $f$ lies below its tangent lines.
\end{theorem}
\begin{definition}
    A point $(c, f(c))$ is an \textbf{inflection point}(拐点) if: \\
    The concavity of the function $f$ changes before and after $x=c$.
\end{definition}
\begin{proposition}
    If $(c, f(c))$ is an inflection point, then $f''(c) = 0$ or $f''(c)$ DNE
\end{proposition}

\subsection{Qualitative Properties of ODEs, 常微分方程定性性质}
\begin{definition}
    The \textbf{slope field}(斜率场)of a $1^{st}$ order ODE $y'=f(y, t)$ is, \\
    a diagram in $t-y$ plane, at the point $(t, y)$, draw line segment of slope $f(y, t)$. \\
    A solution to the ODE is a curve tangent to the slope field.
\end{definition}
\begin{definition}
    An \textbf{equilibrium solution}(平衡解)to an autonomous $1^{st}$ order ODE ($y'=f(y)$) is a constant solution ($y(t) = c$)
\end{definition}
\begin{definition}
    An equilibrium solution is: \\
    \textbf{stable}, if $\forall y_0$ sufficiently close to $c$, the solution $y(t)$ with $y(t_0) = y_0$ satisfies
    $$\lim_{t \to +\infty} y(t) = c$$
    \textbf{non-stable}, if $\forall y_0$ sufficiently close to $c$, the solution $y(t)$ with $y(t_0) = y_0$ has either
    $$\lim_{t \to +\infty} y(t) \ne c$$
    or
    $$\lim_{t \to +\infty} y(t) = DNE$$
\end{definition}
\newpage
\section{Properties of Functions, 函数的性质}

\subsection{Limits at Infinity and Horizontal Asymptotes, 在无限处的极限与水平渐近线}
We discuss a function's long-run behaviour by using limits "at" (approaches) infinity. \\
Case 1:
\begin{quote}
    $$\lim_{x \to +\infty} c = c$$
    $$\lim_{x \to \pm \infty} \frac{1}{x} = 0$$
    If $n > 0$,
    $$\lim_{x \to +\infty} \frac{1}{x^n} = 0$$
    If $n = \frac{k}{r}$, where $r > 0$ and $r$ is not even,
    $$\lim_{x \to -\infty} \frac{1}{x^n} = 0$$
\end{quote}
In this case,
\begin{definition}
    The line $y=L$ is a \textbf{horizontal asymptote} if: \\
    $$\lim_{x \to \infty} f(x) = L$$
    or
    $$\lim_{x \to -\infty} f(x) = L$$
\end{definition}
Case 2: A function has an infinite limit as $x$ approaches infinity. For example,
$$\lim_{x \to +\infty} e^x = \infty$$
$$\lim_{x \to +\infty} \ln{x} = \infty$$
However, what if limit laws do not apply? For example,
$$\lim_{x \to +\infty} xe^x = ?$$
$$\lim_{x \to +\infty} x\ln{x} = ?$$

\subsection{Indeterminate Forms and L'Hôpital's Rule, 不定式与洛必达法则}
Indeterminate forms are forms that if we directly plug in $\infty$ for our limit, we have the form
$$\frac{0}{0}, \frac{\infty}{\infty}, 0 \times \infty, \infty - \infty, 0^0, 1^\infty, \infty^0$$
But $\infty \times \infty$ is not an indeterminate form, as this limit will always diverge to $\infty$. \\
For $\frac{0}{0}$ we usually do some algebraic manipulations, for example, multiplying the numerator and the denominator with the denominator's conjugate; for $\frac{\infty}{\infty}$ and $\infty - \infty$, we then need to compare infinities. \\
A side note is that a polynomial function always has its highest-degree term that dominates the growth of the function.
\begin{theorem}
    The L'Hôpital's Rule states that: \\
    Let $f$ and $g$ be differentiable near $x=a$, with $g'(x) \ne 0$ (except possibly at $x=a$). If
    $$\lima f(x) = \lima g(x) = 0$$
    or
    $$\lima f(x) = \lima g(x) = \infty$$
    then
    $$\lima \frac{f(x)}{g(x)} = \lima \frac{f'(x)}{g'(x)}$$
\end{theorem}
For indeterminate powers,
$$f(x)^{g(x)} = e^{g(x) \ln{f(x)}}$$
we then evaluate the limit only for the part $g(x) \ln{f(x)}$.\\
When encountering $0 \times \infty$, we can divide the reciprocal of one term to convert to $\frac{0}{0}$ or $\frac{\infty}{\infty}$ forms.

\subsection{Local Extrema, 极值点}
Critical points are where $f$ may attain a local extremum.
\begin{theorem}
    Fermat's Theorem states that: \\
    If $f$ has a $\max$ or $\min$ at $x=c$ and $f'(c)$ exists, then $f'(c) = 0$, we call $(c, f(c))$ a critical point. \\
    More generally, for $f'(c)$ DNE, we call it a critical/singular point.
\end{theorem}
To determine whether $f(c)$ is a local extremum or not, we may use the first derivative test:
\begin{quote}
    1. If $f'(c) > 0$ for $x < c$ and $f'(c) < 0$ for $x > c$, the $(c, f(c))$ is a local maximum; \\
    2. If $f'(c) < 0$ for $x < c$ and $f'(c) > 0$ for $x > c$, the $(c, f(c))$ is a local minimum.
\end{quote}

\subsection{Shape of a graph, 函数的形状}
To determine the shape of a graph, we follow the following steps:
\begin{quote}
    1. Find the domain and range of the function. \\
    2. How the function behaves around discontinuities (e.g. vertical asymptotes) \\
    3. How the function behaves in the long run (e.g. $\infty$ and $-\infty$) \\
    4. How the function behaves around an oblique asymptote
    5. Function's increasing and decreasing behaviours from the $1^{st}$ derivative. \\
    6. Function's concavity from the $2^{nd}$ derivative. \\
    7. Function symmetry (even or odd functions)
\end{quote}

\subsection{Global extrema, 最值点}
\begin{definition}
    Define $f(x)$ on $[a, b]$. Consider $c \in [a, b]$, we say $f(x)$ has: \\
    1. A local maximum at $x=c$ if $f(x) \le f(c)$, $\forall x \in (a, b)$ \\
    2. A local minimum at $x=c$ if $f(x) \ge f(c)$, $\forall x \in (a, b)$ \\
    3. A global maximum at $x=c$ if $f(x) \le f(c)$, $\forall x \in [a, b]$ \\
    4. A global minimum at $x=c$ if $f(x) \ge f(c)$, $\forall x \in [a, b]$
\end{definition}
Thus, to determine a function's global extrema, we follow these steps:
\begin{quote}
    1. Find the domain of the function \\
    2. Find the critical points in the domain \\
    3. Compare the $y$-values at the critical points and the endpoints \\
    4. If the domain is unbounded, compare with limits as $x$ approaches $\infty$
\end{quote}

\subsection{Optimization, 最优化}
Optimization is to apply the previous process of finding extrema by setting up proper equations and evaluating accordingly.
\begin{theorem}
    The Extreme Value Theorem states that: \\
    If $f(x)$ is defined and continuous on $[a, b]$, then $f$ attains a maximum and a minimum at least once. \\
    Therefore, if $f(x)$ has a global maximum or minimum at $x = c \in [a, b]$, then we have four possible situations
    $$f'(c) = 0$$
    $$f'(c) = DNE$$
    $$c = a$$
    $$c = b$$
\end{theorem}


\newpage
\section{Taylor Polynomials, 泰勒展开与泰勒多项式}

\subsection{Taylor Polynomials and Maclaurin Polynomials, 泰勒多项式与麦克劳林多项式}
The idea here is to approximate the function around a point using polynomials. \\
The polynomial will have a better and better approximation of the function as we introduce higher and higher degrees of terms. \\
A constant approximation approximates based on the same $y$-value
A linear approximation approximates based on the same slope. \\
A quadratic approximation approximates based on the same concavity. \\
Thus, if we start with an arbitrary polynomial around $x=a$,
$$P(x) = C_0 + C_1(x-a) + C_2(x-a)^2 + \dots$$
Then we need the following rules
\begin{align*}
    P(a) &= f(a) \\
    P'(a) &= f'(a) \\
    P''(a) &= f''(a) \\
    \vdots
\end{align*}
Therefore, we need
\begin{align*}
    C_0 &= f(a) \\
    C_1 &= f'(a) \\
    2C_2 &= f''(a) \\
    \vdots
\end{align*}
Then, we will get a general case where
$$C_n = \frac{f^{(n)}(a)}{n!}$$
This will give us an $n$-th degree Taylor polynomial around $x=a$:
$$T_n(x) = \sum_{k = 1}^n \frac{f^{(k)}(a)}{k!}(x-a)^k$$
We call $a$ to be the \textbf{center of approximation} and $n$ to be the \textbf{order of approximation}. \\
When $a=0$, a Taylor polynomial is also called a Maclaurin polynomial.
\newpage

\subsection{Error Determination, 误差}
An approximation will always have a certain error, for an $n$-th degree Taylor polynomial, we define its approximation error to be
$$R_n(x) = \frac{f^{(n+1)}(c)}{(n+1)!}(x-a)^{n+1}$$
This is the \textbf{Lagrange Remainder Formula}, for some $c \in [a, x]$. \\
Since $c$ is an arbitrary number in that given interval when we determine the error of an $n$-th degree Taylor polynomial, instead of finding an exact value for such an error, we usually find an upper bound for the error.

\subsection{Taylor Polynomial Applications, 泰勒多项式的应用}
With Taylor polynomials, we can use it to find limits of certain functions that are hard to evaluate on the spot. As we transform some part of the function into a Taylor polynomial (technically a Taylor series), we eventually will be working with polynomials. \\
On the other hand, it is also critical to find a proper $n$ for the degree of the Taylor polynomial so that the error of approximation is below a certain value. \\
By this, if we are given an upper bound of error, $\epsilon$, we need to find the proper $n$, such that
$$R_n(x) = \frac{f^{(n+1)}(c)}{(n+1)!}(x-a)^{n+1} < \epsilon$$
In other subjects, we may also use Taylor's polynomial to confirm Newton's Laws under the scope of special relativity. That is, when an object's velocity is significantly smaller than the speed of light.

\end{document}
