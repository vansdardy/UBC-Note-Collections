\section{Spectral Theorem, 谱定理}
\subsection{Eigenvalues and Eigenvectors, 特征值与特征向量}
\begin{definition}
    Let $A$ be a matrix with $a_{ij} \in \F$ and $A: \F^n \to \F^n$, then $\lambda \in \F$ is an eigenvalue of $A$ if
    $$\exists v \ne 0, v \in \F^n, Av = \lambda v$$
    $v$ is the eigenvector corresponding to the eigenvalue $\lambda$
\end{definition}
To find eigenvalues of a matrix $A$, we find the \textbf{\textit{characteristic polynomial}}(特征多项式):
$$\det (A - \lambda \id) = 0$$
Then, to find the eigenvectors corresponding to eigenvalue $\lambda$, we just have to find a basis for $\Ker (A - \lambda \id)$.
\subsubsection{Basis of Eigenvectors, 特征向量的基}
In order for a basis of eigenvectors to exist, we need $A$ to at least be an endomorphism $A: V \to V$, then $V$ \textbf{might} have a basis of eigenvectors of $A$. \\
The matrix $A$ with respect to the eigenvector basis is \textbf{\textit{diagonal}}(对角的). \\
We need eigenvalues to determine whether or not a basis of eigenvectors exists. Therefore, consider the characteristic polynomial
$$\begin{vmatrix}
    a_{11} - \lambda & a_{12} & \dots & a_{1n} \\
    a_{21} & a_{22} - \lambda & \dots & a_{2n} \\
    \vdots & \vdots & & \vdots \\
    a_{n1} & a_{n2} & \dots & a_{nn} - \lambda
\end{vmatrix}$$
which must have the leading term to be some $(-1)^n \lambda^n$. \\
By \textbf{Fundamental Theorem of Algebra}(代数基本定理), an $n$-degree polynomial must have $n$ roots in $\C$, counting with \textbf{\textit{algebraic multiplicity}}(代数重复度). \\
Over $\C$, any polynomial can be factored completely; but over $\R$, the complex roots will come in pairs $\{z, \overline{z}\}$, that is, odd-degree polynomials must have odd number of real roots. \\
In the complex plane, multiplying by $\lambda = \alpha \pm \beta i = |\lambda|e^{i\phi}$, means rotating $\phi$, scaling by $|\lambda| = \sqrt{a^2 + b^2}$, where $\sin{\phi} = \frac{a}{|\lambda|}, \cos{\phi} = \frac{b}{|\lambda|}$.
\subsubsection{Linear Independence of Eigenvectors, 特征向量的线性独立性}
Suppose $\{v_i\}$ is a set of eigenvectors corresponding to distinct eigenvalues $\lambda_i$, then,
\begin{quote}
    Base: $i = 1$, $\{v_1\}$ is linearly independent, \\
    Inductive Hypothesis: Assume for $i = k$ \\
    Inductive Step: we show for $i = k + 1$, \\
    suppose $\alpha_1 v_1 + \dots + \alpha_{k+1} v_{k+1} = 0$ \\
    Then 
    $$A(\alpha_1 v_1 + \dots + \alpha_{k+1} v_{k+1}) = \lambda_1 \alpha_1 v_1 + \dots + \lambda_{k+1} \alpha_{k+1} v_{k+1}$$
    Use this minus $\lambda_{k+1}$ of supposition, we have
    $$\alpha_1(\lambda_1 - \lambda_{k+1})v_1 + \dots + \alpha_k(\lambda_k - \lambda_{k+1})v_k = 0$$
    Since $\lambda_1 \ne \dots \ne \lambda_{k+1}$, then $\alpha_1 = \dots = \alpha_k = 0$, which further indicates $\alpha_{k+1} = 0$ \\
    $\{v_{k+1}\}$ is thus linearly independent.
\end{quote}

\subsection{Matrix Diagonalization, 矩阵对角化}
Over $\C$, consider a linear operator in $\C^n$ with distinct eigenvalues $\lambda_1, \dots, \lambda_n$, then the matrix of $A$ in this corresponding basis of eigenvectors is \textbf{diagonal}. \\
\subsubsection{Change-of-basis Matrix, 基变更矩阵}
Consider in $\R^2$, there is a basis $\{v_1, v_2\}$ and the standard basis $\{e_1, e_2\}$. \\
Let $v_1 = c_{11} e_1 + c_{21} e_2$, and $v_2 = c_{12} e_1 + c_{22} e_2$, then a change-of-basis matrix $C$, is thus
$$C = \begin{pmatrix}
    c_{11} & c_{12} \\
    c_{21} & c_{22}
\end{pmatrix}$$
Then, consider a general vector $x \in V$. We then have
$$x_{e_1, e_2} = C x_{v_1, v_2}$$
$$x_{v_1, v_2} = C^{-1} x_{e_1, e_2}$$
If we have a linear operator $A$ with respect to the basis $\{e_1, e_2\}$, and a linear operator $A'$ with respect to the basis $\{v_1, v_2\}$, then we should have
$$(A x_{e_1, e_2})_{v_1, v_2} = A'x_{v_1, v_2}$$
Therefore,
\begin{align*}
    (A x_{e_1, e_2})_{v_1, v_2} &= A'x_{v_1, v_2} \\
    C^{-1}A x_{e_1, e_2} &= A'C^{-1} x_{e_1, e_2} \\
    C^{-1}A &= A'C^{-1} \\
    C^{-1}AC &= A'
\end{align*}
\subsubsection{Complex Eigenvectors, 复特征向量}
If $A$ have complex eigenvectors, then over $\R$, we can bring $A$ to the form:
$$A = \begin{pmatrix}
    X & 0 & 0 \\
    0 & Y & 0 \\
    0 & 0 & Z
\end{pmatrix}$$
where each $X, Y, Z$ is a $2 \times 2$ matrix, with the form
$$\begin{pmatrix}
    |\lambda|\cos{\phi} & -|\lambda|\sin{\phi} \\
    |\lambda|\sin{\phi} & |\lambda|\cos{\phi}
\end{pmatrix}$$
Recall: $|\lambda|e^{i \phi} = |\lambda|\cos{\phi} + |\lambda|i\sin{\phi}$
\subsubsection{Non-distinct Eigenvalues, 非独特特征值}
Since we know $\det A' = \det (C^{-1}AC) = \lambda_{1}\dots \lambda_{n}$, thus:
\begin{quote}
    $\det A' \iff \exists i, \lambda_i = 0$ \\
    This means, with $0$ to be an eigenvalue, we have $\exists v \ne 0, Av = 0$, that is $\Ker A \ne \{0\}$. Then, $A$ is neither injective, surjective, nor invertible.
\end{quote}
\begin{definition}
    Define \textbf{\textit{geometric multiplicity}}(几何重复度) of eigenvalue $\lambda$ to be
    $$\dim (\Ker (A - \lambda \id))$$
    which is less than or equal to its algebraic multiplicity.
\end{definition}
When the geometric multiplicity of $\lambda$ is less than its algebraic multiplicity, then $A$ cannot be diagonalized. \\
The best we can do is to find $C$, such that $C^{-1}AC$ is of Jordan normal form.
\begin{definition}
    An $n \times n$ Jordan block is of the form
    $$J_\lambda = \begin{pmatrix}
        \lambda & 1 & 0 & \dots & 0 \\
        0 & \lambda & 1 & \dots & 0 \\
        \vdots & \vdots & \vdots & & \vdots \\
        0 & 0 & 0 & \dots & \lambda
    \end{pmatrix}$$
    One Jordan block represents the existence of $1$ eigenvector. \\
    Let
    $$J_0 = \begin{pmatrix}
        0 & 1 & 0 & \dots & 0 & 0 \\
        0 & 0 & 1 & \dots & 0 & 0 \\
        \vdots & \vdots & \vdots & & \vdots & \vdots \\
        0 & 0 & 0 & \dots & 0 & 1 \\
        0 & 0 & 0 & \dots & 0 & 0 \\
    \end{pmatrix}$$
    Then, every $J_\lambda = \lambda \id + J_0$. \\
    With $J_0^n = 0$, it is a nilpotent matrix.
\end{definition}
\begin{lemma}
    The binomial formula states that
    $$(J_\lambda)^n = (\lambda \id + J_0)^n$$
\end{lemma}
For each eigenvalue $\lambda$, we can write a Jordan form with Jordan blocks on the diagonal. The number of Jordan blocks (geometric multiplicity) is the number of eigenvectors associated with this eigenvalue. \\
Why do we care about diagonalization?
\begin{quote}
    We can find powers of $A$, with $D$ being $A$'s diagonalized matrix. \\
    $$A^m = CD^mC^{-1} = C \begin{pmatrix}
        \lambda_1^m & 0 & \dots & 0 \\
        0 & \lambda_2^m & \dots & 0 \\
        \vdots & \vdots & & \vdots \\
        0 & 0 & \dots & \lambda_n^m
    \end{pmatrix} C^{-1}$$
\end{quote}

\subsection{Self-adjoint Matrix, 自伴随矩阵}
\begin{definition}
    Let $A: V \to V$ be an endomorphism, where $V$ is a Euclidean vector space, \\
    $A$ is \textbf{\textit{self-adjoint}}(自伴随) if
    $$(Av, w) = (v, Aw)$$
\end{definition}
Therefore, when considering with respect to an orthonormal basis, the matrix $A$ is symmetric, where
$$A = A^t, a_{ij} = a_{ji}$$
This is true because $(Ae_i, e_j)$ represents $j^{th}$ component of $A$'s $i^{th}$ column ($a_{ji}$), and $(e_i, Ae_j)$ represents $i^{th}$ component of $A$'s $j^{th}$ column ($a_{ij}$).
\subsubsection{Spectral Theorem, 谱定理}
\begin{theorem}
    A self-adjoint linear operator has a basis of eigenvectors, and we can choose an \textbf{orthonormal basis} of eigenvectors. \\
    The process of finding the change-of-basis for self-adjoint linear operators is called the \textbf{\textit{Principal Axes Transformation}}.
\end{theorem}
This is:
\begin{quote}
    1. Eigenvectors corresponding to different eigenvalues are orthogonal. \\
    2. If there are "repeated eigenvalues", then part of the matrix corresponding to $\lambda$ must be $\lambda \id$.
\end{quote}
For a self-adjoint matrix, its eigenvalues must have the same geometric and algebraic multiplicity. \\
Thus, to find an orthonormal basis of eigenvectors (Principal Axes Transformation), we:
\begin{quote}
    1. Find the eigenvalue of $A$. \\
    2. For each $\lambda$, find its eigenvector(s). \\
    3. Within each $\lambda$, orthonormalize (Gram-Schimdt) the found eigenvectors. \\
    4. Write the orthonormalized eigenvectors in columns, which form the change-of-basis matrix $C$. \\
    5. By applying $C^tAC$, we can diagonalize $A$ where on the diagonal is $A$'s eigenvalues. \\
    5. Or, by applying $CDC^t$, we can find the linear transformation $A$ with respect to the standard basis.
\end{quote}
\begin{theorem}
    Spectral decomposition of self-adjoint operators: \\
    If $f: V \to V$ is self-adjoint of a finite-dimensional Euclidean vector space, and $\lambda_1, \dots, \lambda_r$ its distinct eigenvalues, and $P_k: V \to V$ the orthogonal projection onto the eigenspace $E_{\lambda_k}$, then
    $$f = \sum_{k=1}^r \lambda_k P_k$$
\end{theorem}