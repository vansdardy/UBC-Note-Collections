\section{Model}
Reiterating the goal of this paper, it investigates the impact of the number of diners of each dining hour on the amount of food waste generated with each day. After the selection of data, the table consists of the relevant variables for each dining location. Apart from the aforementioned assumption made, two more assumptions are made to set up this model.
\begin{quote}
    1. It is assumed that the food waste is primarily generated due to the number of diners. This assumption is made based upon how the waste data is categorized. Since the waste data breaks down the waste into different preparation and consumption stages, it is assumed that an increase in the number of diners would lead to an increase in waste during preparation stages as well. \\
    2. It is assumed that the number of diners across the semester is consistent and random, where the number of diners on one day is not affected by the day before and dining halls do not suffer significant increases and drops. This assumption incorporates three phenomena in the students. First, the students may go to other cafeterias as there is no restriction on which dining hall students can access. Second, most students go to the dining halls most closely located to their dormitories. Third, students may also choose to skip certain meals due to their timetables.
\end{quote}
Given the goal of the paper and these assumptions, it is possible to set up the following multiple linear regression to answer the research question:
$$W_i = \beta_0 + \beta_1 B_i + \beta_2 L_i + \beta_3 S_i + \epsilon_i$$
The relevant variables are labeled as follows:
\begin{quote}
    1. $W_i$: the total amount of food waste generated on the $i$th day in the dataset, it is assumed that this is measured in pounds (lb). \\
    2. $B_i$: the number of breakfast diners on the $i$th day. \\
    3. $L_i$: the number of lunch diners on the $i$th day. \\
    4. $S_i$: the number of dinner diners on the $i$th day.
\end{quote}
Notice that the same regression model will be applied to each of the dining halls to observer the effect.

By the setup of this regression, $\beta_1$ represents the change in daily food waste subject to $1$-unit increase in the breakfast diners, $\beta_2$ represents the change in daily food waste subject to $1$-unit increase in the lunch diners, and $\beta_3$ is a similar parameter for $1$-unit increase in the dinner diners. Here $\beta_0$ plays the role of representing all the food waste generated by all other sources apart from the number of diners.

\subsection{Specifications}
The main specification from this regression model investigates if the number of diners of at least one dining hour has an impact on the daily food waste. This is characterized by investigating if all the parametric coefficients of the number of diners is $0$. Thus, we set up the main null hypothesis:
$$H_0^{(0)}: \beta_1 = \beta_2 = \beta_3 = 0$$
Three other specifications are also introduced. The main specification only answers if there is an impact, but does not answer which dining hour actually contributes to this consequence. Therefore, it is also motivated to investigate if the parametric coefficients have an impact on the daily food waste independently. Thus, we can set up three more null hypotheses:
$$H_0^{(1)}: \beta_1 = 0$$
$$H_0^{(2)}: \beta_2 = 0$$
$$H_0^{(3)}: \beta_3 = 0$$
To test the robustness of this model, that is, if the variables are grouped differently, would there be a change on the conclusion reached. In this case, it is possible to set up another simple linear regression:
$$W_i = \alpha_0 + \alpha_1 C_i + \eta_i$$
where $W_i$ represents the same thing, but $C_i$ represents the total number of diners on the $i$th day.

Here, it is possible to test if the total diners have an impact on the food waste as well, aligning with the research question. Thus, the null hypothesis for this can be set up as:
$$H_0^{'}: \alpha_1 = 0$$

\newpage