\section{Introduction}
In April 2022, after the Okanagan campus of the University of British Columbia implemented an all-access dining system for their students, the Vancouver campus announced that it will also adopt a similar strategy for dining halls located in primarity first-year residences to promote food security and reduce waste \citep{introduce_all_access}. This dining plan allow students with meal plans to swipe their student card once and enjoy all types of food instead of paying for individual items \citep{ubcallaccessdining}. Students who have not signed up for a meal plan may also access the dining halls by paying between $\$$12 to $\$$20 before tax for a meal depending on the dining hours \citep{ubcallaccessdining}. The food service at UBC Vancouver suggests that this move helps to reduce food waste because food items are served in limited portions and students are encouraged to take multiple trips if they do not feel fulfilled \citep{ubcallaccessdining}. It is believed that this can prevent students from taking too much food all at once which may eventually be dumped because students overestimate their appetite.

According the sustainability goal set by UBC, reducing food waste helps to build a more sustainable food system that ensures local food security. Furthermore, as UBC invests more and more in the effort to recycle, reducing food waste also decreases the strain on recycling in general, where the same resource can be used to recycle other items.

Given this environmental initiative, this paper aims to investigate if the all-access dining plan introduced at the UBC Vancouver campus aligns with this sustainability goal of reducing food waste. Specifically, this paper hopes to unfold the relationship between the number of diners and the amount of food waste generated each day. With the data provided by UBC food service, it is possible to answer the question - \textbf{how does the number of diners of different dining hours impact the amount of food waste generated within each day across three food service facilities?} Here the facilities refer to the cafeteria located in the first-year residences, which are Totem Park (\textit{abbr. TP}), Orchard Commons (\textit{abbr. OC}), and Place Vanier (\textit{abbr. PV}). The provided dataset can answer this question as it includes all the food waste data and diners at each location at different hours for the first semester of the 2023-2024 school year.

Although the dataset suffers from heteroskedasticity, it is still hypothesized that for all three locations, the number of diners during at least one dining hours will have an impact on the amount of food waste generated in each day. However, \textbf{this paper concludes that} for the dining halls at Totem Park and Place Vanier, the number of diners have little to no impact on the food waste produced at these locations, while at Orchard Commons, the number of breakfast diners is negatively correlated with the amount of food waste and the number of lunch diners is positively correlated with the amount of food waste across days.

Given this conclusion, it is recommended that further measures can be taken to reduce food waste focusing on the dinner plates at Place Vanier and on lunch plates at Orchard Commons.
\newpage