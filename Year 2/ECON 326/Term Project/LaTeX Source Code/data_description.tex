\section{Data Description}
In the dataset provided by the UBC Food Service, it mainly consists of two types of data \citep{food_waste}. The first of which compiles all the menu option on the days that dining halls are accessible to students, and the second of which offers the data on the number of diners and the amount of food waste generated from preparation to consumption across different dining hours at three dining halls. Since the focus is to determine the relationship between the number of diners and the amount of food waste, the first type of data is mostly irrelevant to the question of interest. However, it should be mentioned that the dining halls rotate their menu options, thus, one of the assumptions made for this research is that \textbf{diners generally have consistent preferences so that leftovers are primarily generated by students unable to finish the dish rather than disliking the taste.}

Within the relevant dataset, which is the second type, there are several key issues to be resolved. 

The first one occurs in the recorded food waste data. It can be observed across all three dining halls, there are days where part of the process is missing a record, or during some days, it appears that the dinner hours included the previous hours whereas these hours are missing their data. Thus, a decision is made to only focus on all the waste generated within each day, disregarding specific processes and specific dining hours to account for missing records and unreasonable entries resulted from whatever reasons.

The second one occurs in the recorded diner numbers. It is also observed that there are days where the number of diners are not recorded for the specific dining hour. As it is critical for this research to look for the relationship across each dining hour, any data entry with at least one of the dining hours missing diner numbers is eliminated from the final summarized table.

After these selections, there are $89$ reasonable observations recorded at Totem Park, $96$ observations recorded at Orchard Commons, and $85$ observations recorded at Place Vanier. Each of the processed tables consists of the date of entry, the dining hall location, the number of diners for each dining hour, the total number of diners in a day, and the total waste in a day. The graphs of these relevant information is included in the Appendix.

\subsection{Summary Statistics Table}
To generally describe these processed data, a summary statistics table is included to provide basic interpretations of the variables. The relevant statistics include the mean, the standard deviation, and the maximum values of each data column in the table. Some general characteristics include: across all three locations, there are generally more lunch diners and dinner diners than breakfast diners; Orchard Commons appears to be a more popular choice for students than the other two locations; the average food waste across all three locations are between $300$ and $400$ units.

\begin{table}
    \captionsetup{justification=justified,singlelinecheck=false,margin=0pt}
    \caption{Mean, Standard Deviation and Maximum for Relevant Data Columns across Dining Halls}
    \centering
    \mini
    \begin{tblr}{
      cell{1}{1} = {c=2}{},
      cell{2}{1} = {r=3}{},
      cell{5}{1} = {r=3}{},
      cell{8}{1} = {r=3}{},
      vlines,
      hline{1-2,5,8,11} = {-}{},
      hline{3-4,6-7,9-10} = {2-7}{},
    }
                    &      & Breakfast & Lunch     & Dinner    & Diner Total & Waste Total \\
    Totem Park      & Mean & 693.2022  & 1018.7416 & 1109.8539 & 2821.7978   & 399.3076    \\
                    & SD   & 386.6350  & 248.6930  & 388.3880  & 524.6679    & 344.7666    \\
                    & Max  & 1797.00   & 1806.00   & 1884.00   & 3694.00     & 3208.01     \\
    Orchard Commons & Mean & 753.4167  & 1292.0625 & 1456.3333 & 3501.8125   & 314.0145    \\
                    & SD   & 279.6764  & 329.5889  & 342.0098  & 884.5424    & 127.8332    \\
                    & Max  & 1298.00   & 1794.00   & 2093.00   & 4496.00     & 740.68      \\
    Place Vanier    & Mean & 739.8235  & 1014.8824 & 910.0824  & 2664.7882   & 396.0318    \\
                    & SD   & 344.5338  & 327.6320  & 353.1179  & 731.2697    & 241.4976    \\
                    & Max  & 1372.00   & 1513.00   & 1468.00   & 3650.00     & 1659.90
    \end{tblr} \\
    \medskip
    \normalsize
    \raggedright The unit for food waste is never clarified by the data provider, it is assumed that the food waste is measured in pounds (lb).
\end{table}