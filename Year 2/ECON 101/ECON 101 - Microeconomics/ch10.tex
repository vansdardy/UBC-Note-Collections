\section{Monopoly, 垄断市场}
\subsection{Theory of Single-price Monopoly, 单一价格垄断}
There are $3$ characteristics of a single-price monopoly structure:
\begin{quote}
    1. One seller \\
    2. Selling a unique good \\
    3. Impossible to enter or exit
\end{quote}
In this case, the firm is a \textbf{price setter}.
\subsubsection{Demand curve, 需求曲线}
Since the firm is the industry, the demand curve will be \textbf{downward sloping}. \\
The firm will then set a price to maximize profits. What price will the firm sell its goods at? \\
Firstly, a monopolist will never produce when $\epsilon < 1$, as this is equivalent to $MR < 0$, leading to profit loss. \\
Secondly, we know $P = AR = a - bQ$, then $TR = PQ = aQ - bQ^2$, so
$$MR = \dv{TR}{Q} = a - 2bQ$$
giving us the marginal revenue drop decreasing twice as fast as the demand curve.

\subsection{Short Run Profit Maximization, 短期利润最大化}
A monopoly maximizes its profits by producing when $\epsilon \ge 1$, it stays in business ($P \ge AVC$), and where $MR = MC$. \\
Case 1: Economic Profit
\begin{quote}
    1. $MR = MC$, find $Q$ at intersection. \\
    2. Find $P$ on the demand curve given the above $Q$. \\
    3. $P > AC$
\end{quote}
Case 2: Normal Profit
\begin{quote}
    1. $MR = MC$, find $P$ and $Q$ at intersection. \\
    2. Find $P$ on the demand curve given the above $Q$. \\
    3. $P = AC$
\end{quote}
Case 3: Economic Loss
\begin{quote}
    1. $MR = MC$, find $P$ and $Q$ at intersection. \\
    2. Find $P$ on the demand curve given the above $Q$. \\
    3. $AVC < P < AC$
\end{quote}
For a monopoly, when not producing at profit-maximization outputs, it does not imply an economic loss. Therefore, monopolies have the flexibility to trade off some profit for policies.

\subsection{Long Run and Barriers to Entry, 长期生产与参与阻碍}
\subsubsection{Barriers to Entry}
Profits still signal firms to enter a monopoly, so the monopoly must \textbf{impede} this incentive. \\
A \textbf{barrier to entry} impedes the entry of new firms into the market. \\
BTE will thus allow a monopoly to have long run \textbf{economic profits}. \\
The natural BTE includes:
\begin{quote}
    1. High startup costs \\
    2. Economies of Scale, Scope, Network
    \begin{quote}
        Natural Monopoly: occurs when demand is less than MES \\
        Economies of scale: LRAC falls as the firm produces more of the \textbf{same} good \\
        Economies of scope: LRAC falls as the firm produces \textbf{more than one} good \\
        Network economies: a product's value increases as more people use the product
    \end{quote}
\end{quote}
The artificial BTE includes:
\begin{quote}
    1. Government
    \begin{quote}
        Patents, franchises, charters, licenses, environmental regulations, red tape, procurement policies
    \end{quote}
    2. Firm
    \begin{quote}
        Predatory pricing, product differentiation
    \end{quote}
\end{quote}
In the long run, monopolies reach productive efficiency, but NOT allocative efficiency as there exists DWSL. \\
In the very long run, monopolies rarely persist as new ideas are created to "destroy" old ideas (creative destruction), creating economic growth, UNLESS, it is protected by the government.

\subsection{Cartels, 垄断集团}
A \textbf{cartel} is a voluntary association of producers who agree to act as a monopoly to maximize \textbf{joint profits}. \\
For example, OPEC is an oil cartel, whose joint decisions led to the oil embargo in 1972 and 1979. \\
A cartel can form through either formal collusion (formal agreements made) or tacit collusion (no formal agreements made).

\subsection{Theory of Multiprice Monopoly, 多价垄断理论}
\textbf{Price discrimination} is the \textbf{same} producer charges \textbf{different prices} for the \textbf{same} good (due to elasticity), \textbf{for reasons other than costs}. \\
The conditions for price discrimination are:
\begin{quote}
    1. Monopoly power: the seller can charge different prices \\
    2. Consumers must value different units of the same product differently \\
    3. No arbitrage (consumers buy at a low price and \textbf{resell} at a higher price), so that the buyers cannot defeat the seller's objective.
\end{quote}
An advantage of general price discrimination is that a part of the consumer's surplus is now converted to the producer's surplus. Then, a perfect price-discriminating monopoly would produce up to $Q$ where $D = MC$. In this case, the consumer surplus triangle is transformed into monopoly profits, and DWSL disappears, reaching allocative efficiency. \\
There are three types of price discrimination:
\begin{quote}
    1. First degree (Perfect price discrimination) = charging the reservation price \\
    2. Second degree = charging several different prices \\
    3. Third degree = charging different groups
\end{quote}
Price discrimination can result in the following ways:
\begin{quote}
    1. Higher profits for the monopoly \\
    2. Higher output from the monopoly \\
    3. Normative effects
    \begin{quote}
        The consumer surplus is transferred to the producer surplus, and the total surplus remains the same.
    \end{quote}
\end{quote}
Another form of price discrimination is \textbf{hurdle pricing}, which is the monopolist segmenting the market by charging reservation prices for price-sensitive buyers and the monopoly price for the rest.
\newpage