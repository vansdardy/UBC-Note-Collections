\section{Government Intervention, 政府干预}
\subsection{Basic functions of government}
\textbf{Government} must have \textbf{monopoly of violence} through the military and the courts to deprive liberty. It must define and enforce "Property Rights".

\subsection{Case for the free market}
1. Free market can reach allocative efficiency
\begin{quote}
    Problem: Most markets are imperfectly competitive
\end{quote}
2. Free market coordinates automatically through the price system. \\
3. It stimulates innovation and economic growth. \\
4. It decentralizes power.

\subsection{Case for government intervention}
\subsubsection{Monopoly power}
It is inevitable because:
\begin{quote}
    Economies of scale, scope, networks \\
    Differentiated products \\
    Innovation
\end{quote}
It leads to allocative inefficiency, so the government intervenes with economic policy to balance it.
\subsubsection{Externalities, 外部性}
\textbf{Externality} is non-priced (excluded from the original D and S analysis) third-party cost or benefit. \\
A production externality affects supply, and a consumption externality affects demand. \\
A positive externality is an external economy (increase in S or D), and a negative externality is an external diseconomy (decrease in S or D). \\
\\
\textbf{Private cost} is an opportunity cost to the seller (seller: first-party, buyer: second-party). \\
\textbf{External cost} is the opportunity cost of third-party. \\
\textbf{Social cost} is private cost $+$ third-party opportunity cost to society. \\
\\
\textbf{Network externality} is the effect of an extra user on others. \\
\textbf{Pecurniary externality} is not an externality: I buy an iPhone, the price to third-party rises. \\
\\
To determine social underproduction or overproduction, we first assume perfect competition and no externalities. Then, apply the corresponding externalities to shift S or D. Compare the final $Q_E$ with the original $Q_E$. \\
\\
Some common applications of externalities include
\begin{quote}
    1. Environmental pollution \\
    2. Open access resources \\
    3. Highway congestions
\end{quote}

\subsubsection{Public goods}
\textbf{Public goods} are characterized by non-rivalry (consumption of one does not diminish consumption by another) and non-excludability (if produced, it must be consumed equally by all). \\
By such categorization, we have:
\begin{quote}
    1. Private good: rivalrous and excludable, e.g. chocolate \\
    2. Club good: non-rivalrous and excludable, e.g. art gallery, roads \\
    3. Common property good: rivalrous and non-excludable, e.g. fishery \\
    4. Public good: non-rivalrous and non-excludable, e.g. national defence
\end{quote}
The problem with club good:
\begin{quote}
    1. Non-rivalry: $MC = 0$ to produce, so $P = 0$ for efficiency; \\
    2. Excludability: allows market pricing, so $P > 0$ to cover fixed costs \\
    This leads to inefficiency
\end{quote}
The problem with public good:
\begin{quote}
    One can enjoy a positive production externality. If the good does not exist yet, this then leads to no production.
\end{quote}
The problem with common property goods:
\begin{quote}
    There is a negative consumption externality, leading to the Tragedy of the Commons, the market overuses.
\end{quote}
\subsubsection{Asymmetry of information, 信息不对称}
This means buyers and sellers have different relevant information about the goods. This leads to:
\begin{quote}
    1. Adverse selection \\
    2. Moral hazard \\
    3. Principal-agent problem \\
    4. Signaling
\end{quote}
Adverse selection is tilting the selection of goods towards poor quality before a contract is made.
\begin{quote}
    1. Seller knows more - The Lemons Problem
    \begin{quote}
        1. Increase buyer's ability to observe quality \\
        2. Incentives for truthful quality reporting \\
        3. Increasing average quality
    \end{quote}
    2. Buyer knows more - Insurance
    \begin{quote}
        1. Group policies (even out the risk) \\
        2. Screening
    \end{quote}
\end{quote}
A moral hazard is a party that is insulated from risk, after the contract is made, and therefore may behave adversely.
\begin{quote}
    1. Setting up deductible: portion of claim for which policyholder not compensated \\
    2. Co-payment: portion of the fee that the policyholder pays upfront \\
    3. Co-insurance: predetermined apportionment of claim
\end{quote}
The principal-agent problem is a type of moral hazard where the agent has the incentive to change behaviour after being hired, and the behaviour of the agent cannot be fully known. This is a sequential game.
\begin{quote}
    1. Flat rate \\
    2. Pay for performance \\
    3. Commission \\
    4. Bonuses \\
    5. Stock options \\
    6. "You're fired"
\end{quote}
A signal is a non-free message from the owner of the goods evincing unknown information.
\begin{quote}
    Education is a classic signalling example, as it signals higher productivity with a university degree. \\
    This can lead to a prisoner's dilemma effect where all students follow the dominant strategy to attend expensive schools to send better signals, but are all worse off because all spent more money for the same signal. \\
    These cases do not equate MSB (marginal social benefit) to MSC.
\end{quote}\

\subsubsection{Social goals}
These can include
\begin{quote}
    1. Income redistribution \\
    2. Merit goods: health, education \\
    3. Social obligations: military, jury \\
    4. Protecting individuals from others: minimum wage \\
    5. Protecting individuals from themselves: seat belt
\end{quote}
They are due to normative "value judgments"

\subsection{Case against government regulation}
\subsubsection{Methods}
1. Cost-benefit analysis to determine whether or not the government should provide the public good. \\
2. Problems arise
\begin{quote}
    1. Quantifying costs/benefits \\
    2. Forecasting \\
    3. Discounting future costs/benefits to present
\end{quote}
3. Methods include
\begin{quote}
    1. Public provision \\
    2. Redistribution \\
    3. Regulation
\end{quote}
4. Government can induce/impose behaviour, and change incentives.

\subsubsection{Costs of intervention}
1. Direct Resource Costs \\
2. Indirect resource costs
\begin{quote}
    Externalities: \\
    1. Costs of production: safety standards \\
    2. Costs of compliance: pay equity \\
    3. "Rent-seeking": lobbying for economic advantage
\end{quote}

\subsubsection{Causes of government failure}
1. Politicians' self-interests \\
2. "Public Choice Theory"
\begin{quote}
    1. Politicians (votes), bureaucrats (authority), and electorates (utility) do not care about the public interest \\
    2. Rational ignorance: no incentive to become informed, costs $>$ benefits
    \begin{quote}
        No need to be informed on tax lax change (cost), when a vote worth 1 in 30 mn (benefit)
    \end{quote}
    3. Democracy and inefficient public choice: "One person, one vote" fails to account for preferences, and Arrow's theory of social choice
\end{quote}
3. Government monopolies