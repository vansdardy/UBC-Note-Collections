\section{Competitive Market, 完全竞争市场}

\subsection{Market Structure and Firm Behaviour, 市场结构与公司行为}
\textbf{Market structure} is the genetic feature that affects firm behaviours (i.e. degree of market power). This includes:
\begin{quote}
    1. Number and size of firms \\
    2. Type of good - degree of product differentiation \\
    3. Freedom of entry and exit \\
    4. Pricing behaviour
\end{quote}
\textbf{Market power} is a single firm's ability to affect the commodity's price. \\
\textbf{Competitve behaviour} is to have some market power.

\subsection{Theory of Perfect Competition, 完全竞争理论}
\textbf{Perfectly competitive market structure} has NO market power, as not one firm in this market can affect the commodity's price. \\
This market structure has $3$ characteristics:
\begin{quote}
    1. Many many sellers \\
    2. Selling a homogeneous (identical) good \\
    3. Free entry and exit
\end{quote}
$1+2$: Firms are \textbf{price takers}. \\
$3$: Firms will produce at LRAC's minimum.
\subsubsection{Demand curve, 需求曲线}
Since firms are price takers, the demand curve for the \textbf{industry} is still downward sloping, but for each \textbf{firm}, the demand curve is horizontal. \\
This means:
\begin{quote}
    1. The firm can sell all it wants at the going price. \\
    2. If a firm raises its price, it loses all buyers. \\
    3. If a firm lowers its price, it loses its profits.
\end{quote}
\subsubsection{Revenue, 营业额}
Due to its horizontal demand curve, for any firm in perfect competition, we have the following relation:
$$P = AR = MR$$
$$\frac{TR}{Q} = \dv{TR}{Q}$$

\subsection{Theory of Profit Maximization, 利益最大化理论}
To maximize profits, any firm should follow two rules:
\begin{quote}
    1. Firm should only produce when it can cover \textbf{TVC}, that is $TR \ge TVC$ or $P \ge AVC$ \\
    2. Firm should produce up to the output when $MR = MC$.
\end{quote}

\subsection{Short Run Production Decision, 短期生产}
Case 1: Economic Profit
\begin{quote}
    1. $MR = MC$, find $P$ and $Q$ at intersection. \\
    2. $P > AC$
\end{quote}
Case 2: Normal Profit
\begin{quote}
    1. $MR = MC$, find $P$ and $Q$ at intersection. \\
    2. $P = AC$
\end{quote}
Case 3: Economic Loss
\begin{quote}
    1. $MR = MC$, find $P$ and $Q$ at intersection. \\
    2. $AVC < P < AC$
\end{quote}
Case 4: Shut-down point
\begin{quote}
    1. $MR = MC$, find $P$ and $Q$ at intersection. \\
    2. $P = AVC$
\end{quote}
In the short run, the marginal cost curve above the AVC is the firm's supply curve. \\

\subsection{Long Run Production Decisions, 长期生产}
\subsubsection{Entry and exit, 进入与退出}
When firms are producing at an \textbf{economic profits}, new firms are incentivized to \textbf{enter}. \\
When firms are producing at a \textbf{normal profit}, there will neither be entry nor exit. \\
When firms are producing at an \textbf{economic loss}, current firms will exit the industry. \\
For example,
\begin{quote}
    If original supply $S_1$ and demand $D$ for the industry has an equilibrium price at $P_1$, and $P_1 > AC$, then new firms will enter, shifting $S_1$ to the right to $S_2$, where the new equilibrium price is $P_2$. In this case, $P_2 = AC$. \\
    Industry output will rise, but firm output will fall.
\end{quote}
\subsubsection{Long run equilibrium, 长期平衡}
From the previous entry and exit behaviour, firms will produce at the minimum of SRAC. \\
Since LRAC would lie below SRAC, firms will lower costs even further (e.g. increasing the size of production). This will lead to firms increasing output until $Q_{output} = MES$ (minimum efficient scale). \\
At this equilibrium we have
\begin{quote}
    1. $P = MR = MC$ \\
    2. $P = \min{SRAC}$ \\
    3. $P = \min{LRAC}$
\end{quote}
Only \textbf{normal profits} is possible. \\
And we have reached productive efficiency and allocative efficiency. \\
If we assume continuous technological change, then
\begin{quote}
    1. New firms would enter with lower costs. \\
    2. Old firms would still exist as long as $P > AVC$. \\
    3. Price will eventually be determined by the new firms. \\
    4. Old firms would be "economically obsolete", not "physically obsolescent", resulting in closing when $P < AVC$.
\end{quote}
In other cases, an industry can be declining because the industry in long-run equilibrium suffers continuous decreasing demand. \\
The firms will respond by trying to cut costs, leading to failure to upgrade equipment, and eventually leading to long-run losses, a "vicious cycle". \\
In this case, the government may intervene by subsidizing these industries to "save jobs".

\newpage