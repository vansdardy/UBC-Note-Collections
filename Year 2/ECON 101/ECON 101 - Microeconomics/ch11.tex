\section{Imperfect Competition, 不完全竞争}

\subsection{Structure of the Canadian Economy, 加拿大经济结构}
\textbf{Canada is a large country with a small population}
\begin{quote}
    1. Large geography - high transport costs and natural BTE
    2. Small population - low demand leads to firms not able to minimize AC, thus leading to a natural monopoly
\end{quote}
Canada decides "competition" through the Industrial Concentration Ratio, and the Competition Act states that $CR_4 > 65\%$ means monopoly power. \\
$CR_4$: Fraction of total sales controlled by the top 4 firms in the industry; the higher the concentration ratio, the higher the market power. \\
However, several problems arise from such a measurement:
\begin{quote}
    1. Definition of "relevant market": it depends on the product and geography; \\
    2. It ties competition to the number of firms; \\
    3. It \textbf{overstates} the degree of concentration in Canada because Canada is an open economy.
\end{quote}

\subsection{Imperfect Competition}
This type of market exhibits: rivalrous behaviour, with some market power, to set price, within a range
There are two market structures of imperfect competition:
\begin{quote}
    1. Monopolistic Competition: Large number of small firms and non-strategic behaviour (ignoring what other firms do)
    2. Oligopoly: Small number of large firms, and exhibit strategic behaviour
\end{quote}
The characteristics of imperfect competition include:
\begin{quote}
    1. Some to many sellers \\
    2. Products are differentiated \\
    3. Entry and exit are not easy \\
    4. Price setters within a range
\end{quote}
The firms in this market structure present these common behaviours:
\begin{quote}
    1. They select their products (differentiated - homogeneous goods in the same market but distinguished in consumer's eyes). And these products are not perfect substitutes. \\
    2. They select their prices (administered - set by individual firms within a range by referring to S and D, but not completely driven by market forces). The prices set tend to be sticky, as firms will respond to changes in demand by adjusting output not price. \\
    3. They engage in non-price competition (e.g. advertisement, warranties, services)
\end{quote}

\subsection{Monopolistic Competition, 垄断性竞争}
The characteristics of monopolistic competition include:
\begin{quote}
    1. Many sellers - non-strategic behaviour, like Perfect Competition \\
    2. Differentiated good - like Monopoly \\
    3. Significant entry and exit - like Perfect Competition \\
    4. Price setter within a range - sticky prices
\end{quote}
Therefore, firms in monopolistic competition behave like monopolies in SR, because of differentiation. \\
They behave like perfect competition because of the significant entry in LR. \\
In the short run,
\begin{quote}
    The firm will equate $MR = MC$, and find $P$ based on intersected $Q$. The firm then makes monopoly profits.
\end{quote}
In the long run,
\begin{quote}
    The firm enjoys economic profits, and signals other firms to enter, shifting the D of each firm to the left until D is tangent to AC. Firms then only make normal profits.
\end{quote}
\subsubsection{Excess Capacity Theorem}
\textbf{Excess Capacity} $=$ Capacity $Q$ - LR $Q_E$
In imperfect competition, there is an excess capacity where firms produce less at higher prices. \\
It is now accepted that consumers benefit from more brands arising from monopolistic competition.

\subsection{Oligopoly, 寡头}
The characteristics of monopolistic competition include:
\begin{quote}
    1. Several sellers\\
    2. Similar or differentiated good - present strategic behaviour, they are \textbf{interdependent} \\
    3. Formidable entry and exit - firms' output affects industry supply \\
    4. Price setter within a range - administered prices
\end{quote}
Oligopoly behaves strategically to have monopoly power over price, usually through merging and acquisition. This is to decrease rivalry and increase profits. \\
Both natural causes (high start-up costs) and artificial causes (government intervention) can lead to such a market structure.

\subsection{Game Theory, 博弈论}
A \textbf{game} is a decision-making process of $2+$ players who are interdependent and know the outcomes \\
A \textbf{simultaneous game} is a game where both players make decisions at the same time. \\
A \textbf{sequential game} is a game where one player makes a decision, then the other reacts. \\
\subsubsection{Simultaneous game, 同步博弈}
The \textbf{best response} of a player is the strategy taken when the player cannot gain utility by switching to a different strategy, given the strategy of the other player. \\
A \textbf{Nash equilibrium} is when each player is playing the best responses. An equilibrium is usually reached by rational non-cooperation. \\
A \textbf{dominant strategy} is a strategy that yields a higher payoff, regardless of the strategy of the other player.
A \textbf{prisoner's dilemma game}(囚徒困境)is a game where
\begin{quote}
    1. Each player has a dominant strategy \\
    2. that leads to a Nash equilibrium \\
    3. with a lower payoff \\
    4. than if they had not played their dominant strategy \\
    In general, they are better off cooperating but worse off following self-interest respectively.
\end{quote}
The \textbf{Pareto optimum} is a situation where one cannot make someone better off without making someone else worse off.
\begin{quote}
    1. Allocative Efficiency = Pareto optimality/efficiency \\
    2. Pareto Improvement is making someone better off without making someone else worse off \\
    3. Pareto Dis-improvement is making someone worse off without affecting others.
\end{quote}
\subsubsection{Sequential game, 序贯博弈}
A \textbf{decision tree} is a diagram that shows the possible decisions, in sequence, with the payoffs for each possible decision. \\
Player $2$ will have the best response according to Player $1$'s choice, thus, through backward induction, Player $1$ can have a certain best response. \\
An \textbf{ultimatum game} is when the first player has the power to impose a "take it or leave it" offer. A minimum acceptance threshold can be imposed by the Player $2$ in advance. \\
A \textbf{credible threat/promise} is a promise which is in the promisor's interest to perform.

\subsubsection{Problems in games and strategic role of preferences}
For a game, we may have the players:
\begin{quote}
    1. Not having full knowledge \\
    2. Not having all relevant information \\
    3. Have a commitment problem (player's inability to make a credible promise) \\
    4. Not have a commitment device (methods to create a credible promise by changing incentives) \\
    5. Motivated by self-interest \\
    6. Have an advantage because of first-mover ($2$ Nash equilibrium)
\end{quote}
The solution to a commitment problem can include:
\begin{quote}
    1. Alter the material incentives the players face \\
    2. Alter the psychological incentives the players face and ensure the other party knows of these incentives (Stockholm Syndrome)
\end{quote}

\subsection{Oligopoly in practice}
A \textbf{collusion} is an agreement to cooperate to restrict $Q$ and raise $P$:
\begin{quote}
    1. Explicit/Overt collusion - Can lead to cheating, Conflicts in determining market share of each member \\
    2. Implicit/Covert/Tacit collusion
    \begin{quote}
        May or may not cheat, \\
        1. Conscious parallelism: recognition of common interests without explicit agreement. \\
        2. Focal point pricing: MSRP.
    \end{quote}
\end{quote}
There can still be rivalry within collusion: market share competition, and 
innovation competition. \\
\textbf{Contestable markets} allow free entry and exit, and allow "hit and run entry". \\
Oligopoly structures are tradeoffs: monopoly power vs. cost saving and innovation.

\newpage