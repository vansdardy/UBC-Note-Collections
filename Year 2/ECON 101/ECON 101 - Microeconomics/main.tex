\documentclass{article}

\usepackage{amsmath, amsthm, amssymb, amsfonts}
\usepackage{thmtools}
\usepackage{graphicx}
\usepackage{setspace}
\usepackage{geometry}
\usepackage{float}
\usepackage{hyperref}
\usepackage[utf8]{inputenc}
\usepackage[english]{babel}
\usepackage{framed}
\usepackage[dvipsnames]{xcolor}
\usepackage{tcolorbox}
\usepackage{xeCJK}
\usepackage{physics}
\usepackage{tikz-cd}
\usepackage[mathscr]{euscript}

\colorlet{LightGray}{White!90!Periwinkle}
\colorlet{LightOrange}{Orange!15}
\colorlet{LightGreen}{Green!15}
\colorlet{LightBlue}{Blue!15}

\newcommand{\HRule}[1]{\rule{\linewidth}{#1}}
\newcommand{\R}{\mathbb{R}}
\newcommand{\C}{\mathbb{C}}
\newcommand{\Q}{\mathbb{Q}}
\newcommand{\N}{\mathbb{N}}
\newcommand{\U}{\mathbb{U}}
\newcommand{\lima}{\lim_{x \to a}}
\newcommand{\dps}{\displaystyle}

\declaretheoremstyle[name=Theorem,]{thmsty}
\declaretheorem[style=thmsty,numberwithin=section]{theorem}
\tcolorboxenvironment{theorem}{colback=LightBlue}

\declaretheoremstyle[name=Proposition,]{prosty}
\declaretheorem[style=prosty,numberwithin=section]{proposition}
\tcolorboxenvironment{proposition}{colback=LightGreen}

\declaretheoremstyle[name=Principle,]{prcpsty}
\declaretheorem[style=prcpsty,numberlike=theorem]{principle}
\tcolorboxenvironment{principle}{colback=LightGray}

\declaretheoremstyle[name=Definition,]{defnsty}
\declaretheorem[style=defnsty,numberwithin=section]{definition}
\tcolorboxenvironment{definition}{colback=LightOrange}

\setstretch{1.2}
\geometry{
    textheight=9in,
    textwidth=5.5in,
    top=1in,
    headheight=12pt,
    headsep=25pt,
    footskip=30pt
}

% ------------------------------------------------------------------------------

\begin{document}

\title{ \normalsize \textsc{}
		\\ [2.0cm]
		\HRule{1.5pt} \\
		\LARGE \textbf{\uppercase{ECON 301 - Intermediate Microeconomic Analysis}
		\HRule{2.0pt} \\ [0.6cm] \LARGE{进阶微观经济学分析} \vspace*{10\baselineskip}}
		}
\date{}
\author{\textbf{Author} \\ 
		Wenyou (Tobias) Tian \\
        田文友 \\
		University of British Columbia \\
        英属哥伦比亚大学 \\
		2024}

\maketitle
\newpage

\tableofcontents
\newpage
\section{Economic Issues and Concepts}
See Midterm 1 Notes.
\newpage
\section{Functions of Several Variables, Limits, Continuity, Partial Derivatives, 多元函数,极限,连续性,偏微分}
\subsection{Functions and Surfaces, 函数与表面}
Functions of several variables in $\R^n$ maps
$$(x_1, \dots, x_n) \mapsto f(x_1, \dots, x_n)$$
where $(x_1, \dots, x_n) \in \mathscr{D}(f)$. The natural domain of $f$ is the set of $(x_1, \dots, x_n)$ where $f$ is naturally defined. The range of $f$ is then set of all values of $f$, where $f(x_1, \dots, x_n) \in \R$. \\
The corresponding graph has the set
$$\{(x_1, \dots, x_n, f(x_1, \dots, x_n))\}$$
which is a subset of $\R^{n+1}$. \\
We can visualize high-dimension graphs using level curves. For a function of $2$ variables, level curves on the $x-y$ plane is formed by having $f(x, y)=c$, similarly, for a function of $3$ variables, level curves in the $xyz$ space is formed by having $f(x, y, z) = c$.

\subsection{Limits and Continuity, 极限与连续性}
For functions of $2$ variables, 
$$\lim_{(x, y) \to (a, b)} f(x, y) = L$$
if:
\begin{quote}
    1. Every neighbourhood of $(a, b)$ contains points of $\mathscr{D}(f)$ other than $(a, b)$ \\
    2. $\forall \epsilon  > 0, \exists \delta > 0, (x, y) \in \mathscr{D}(f) \land 0 < \sqrt{(x-a)^2 + (y-b)^2} < \delta \implies |f(x, y) - L| < \epsilon$
\end{quote}
The rules for limits from single-variable calculus still apply.
\begin{definition}
    $f(x, y)$ is continuous at $(a, b)$ if: \\
    1. $(a, b) \in \mathscr{D}(f)$, \\
    2. $$\lim_{(x, y) \to (a, b)} f(x, y) = f(a, b)$$
\end{definition}
\begin{theorem}
    The Squeeze Theorem for multivariable calculus states that: \\
    If $f, g, h$ are defined on some $B_r(a, b)$, except at $(a, b)$ and
    $$\lim_{(x, y) \to (a, b)} f(x, y) = \lim_{(x, y) \to (a, b)} h(x, y) = L$$
    if $g$ lies between $f, h$ on that neighbourhood, then
    $$\lim_{(x, y) \to (a, b)} g(x, y) = L$$
\end{theorem}
For a multivariable function to have a limit at a certain point, then whichever path we choose to approach this point, the limit should all be the same; thus, if for two paths we choose to approach this point, we evaluated to have different limits, the limit at this point does not exist. \\
For functions that do have a limit at a point, we usually use rules of limits and the Squeeze Theorem to evaluate such a limit, $\epsilon-\delta$ proof is usually not required. \\
Note the useful inequality
$$|a+b| \le |a| + |b|$$

\subsection{Partial Derivatives, 偏微分}
\begin{definition}
    $f_i(x_1, \dots, x_n)$ is the partial derivative with respect to $x_i$ with $x_1, \dots, x_{i-1}, x_{i+1}, \dots, x_n$ fixed, where
    $$f_i(x_1, \dots, x_n) = \lim_{h \to 0} \frac{1}{h} (f(x_1, \dots, x_i + h, \dots, x_n) - f(x_1, \dots, x_n))$$
\end{definition}
Notation-wise, we have
$$f_1(x, y) = f_x(x, y) = \pdv{f(x, y)}{x} = D_1(x, y) = D_x(x, y)$$
Different from single-variable calculus, if $f_1, f_2$ exists, this does not imply that $f$ is continuous at $(x, y)$. \\
A function being differentiable at a point implies that we can approximate $f(x, y)$ by a linear function. \\
Given Cantor lines, we can also estimate partial derivatives.

\subsection{Tangent Planes, 切面}
Given a function $f$, if $f_x(P), f_y(P)$ exist and $f$ is continuous on a neighbourhood of $P$, then a plane tangent to $f$ at $P$ exists. \\
For two variables, where $z = f(x, y)$, a tangent plane at $(a, b)$ can be calculated to be
$$z = f(a, b) + f_1(x, y)(x-a) + f_2(x, y)(y - b)$$
The line through $P$ that is perpendicular to the surface has the direction vector
$$\Vec{n} = \langle -f_1(x, y), -f_2(x, y), 1 \rangle$$

\subsection{Higher Order Derivatives, 高阶导数}
Notation-wise, consider the following equality
$$\pderivative{x}(\pdv{z}{y}) = \pdv{z}{x}{y} = f_{21}(x, y) = f_{yx}(x, y)$$
If we assume $\pdv{f}{x}{y}$ and $\pdv{f}{y}{x}$ are continuous at $P = (a, b)$, and $f, \pdv{f}{x}, \pdv{f}{y}$ are continuous on neighbourhoods of $P$, then
$$\pdv{f}{x}{y} = \pdv{f}{y}{x}$$
A \textit{partial differential equation} is an equation involving the partial derivatives of some function. At this current stage, we can verify some function to be the solution to a PDE.

\subsection{Chain Rule, 链式法则}
If $f(x_1, \dots, x_n)$ is $C^k$, then $f$ and its partial derivatives up to order $k$ are continuous on $\mathscr{D}(f)$. \\
Recall the single-variable chain rule, consider a function $f(x, y)$, where $x = x(t)$, and $y = y(t)$, then
$$\dv{t}f(x(t), y(t)) = \pdv{f}{x} \cdot \dv{x}{t} + \pdv{f}{y} \cdot \dv{y}{t}$$
For variables, that is, if $x = x(u, v), y = y(u, v)$, then
$$\pdv{u}f(x, y) = \pdv{f}{x} \pdv{x}{u} + \pdv{f}{y} \pdv{y}{u}$$
$$\pdv{v}f(x, y) = \pdv{f}{x} \pdv{x}{v} + \pdv{f}{y} \pdv{y}{v}$$
Moreover, if we are differentiating with respect to a variable where $x = x(t), y = y(t), f(x, y, t)$, then
$$\dv{t}f(x, y, t) = \pdv{f}{x} \cdot \dv{x}{t} + \pdv{f}{y} \cdot \dv{y}{t} + \pdv{f}{t} $$
For higher-order partials, we iterate the chain rule. \\

\newpage
\section{Topics in Differentiation, 微分}
\subsection{Linear Approximation and Differentiability, 线性逼近与可微}
For $f(x, y)$, if we approximate the surface near $(a, b)$ with a linear function $L(x, y)$, then this linear function has the form
$$z = L(x, y) = f(a, b) + f_1(x, y)(x-a) + f_2(x, y)(y-b)$$
Then, $f$ is differentiable at $(a, b)$ if
$$f(a, b) - L(a, b) = o(\sqrt{(x-a)^2 + (y-b)^2})$$
which is equivalent to
$$\lim_{(x, y) \to (a, b)}\frac{|f(x, y) - L(x, y)|}{\sqrt{(x-a)^2 + (y-b)^2}} = 0$$
Furthermore, if $f(x, y)$ is differentiable at $(a, b)$, then $f(x, y)$ is continuous at $(a, b)$, and $f_1, f_2$ exists at $(a, b)$. \\
If $f(x, y)$ is $C^1$ on a neighbourhood of $(a, b)$, then it is differentiable at $(a, b)$. \\
The linear approximation can be interpreted as \textbf{differentials}(微分)where if $z = f(x, y)$ then
$$\dd z = \dd f = \pdv{f}{x} \dd x + \pdv{f}{y} \dd y$$
We then can prove whether or not a function is differentiable at a point from the definition.

\subsection{Gradients and Directional Derivatives, 梯度与方向导数}
Given a function $f(x_1, \dots, x_n)$, then the \textbf{\textit{gradient}} of the function is a vector where
$$\nabla f = \langle f_1, \dots, f_n \rangle$$
In general, $\nabla f$ is always perpendicular/normal to the level curve/surface of $f$, that is $\nabla f \cdot \Vec{T} = 0$
To find the direction derivative of $f$, we first normalize the direction vector to $\Vec{u} = \langle u_1, u_2 \rangle$, where $|\Vec{u}| = 1$, then if $f$ is differentiable at $(a, b)$, we have
$$D_{\Vec{u}}f(x, y)|_{(x, y) = (a, b)} = \nabla f(a, b) \cdot \Vec{u}$$
In general, we define such a derivative with a limit definition where,
$$D_{\Vec{u}}f(x, y)|_{(x, y) = (a, b)} := \lim_{h \to 0} \frac{1}{h} (f(a+hu_1, b+hu_2) - f(a, b))$$
To find those directions with the steepest ascent/descent, we want to maximize/minimize $|D_{\Vec{u}}f(x, y)|_{(x, y) = (a, b)}|$, since it is equal to $\nabla f \cdot \Vec{u} = |\nabla f||\Vec{u}|\cos{\theta}$, 
\begin{quote}
    1. To maximize, we choose $\theta = 0$, where $\Vec{u} = \frac{\nabla f}{|\nabla f|}$ \\
    2. To minimize, we choose $\theta = \pi$, where $\Vec{u} = -\frac{\nabla f}{|\nabla f|}$
\end{quote}

\subsection{Implicit Differentiation, 隐函数求导}
Given equation $F(x, y) = c$, if we define $y$ as a function of $x$, find $\dv{y}{x}$:
$$\dv{x} c = \dv{F}{x} = F_1(x, y(x)) + F_2(x, y(x))\dv{y}{x}$$
Then 
$$\dv{y}{x} = -\frac{F_1}{F_2}$$
If $F_2 = 0$, then either $\nabla F_1(P) = F_1(P)i$, or the tangent line is vertical. \\
Given equation $F(x, y, z) = c$, if we define $z$ as a function of $x, y$, find $\pdv{z}{x}, \pdv{z}{y}$:
$$\pdv{x} c = \pdv{F}{x} = F_1(x, y, z) + F_3(x, y, z)\pdv{z}{x}$$
$$\pdv{y} c = \pdv{F}{y} = F_1(x, y, z) + F_3(x, y, z)\pdv{z}{y}$$
Then 
$$\pdv{z}{x} = -\frac{F_1}{F_3}$$
$$\pdv{z}{y} = -\frac{F_2}{F_3}$$
If $F_3 = 0$, then there is a vertical plane, and the result is inconclusive. \\
\\
Given a system of equations
\begin{align*}
    u &= f(x, y) \\
    v &= g(x, y)
\end{align*}
Find $\pdv{x}{u}, \pdv{x}{v}, \pdv{y}{u}, \pdv{y}{v}$. \\
Assume $x = x(u, v), y = y(u, v)$, then
\begin{align*}
    \pderivative{u} u &= \pderivative{u} f(x, y) \\
    \pderivative{v} u &= \pderivative{v} f(x, y)
\end{align*}
This gives us
\begin{align*}
    1 &= f_1 \pdv{x}{u} + f_2 \pdv{y}{u} \\
    0 &= g_1 \pdv{x}{u} + g_2 \pdv{y}{u}
\end{align*}
Then by either solving directly or using Cramer's Rule, we can find $\pdv{x}{u}, \pdv{y}{u}$, with Cramer's Rule, we have
$$\pdv{x}{u} = \frac{\begin{vmatrix}
    1 & f_2 \\
    0 & g_2
\end{vmatrix}}{\begin{vmatrix}
    f_1 & f_2 \\
    g_1 & g_2
\end{vmatrix}}, \pdv{y}{u} = \frac{\begin{vmatrix}
    f_1 & 1 \\
    g_1 & 0
\end{vmatrix}}{\begin{vmatrix}
    f_1 & f_2 \\
    g_1 & g_2
\end{vmatrix}}$$
Then, a $2 \times 2$ Jacobian is written as
$$\pdv{(u, v)}{(x, y)} = \begin{vmatrix}
    f_1 & f_2 \\
    g_1 & g_2
\end{vmatrix}$$

\subsection{Taylor Polynomial, 泰勒多项式}
The first-order Taylor approximation is the linear approximation mentioned previously, thus, we focus on the second-order Taylor approximation, where we approximate surfaces using quadric surfaces.
\begin{align*}
    p_2(x, y) &= f(a, b) \\
    &+ (f_1(a, b)(x-a) + f_2(a, b)(y-b)) \\
    &+ \frac{1}{2}(f_{11}(a, b)(x-a)^2 + f_{12}(a, b)(x-a)(y-b) + f_{21}(a, b)(x-a)(y-b) + f_{22}(a, b)(y-b)^2)
\end{align*}
Then with the \textit{Hessian matrix}:
$$\mathscr{H} f(a, b) = \begin{pmatrix}
    f_{11} & f_{12} \\
    f_{21} & f_{22}
\end{pmatrix}$$
The second-order Taylor polynomial becomes
$$p_2(x, y) = f(a, b) + \nabla f(a, b) \cdot \langle x-a, y-b \rangle + \frac{1}{2} \begin{pmatrix}
    x-a & y-b
\end{pmatrix} \mathscr{H} f(a, b) \begin{pmatrix}
    x-a \\
    y-b
\end{pmatrix}$$
For error estimation of a second-order Taylor polynomial, we would have
$$\lim_{(x, y) \to (a, b)} \frac{f(x, y) - p_2(x, y)}{(\sqrt{(x-a)^2 + (y-b)^2})^2} = 0$$
For higher dimensions, suppose $f = f(x_1, \dots, x_n)$ in a neighbourhood of $\textbf{a} = (a_1, \dots, a_n)$, then we have the Taylor polynomial to be
$$f(\textbf{x}) \approx p_2(\textbf{x}) := f(\textbf{a}) + \nabla f(\textbf{a}) \cdot (\textbf{x} - \textbf{a}) + \frac{1}{2}(\textbf{x} - \textbf{a}) \mathscr{H}f(\textbf{a}) (\textbf{x} - \textbf{a})^T$$
where
$$\mathscr{H}f(\textbf{a}) = \begin{pmatrix}
    f_{11} & \dots & f_{1n} \\
    \vdots & \ddots & \vdots \\
    f_{n1} & \dots & f_{nn}
\end{pmatrix}$$
\newpage
\section{Linear Maps and Matrices, 线性映射与矩阵}
\subsection{Linear Maps, 线性映射}
\begin{definition}
    Let $V$ and $W$ be vector spaces over $\F$, a map:
    $$f: V \to W$$
    is called a \textbf{\textbf{linear transformation}}(线性变换) if it satisfies:
    $$\forall x, y \in V, f(x+y) = f(x) + f(y)$$
    $$\forall \lambda \in \F, x \in V, f(\lambda x) = \lambda f(x)$$
\end{definition}
This linear map is also called a \textbf{\textit{homomorphism}}(同态). \\
The set of homomorphisms between $V$ and $W$ also forms a vector space over $\F$, denoted as:
$$\Hom_\F(V, W):=\{f: V \to W\}$$
with addition and multiplication to be defined as
$$+: (f+g)(v) = f(v) + g(v)$$
$$\cdot: (\lambda f)(v) = \lambda f(v)$$
More on notations of linear maps,
\begin{definition}
    A linear map $f: V \to W$ is called: \\
    a \textbf{\textit{monomorphism}}(单态射) if it is injective; \\
    an \textbf{\textit{epimorphism}} (满态射)if it is surjective; \\
    an \textbf{\textit{isomorphism}} (同构)if it is bijective; \\
    an \textbf{\textit{endomorphism}} (自同态)if $V = W$; \\
    an \textbf{\textit{automorphism}} (自同构)if it is bijective and $V = W$
\end{definition}
More on an isomorphism, it means that if $f$ is an isomorphism, then these necessarily follows:
\begin{quote}
    $f$ is linear. \\
    There exists $g:W \to V$, such that $f \circ g = \id_W$ and $g \circ f = \id_V$. \\
    Given $\{v_1,\dots,v_n\}$ to be the basis of $V$, $\{f(v_1),\dots,f(v_n)\}$ is a basis of $W$, the converse is true as well.
\end{quote}
We then call $g$ the inverse of $f$, and we can denote it as $f^{-1}$. \\
Moreover, any two $n$-dimensional vector space over the same field $\F$ are isomorphic. \\
Linear maps between two $n$-dimensional spaces are surjective if and only if they are injective. \\
\\
We are then interested in some special spaces of these linear maps, 
\begin{definition}
    Let $f: V \to W$ be a linear map, then: \\
    the \textbf{\textit{kernel}}(核,零空间) of $f$ is defined as
    $$\Ker f := \{v \in V: f(v) = 0\}$$
    the \textbf{\textit{image}}(像) of $f$ is defined as
    $$\im f := \{f(v): v \in V\}$$
    Then, naturally, $\Ker f \subset V$,$\im f \subset W$.
\end{definition}
We then define a feature about the image of $f$,
\begin{definition}
    The \textbf{\textit{rank}}(秩) of the linear map $f$ is defined to be
    $$\rk f := \dim (\im f)$$
\end{definition}
This then will give us an extremely useful theorem,
\begin{theorem}
    The \textbf{\textit{rank-nullity theorem}}(秩——零化度定理) states that:
    $$\rk f + \dim (\Ker f) = \dim V$$
\end{theorem}
If we further consider the image of $f$, then given $\{v_1,\dots,v_n\}$ to be the basis of $V$, there exists a unique set of vectors $\{w_1, \dots, w_n\}$ such that for every $i=1,\dots, n$, $f(v_i)=w_i$.

\subsection{Matrices, 矩阵}
Consider a general $m\times n$ matrix, where $m$ is the number of rows, and $n$ is the number of columns,
$$A = (a_{ij}) =
\begin{pmatrix}
    a_{11} & \dots & a_{1n} \\
    \vdots & & \vdots \\
    a_{m1} & \dots & a_{mn}
\end{pmatrix}
$$
If we define $M_{m \times n}(\F)$ to be the set of all $m \times n$ matrices with entries in $\F$, then naturally, $A \in M_{m \times n}(\F)$, and we can build a \textbf{bijective} map
$$L: \Hom_\F(V, W) \to M_{m \times n}(\F)$$
so that each linear map (homomorphism) between $V$ and $W$ can be represented as a $m \times n$ matrix, given that $\dim V = n$ and $\dim W = m$. \\
If $v \in V$ and $v = \lambda_1 v_1 + \dots + \lambda_n v_n$ given the basis of $V$ to be $\{v_1, \dots, v_n\}$, then by writing the coefficients in a column, we can represent
$$v = 
\begin{pmatrix}
    \lambda_1 \\
    \vdots \\
    \lambda_n
\end{pmatrix}$$
Then,
$$Av = 
\begin{pmatrix}
    a_{11} & \dots & a_{1n} \\
    \vdots & & \vdots \\
    a_{m1} & \dots & a_{mn}
\end{pmatrix}
\begin{pmatrix}
    \lambda_1 \\
    \vdots \\
    \lambda_n
\end{pmatrix}
=
\begin{pmatrix}
    \sum_{i=1}^n a_{1i}\lambda_i \\
    \vdots \\
    \sum_{i=1}^n a_{mi}\lambda_i
\end{pmatrix}$$
The general rule behind any matrix calculation is \textbf{row $\times$ column}. \\
Thus, essentially, a $m \times n$ matrix is map $\F^n \to \F^m$. \\
By the above arithmetic rule, we can know that
\begin{quote}
    The columns of a matrix are the images of the \textbf{unit vectors}, $e_1, \dots, e_n$.
\end{quote}
From above, by choosing a basis in $V$, not only did we get a coordinate system in $V$, but also we can write any vector in $V$ as a column with entries in the field, that is choosing an isomorphism $\F^n \to V$, where
\begin{definition}
    A \textbf{\textit{canonical basis isomorphism}} is defined to be
    \begin{align*}
        \Phi_{\{v_1,\dots, v_n\}}: \F^n & \to V \\
        (\lambda_1, \dots, \lambda_n) & \mapsto \lambda_1 v_1 + \dots + \lambda_n v_n
    \end{align*}
    if we have chosen the basis of $V$ to be $\{v_1,\dots, v_n\}$.
\end{definition}
Now, since we can write any linear map as a matrix, then if we choose the basis of $V$ to be $\{v_1,\dots, v_n\}$ and the basis of $W$ to be $\{w_1,\dots, w_m\}$, where $f: V \to W$, $A$ represents $f$, then the entries of the $i^{th}$ column of $A$ is the coordinate of $Av_i$ in terms of $\{w_1,\dots, w_m\}$. That is
$$f(v_i) = Av_i = a_{1i}w_1 + \dots + a_{mi}w_m$$
The following commutative diagram can assist in understanding:
$$\begin{tikzcd}
    \F^n \arrow[r, "A"] \arrow[d, "\Phi_{\{v_1,\dots, v_n\}}"]
        & \F^m \arrow[d, "\Phi_{\{w_1,\dots, w_m\}}"] \\
    V \arrow [r, "f"]
        & W
\end{tikzcd}$$
The last thing to highlight is that consider the previously mentioned map $L$, if we know $L(f) = A$, then
\begin{definition}
    The \textbf{\textit{rank}} of a matrix $A$ is its column rank, that is, the maximal number of linearly independent columns,
    $$\rk A = \rk f$$
    given the above assumption. \\
    Similarly, the row rank of the same matrix is the maximal number of linearly independent rows.
\end{definition}
\begin{theorem}
    Given a matrix $A$, its column rank $=$ its row rank.
\end{theorem}
\newpage
\section{Matrix Calculus, 矩阵运算}

\subsection{Matrix Multiplication, 矩阵乘法}
First, we define matrix addition and scalar multiplication:
\begin{definition}
    Let $(a_{ij}), (b_{ij}) \in M_{m \times n}(\F)$ and $\lambda \in \F$, then
    $$(a_{ij}) + (b_{ij}) := (a_{ij} + b_{ij}) \in M_{m \times n}(\F)$$
    $$\lambda(a_{ij}) := (\lambda a_{ij}) \in M_{m \times n}(\F)$$
\end{definition}
Now, consider a diagram
$$\begin{tikzcd}
    V \arrow[r, "B"]
        & W \arrow[r, "A"]
        & U
\end{tikzcd}$$
Then, $AB: V \to U$, that is $AB := A \circ B$
\begin{definition}
    Let $\dim V = n, \dim W = m, \dim U = r$, if $A = (a_{ik}) \in M_{r \times m}(\F)$ and $B = (b_{kj}) \in M_{m \times n}(\F)$, the \textbf{\textit{product}} $AB \in M_{r \times n}(\F)$ is defined by
    $$AB := (\sum_{k=1}^m a_{ik}b_{kj})_{(i = 1, \dots, r), (j = 1, \dots, n)}$$
\end{definition}
Matrix multiplication is associative $A(BC) = (AB)C$, and distributive with respect to addition $A(B+C) = AB + AC$ and $(A+B)C = AC + BC$, but they are NOT commutative.
\begin{definition}
    A matrix $A$ is \textbf{\textit{invertible}}(可逆的)if the associated linear map is an isomorphism; the matrix of the inverse map is then called the matrix \textbf{\textit{inverse}}(逆矩阵)to $A$ and is denoted by $A^{-1}$.
\end{definition}
There are several useful remarks regarding matrix inversion:
\begin{quote}
    1. Each invertible matrix $A$ is square. \\
    2. If $A$ is invertible, then $A^{-1}$ is invertible. \\
    3. If $A, B$ are invertible, then $AB$ is invertible and $(AB)^{-1}=B^{-1}A^{-1}$ \\
    4. If $A, B$ are square matrices, then
    $$AB = \id \iff BA = \id \iff B = A^{-1}$$
\end{quote}

\subsection{Elementary Transformations}
There are three elementary ROW transformations:
\begin{definition}
    For a matrix $A \in M_{m \times n}(\F)$, we have: \\
    (R1): Interchanging two rows. \\
    (R2): Multiplication of a row by a scalar $\lambda \ne 0, \lambda \in \F$. \\
    (R3): Addition of an arbitrary multiple of one row to another row.
\end{definition}
Elementary transformations do not alter the rank of the matrix. \\
Thus, if after an arbitrary number of transformations, if the first $r$ entries on the diagonal of the operated matrix are distinct from $0$, and the rest $m-r$ rows and all the entries below the diagonal are $0$, then $\rk A = r$.
\newpage
\section{System of Linear Equations, 线性方程组}
\subsection{System of Linear Equations}
Generally, if $A \in M_{m \times n}(\F)$, then we consider
$$Ax = b$$
to represent a system of linear equations where
\begin{align*}
    a_{11}x_1 + a_{12}x_2 + \dots + a_{1n}x_n &= b_1 \\
    a_{21}x_1 + a_{22}x_2 + \dots + a_{2n}x_n &= b_2 \\
    \vdots \\
    a_{m1}x_1 + a_{m2}x_2 + \dots + a_{mn}x_n &= b_m
\end{align*}
where
$$A = \begin{pmatrix}
    a_{11} & a_{12} & \dots & a_{1n} \\
    a_{21} & a_{22} & \dots & a_{2n} \\
    \vdots & \vdots &       & \vdots \\
    a_{m1} & a_{m2} & \dots & a_{mn}
\end{pmatrix},
x = \begin{pmatrix}
    x_1 \\
    x_2 \\
    \vdots \\
    x_n
\end{pmatrix},
b = \begin{pmatrix}
    b_1 \\
    b_2 \\
    \vdots \\
    b_m
\end{pmatrix}$$
If $b_i = 0$, then the system is said to be \textbf{\textit{homogeneous}}(齐次). \\
Then we define
\begin{definition}
    Given a system of linear equations $Ax = b$, the solution set is
    $$A^{-1}(b) := \{x: Ax = b\}$$
\end{definition}
If we already know $Ax_0 = b$, then every solution to this linear system is of form $x_0 + k$, where $k \in \Ker A$. \\
A useful remark for the solvability of a linear system is that, $Ax=b$ is solvable if and only if
$$\rk A = \rk (A|b)$$

\subsection{Solving Linear Systems, 求解线性方程组}
We apply Gaussian elimination (elementary row transformation) to reduce $A$ to its echelon form, that is for $i^{th}$ row, the leading $1$ should appear at or after $i^{th}$ entry of the row, and all entries before and below the leading $1$s should vanish to $0$. \\
A row-reduced echelon form follows all the rules above while having $0$s above all leading $1$s. \\
If the linear system has a solution, then $b \in \im A$. \\
If the linear system has a UNIQUE solution, then $\Ker A = \{0\}$ and $A$ is injective. \\
If $A$ is a square matrix, then $A$ is surjective, then a solution exists for every $b$. \\
To define a vector space using $Ax = 0$, the vector space is the solution set to this linear system, that is, $\Ker A$. If $A$ is not full rank, then we can express the kernel in parametric form to find bases for the vector space. For example
$$\begin{pmatrix}
    1 & 2 & 3 \\
    4 & 5 & 6 \\
    7 & 8 & 9
\end{pmatrix}
\to
\begin{pmatrix}
    1 & 0 & -1 \\
    0 & 1 & 2 \\
    0 & 0 & 0
\end{pmatrix}$$
Let $z = t$, then $x = t$, and $y = -2t$, then
$$\Ker A = t \begin{pmatrix}
    1 \\
    -2 \\
    1
\end{pmatrix}$$
where $\begin{pmatrix}
    1 \\
    -2 \\
    1
\end{pmatrix}$ is the basis of this vector space. \\
Since row operations preserve $\Ker A$ and $\im A$, we already know the dimension of the kernel is $1$ from above, then by rank-nullity theorem, $\dim (\im A)$. Furthermore, since the columns of $A$ represent the image of the unit vectors, any $2$ vectors can span $\im A$, thus, we have an equation for $\im A$, which is
$$s\begin{pmatrix}
    1 \\
    4 \\
    7
\end{pmatrix} + t \begin{pmatrix}
    2 \\
    5 \\
    8
\end{pmatrix}$$
After Gaussian elimination, we define \textbf{\textit{pivot columns}} to be columns with a leading "1", where \textbf{\textit{non-pivot columns}} to be columns without a leading "1". Each of the non-pivot columns contributes a parameter.

\subsection{Inverting a Matrix, 求逆矩阵}
First, consider an $n \times n$ square matrix $A$, where $Ax = b$.
\begin{quote}
    1. If $\rk A < n$, $A^{-1}$ does not exist. \\
    2. For some $b$, there will be infinitely many solutions, then these $b$ form a linear subspace of $\im A$. \\
    3. For some $b$, there will be no solution, then these $b$ have the property that $b \notin \im A$.
\end{quote}
Since row operations are essentially matrix compositions:
$$\begin{pmatrix}
    0 & 1 \\
    1 & 0
\end{pmatrix}A: \text{swapping rows}$$
$$\begin{pmatrix}
    \lambda & 0 \\
    0 & 1
\end{pmatrix}A: \text{multiplying a row by $\lambda$}$$
$$\begin{pmatrix}
    1 & \lambda \\
    0 & 1
\end{pmatrix}A: \text{Adding multiples of one row to another}$$
Thus, if the row-reduced echelon form of $A$ is equivalent to $\id$, then $A^{-1}$ exists. So $x = A^{-1}b$ from the above case. \\
To find $A^{-1}$, the most direct way is to
$$\begin{pmatrix}
    A & | & \id
\end{pmatrix} \xrightarrow{Gaussian Elimination} \begin{pmatrix}
    \id & | & A^{-1}
\end{pmatrix}$$
\newpage
\section{Producers in Short Run, 生产者短期生产模式}
No Midterm 2 Notes.
\newpage
\section{Producers in the Long Run, 生产者长期生产模式}
No Midterm 2 Notes.
\newpage
\section{Spectral Theorem, 谱定理}
\subsection{Eigenvalues and Eigenvectors, 特征值与特征向量}
\begin{definition}
    Let $A$ be a matrix with $a_{ij} \in \F$ and $A: \F^n \to \F^n$, then $\lambda \in \F$ is an eigenvalue of $A$ if
    $$\exists v \ne 0, v \in \F^n, Av = \lambda v$$
    $v$ is the eigenvector corresponding to the eigenvalue $\lambda$
\end{definition}
To find eigenvalues of a matrix $A$, we find the \textbf{\textit{characteristic polynomial}}(特征多项式):
$$\det (A - \lambda \id) = 0$$
Then, to find the eigenvectors corresponding to eigenvalue $\lambda$, we just have to find a basis for $\Ker (A - \lambda \id)$.
\subsubsection{Basis of Eigenvectors, 特征向量的基}
In order for a basis of eigenvectors to exist, we need $A$ to at least be an endomorphism $A: V \to V$, then $V$ \textbf{might} have a basis of eigenvectors of $A$. \\
The matrix $A$ with respect to the eigenvector basis is \textbf{\textit{diagonal}}(对角的). \\
We need eigenvalues to determine whether or not a basis of eigenvectors exists. Therefore, consider the characteristic polynomial
$$\begin{vmatrix}
    a_{11} - \lambda & a_{12} & \dots & a_{1n} \\
    a_{21} & a_{22} - \lambda & \dots & a_{2n} \\
    \vdots & \vdots & & \vdots \\
    a_{n1} & a_{n2} & \dots & a_{nn} - \lambda
\end{vmatrix}$$
which must have the leading term to be some $(-1)^n \lambda^n$. \\
By \textbf{Fundamental Theorem of Algebra}(代数基本定理), an $n$-degree polynomial must have $n$ roots in $\C$, counting with \textbf{\textit{algebraic multiplicity}}(代数重复度). \\
Over $\C$, any polynomial can be factored completely; but over $\R$, the complex roots will come in pairs $\{z, \overline{z}\}$, that is, odd-degree polynomials must have odd number of real roots. \\
In the complex plane, multiplying by $\lambda = \alpha \pm \beta i = |\lambda|e^{i\phi}$, means rotating $\phi$, scaling by $|\lambda| = \sqrt{a^2 + b^2}$, where $\sin{\phi} = \frac{a}{|\lambda|}, \cos{\phi} = \frac{b}{|\lambda|}$.
\subsubsection{Linear Independence of Eigenvectors, 特征向量的线性独立性}
Suppose $\{v_i\}$ is a set of eigenvectors corresponding to distinct eigenvalues $\lambda_i$, then,
\begin{quote}
    Base: $i = 1$, $\{v_1\}$ is linearly independent, \\
    Inductive Hypothesis: Assume for $i = k$ \\
    Inductive Step: we show for $i = k + 1$, \\
    suppose $\alpha_1 v_1 + \dots + \alpha_{k+1} v_{k+1} = 0$ \\
    Then 
    $$A(\alpha_1 v_1 + \dots + \alpha_{k+1} v_{k+1}) = \lambda_1 \alpha_1 v_1 + \dots + \lambda_{k+1} \alpha_{k+1} v_{k+1}$$
    Use this minus $\lambda_{k+1}$ of supposition, we have
    $$\alpha_1(\lambda_1 - \lambda_{k+1})v_1 + \dots + \alpha_k(\lambda_k - \lambda_{k+1})v_k = 0$$
    Since $\lambda_1 \ne \dots \ne \lambda_{k+1}$, then $\alpha_1 = \dots = \alpha_k = 0$, which further indicates $\alpha_{k+1} = 0$ \\
    $\{v_{k+1}\}$ is thus linearly independent.
\end{quote}

\subsection{Matrix Diagonalization, 矩阵对角化}
Over $\C$, consider a linear operator in $\C^n$ with distinct eigenvalues $\lambda_1, \dots, \lambda_n$, then the matrix of $A$ in this corresponding basis of eigenvectors is \textbf{diagonal}. \\
\subsubsection{Change-of-basis Matrix, 基变更矩阵}
Consider in $\R^2$, there is a basis $\{v_1, v_2\}$ and the standard basis $\{e_1, e_2\}$. \\
Let $v_1 = c_{11} e_1 + c_{21} e_2$, and $v_2 = c_{12} e_1 + c_{22} e_2$, then a change-of-basis matrix $C$, is thus
$$C = \begin{pmatrix}
    c_{11} & c_{12} \\
    c_{21} & c_{22}
\end{pmatrix}$$
Then, consider a general vector $x \in V$. We then have
$$x_{e_1, e_2} = C x_{v_1, v_2}$$
$$x_{v_1, v_2} = C^{-1} x_{e_1, e_2}$$
If we have a linear operator $A$ with respect to the basis $\{e_1, e_2\}$, and a linear operator $A'$ with respect to the basis $\{v_1, v_2\}$, then we should have
$$(A x_{e_1, e_2})_{v_1, v_2} = A'x_{v_1, v_2}$$
Therefore,
\begin{align*}
    (A x_{e_1, e_2})_{v_1, v_2} &= A'x_{v_1, v_2} \\
    C^{-1}A x_{e_1, e_2} &= A'C^{-1} x_{e_1, e_2} \\
    C^{-1}A &= A'C^{-1} \\
    C^{-1}AC &= A'
\end{align*}
\subsubsection{Complex Eigenvectors, 复特征向量}
If $A$ have complex eigenvectors, then over $\R$, we can bring $A$ to the form:
$$A = \begin{pmatrix}
    X & 0 & 0 \\
    0 & Y & 0 \\
    0 & 0 & Z
\end{pmatrix}$$
where each $X, Y, Z$ is a $2 \times 2$ matrix, with the form
$$\begin{pmatrix}
    |\lambda|\cos{\phi} & -|\lambda|\sin{\phi} \\
    |\lambda|\sin{\phi} & |\lambda|\cos{\phi}
\end{pmatrix}$$
Recall: $|\lambda|e^{i \phi} = |\lambda|\cos{\phi} + |\lambda|i\sin{\phi}$
\subsubsection{Non-distinct Eigenvalues, 非独特特征值}
Since we know $\det A' = \det (C^{-1}AC) = \lambda_{1}\dots \lambda_{n}$, thus:
\begin{quote}
    $\det A' \iff \exists i, \lambda_i = 0$ \\
    This means, with $0$ to be an eigenvalue, we have $\exists v \ne 0, Av = 0$, that is $\Ker A \ne \{0\}$. Then, $A$ is neither injective, surjective, nor invertible.
\end{quote}
\begin{definition}
    Define \textbf{\textit{geometric multiplicity}}(几何重复度) of eigenvalue $\lambda$ to be
    $$\dim (\Ker (A - \lambda \id))$$
    which is less than or equal to its algebraic multiplicity.
\end{definition}
When the geometric multiplicity of $\lambda$ is less than its algebraic multiplicity, then $A$ cannot be diagonalized. \\
The best we can do is to find $C$, such that $C^{-1}AC$ is of Jordan normal form.
\begin{definition}
    An $n \times n$ Jordan block is of the form
    $$J_\lambda = \begin{pmatrix}
        \lambda & 1 & 0 & \dots & 0 \\
        0 & \lambda & 1 & \dots & 0 \\
        \vdots & \vdots & \vdots & & \vdots \\
        0 & 0 & 0 & \dots & \lambda
    \end{pmatrix}$$
    One Jordan block represents the existence of $1$ eigenvector. \\
    Let
    $$J_0 = \begin{pmatrix}
        0 & 1 & 0 & \dots & 0 & 0 \\
        0 & 0 & 1 & \dots & 0 & 0 \\
        \vdots & \vdots & \vdots & & \vdots & \vdots \\
        0 & 0 & 0 & \dots & 0 & 1 \\
        0 & 0 & 0 & \dots & 0 & 0 \\
    \end{pmatrix}$$
    Then, every $J_\lambda = \lambda \id + J_0$. \\
    With $J_0^n = 0$, it is a nilpotent matrix.
\end{definition}
\begin{lemma}
    The binomial formula states that
    $$(J_\lambda)^n = (\lambda \id + J_0)^n$$
\end{lemma}
For each eigenvalue $\lambda$, we can write a Jordan form with Jordan blocks on the diagonal. The number of Jordan blocks (geometric multiplicity) is the number of eigenvectors associated with this eigenvalue. \\
Why do we care about diagonalization?
\begin{quote}
    We can find powers of $A$, with $D$ being $A$'s diagonalized matrix. \\
    $$A^m = CD^mC^{-1} = C \begin{pmatrix}
        \lambda_1^m & 0 & \dots & 0 \\
        0 & \lambda_2^m & \dots & 0 \\
        \vdots & \vdots & & \vdots \\
        0 & 0 & \dots & \lambda_n^m
    \end{pmatrix} C^{-1}$$
\end{quote}

\subsection{Self-adjoint Matrix, 自伴随矩阵}
\begin{definition}
    Let $A: V \to V$ be an endomorphism, where $V$ is a Euclidean vector space, \\
    $A$ is \textbf{\textit{self-adjoint}}(自伴随) if
    $$(Av, w) = (v, Aw)$$
\end{definition}
Therefore, when considering with respect to an orthonormal basis, the matrix $A$ is symmetric, where
$$A = A^t, a_{ij} = a_{ji}$$
This is true because $(Ae_i, e_j)$ represents $j^{th}$ component of $A$'s $i^{th}$ column ($a_{ji}$), and $(e_i, Ae_j)$ represents $i^{th}$ component of $A$'s $j^{th}$ column ($a_{ij}$).
\subsubsection{Spectral Theorem, 谱定理}
\begin{theorem}
    A self-adjoint linear operator has a basis of eigenvectors, and we can choose an \textbf{orthonormal basis} of eigenvectors. \\
    The process of finding the change-of-basis for self-adjoint linear operators is called the \textbf{\textit{Principal Axes Transformation}}.
\end{theorem}
This is:
\begin{quote}
    1. Eigenvectors corresponding to different eigenvalues are orthogonal. \\
    2. If there are "repeated eigenvalues", then part of the matrix corresponding to $\lambda$ must be $\lambda \id$.
\end{quote}
For a self-adjoint matrix, its eigenvalues must have the same geometric and algebraic multiplicity. \\
Thus, to find an orthonormal basis of eigenvectors (Principal Axes Transformation), we:
\begin{quote}
    1. Find the eigenvalue of $A$. \\
    2. For each $\lambda$, find its eigenvector(s). \\
    3. Within each $\lambda$, orthonormalize (Gram-Schimdt) the found eigenvectors. \\
    4. Write the orthonormalized eigenvectors in columns, which form the change-of-basis matrix $C$. \\
    5. By applying $C^tAC$, we can diagonalize $A$ where on the diagonal is $A$'s eigenvalues. \\
    5. Or, by applying $CDC^t$, we can find the linear transformation $A$ with respect to the standard basis.
\end{quote}
\begin{theorem}
    Spectral decomposition of self-adjoint operators: \\
    If $f: V \to V$ is self-adjoint of a finite-dimensional Euclidean vector space, and $\lambda_1, \dots, \lambda_r$ its distinct eigenvalues, and $P_k: V \to V$ the orthogonal projection onto the eigenspace $E_{\lambda_k}$, then
    $$f = \sum_{k=1}^r \lambda_k P_k$$
\end{theorem}
\section{Monopoly, 垄断市场}
\subsection{Theory of Single-price Monopoly, 单一价格垄断}
There are $3$ characteristics of a single-price monopoly structure:
\begin{quote}
    1. One seller \\
    2. Selling a unique good \\
    3. Impossible to enter or exit
\end{quote}
In this case, the firm is a \textbf{price setter}.
\subsubsection{Demand curve, 需求曲线}
Since the firm is the industry, the demand curve will be \textbf{downward sloping}. \\
The firm will then set a price to maximize profits. What price will the firm sell its goods at? \\
Firstly, a monopolist will never produce when $\epsilon < 1$, as this is equivalent to $MR < 0$, leading to profit loss. \\
Secondly, we know $P = AR = a - bQ$, then $TR = PQ = aQ - bQ^2$, so
$$MR = \dv{TR}{Q} = a - 2bQ$$
giving us the marginal revenue drop decreasing twice as fast as the demand curve.

\subsection{Short Run Profit Maximization, 短期利润最大化}
A monopoly maximizes its profits by producing when $\epsilon \ge 1$, it stays in business ($P \ge AVC$), and where $MR = MC$. \\
Case 1: Economic Profit
\begin{quote}
    1. $MR = MC$, find $Q$ at intersection. \\
    2. Find $P$ on the demand curve given the above $Q$. \\
    3. $P > AC$
\end{quote}
Case 2: Normal Profit
\begin{quote}
    1. $MR = MC$, find $P$ and $Q$ at intersection. \\
    2. Find $P$ on the demand curve given the above $Q$. \\
    3. $P = AC$
\end{quote}
Case 3: Economic Loss
\begin{quote}
    1. $MR = MC$, find $P$ and $Q$ at intersection. \\
    2. Find $P$ on the demand curve given the above $Q$. \\
    3. $AVC < P < AC$
\end{quote}
For a monopoly, when not producing at profit-maximization outputs, it does not imply an economic loss. Therefore, monopolies have the flexibility to trade off some profit for policies.

\subsection{Long Run and Barriers to Entry, 长期生产与参与阻碍}
\subsubsection{Barriers to Entry}
Profits still signal firms to enter a monopoly, so the monopoly must \textbf{impede} this incentive. \\
A \textbf{barrier to entry} impedes the entry of new firms into the market. \\
BTE will thus allow a monopoly to have long run \textbf{economic profits}. \\
The natural BTE includes:
\begin{quote}
    1. High startup costs \\
    2. Economies of Scale, Scope, Network
    \begin{quote}
        Natural Monopoly: occurs when demand is less than MES \\
        Economies of scale: LRAC falls as the firm produces more of the \textbf{same} good \\
        Economies of scope: LRAC falls as the firm produces \textbf{more than one} good \\
        Network economies: a product's value increases as more people use the product
    \end{quote}
\end{quote}
The artificial BTE includes:
\begin{quote}
    1. Government
    \begin{quote}
        Patents, franchises, charters, licenses, environmental regulations, red tape, procurement policies
    \end{quote}
    2. Firm
    \begin{quote}
        Predatory pricing, product differentiation
    \end{quote}
\end{quote}
In the long run, monopolies reach productive efficiency, but NOT allocative efficiency as there exists DWSL. \\
In the very long run, monopolies rarely persist as new ideas are created to "destroy" old ideas (creative destruction), creating economic growth, UNLESS, it is protected by the government.

\subsection{Cartels, 垄断集团}
A \textbf{cartel} is a voluntary association of producers who agree to act as a monopoly to maximize \textbf{joint profits}. \\
For example, OPEC is an oil cartel, whose joint decisions led to the oil embargo in 1972 and 1979. \\
A cartel can form through either formal collusion (formal agreements made) or tacit collusion (no formal agreements made).

\subsection{Theory of Multiprice Monopoly, 多价垄断理论}
\textbf{Price discrimination} is the \textbf{same} producer charges \textbf{different prices} for the \textbf{same} good (due to elasticity), \textbf{for reasons other than costs}. \\
The conditions for price discrimination are:
\begin{quote}
    1. Monopoly power: the seller can charge different prices \\
    2. Consumers must value different units of the same product differently \\
    3. No arbitrage (consumers buy at a low price and \textbf{resell} at a higher price), so that the buyers cannot defeat the seller's objective.
\end{quote}
An advantage of general price discrimination is that a part of the consumer's surplus is now converted to the producer's surplus. Then, a perfect price-discriminating monopoly would produce up to $Q$ where $D = MC$. In this case, the consumer surplus triangle is transformed into monopoly profits, and DWSL disappears, reaching allocative efficiency. \\
There are three types of price discrimination:
\begin{quote}
    1. First degree (Perfect price discrimination) = charging the reservation price \\
    2. Second degree = charging several different prices \\
    3. Third degree = charging different groups
\end{quote}
Price discrimination can result in the following ways:
\begin{quote}
    1. Higher profits for the monopoly \\
    2. Higher output from the monopoly \\
    3. Normative effects
    \begin{quote}
        The consumer surplus is transferred to the producer surplus, and the total surplus remains the same.
    \end{quote}
\end{quote}
Another form of price discrimination is \textbf{hurdle pricing}, which is the monopolist segmenting the market by charging reservation prices for price-sensitive buyers and the monopoly price for the rest.
\newpage
\section{Imperfect Competition, 不完全竞争}

\subsection{Structure of the Canadian Economy, 加拿大经济结构}
\textbf{Canada is a large country with a small population}
\begin{quote}
    1. Large geography - high transport costs and natural BTE
    2. Small population - low demand leads to firms not able to minimize AC, thus leading to a natural monopoly
\end{quote}
Canada decides "competition" through the Industrial Concentration Ratio, and the Competition Act states that $CR_4 > 65\%$ means monopoly power. \\
$CR_4$: Fraction of total sales controlled by the top 4 firms in the industry; the higher the concentration ratio, the higher the market power. \\
However, several problems arise from such a measurement:
\begin{quote}
    1. Definition of "relevant market": it depends on the product and geography; \\
    2. It ties competition to the number of firms; \\
    3. It \textbf{overstates} the degree of concentration in Canada because Canada is an open economy.
\end{quote}

\subsection{Imperfect Competition}
This type of market exhibits: rivalrous behaviour, with some market power, to set price, within a range
There are two market structures of imperfect competition:
\begin{quote}
    1. Monopolistic Competition: Large number of small firms and non-strategic behaviour (ignoring what other firms do)
    2. Oligopoly: Small number of large firms, and exhibit strategic behaviour
\end{quote}
The characteristics of imperfect competition include:
\begin{quote}
    1. Some to many sellers \\
    2. Products are differentiated \\
    3. Entry and exit are not easy \\
    4. Price setters within a range
\end{quote}
The firms in this market structure present these common behaviours:
\begin{quote}
    1. They select their products (differentiated - homogeneous goods in the same market but distinguished in consumer's eyes). And these products are not perfect substitutes. \\
    2. They select their prices (administered - set by individual firms within a range by referring to S and D, but not completely driven by market forces). The prices set tend to be sticky, as firms will respond to changes in demand by adjusting output not price. \\
    3. They engage in non-price competition (e.g. advertisement, warranties, services)
\end{quote}

\subsection{Monopolistic Competition, 垄断性竞争}
The characteristics of monopolistic competition include:
\begin{quote}
    1. Many sellers - non-strategic behaviour, like Perfect Competition \\
    2. Differentiated good - like Monopoly \\
    3. Significant entry and exit - like Perfect Competition \\
    4. Price setter within a range - sticky prices
\end{quote}
Therefore, firms in monopolistic competition behave like monopolies in SR, because of differentiation. \\
They behave like perfect competition because of the significant entry in LR. \\
In the short run,
\begin{quote}
    The firm will equate $MR = MC$, and find $P$ based on intersected $Q$. The firm then makes monopoly profits.
\end{quote}
In the long run,
\begin{quote}
    The firm enjoys economic profits, and signals other firms to enter, shifting the D of each firm to the left until D is tangent to AC. Firms then only make normal profits.
\end{quote}
\subsubsection{Excess Capacity Theorem}
\textbf{Excess Capacity} $=$ Capacity $Q$ - LR $Q_E$
In imperfect competition, there is an excess capacity where firms produce less at higher prices. \\
It is now accepted that consumers benefit from more brands arising from monopolistic competition.

\subsection{Oligopoly, 寡头}
The characteristics of monopolistic competition include:
\begin{quote}
    1. Several sellers\\
    2. Similar or differentiated good - present strategic behaviour, they are \textbf{interdependent} \\
    3. Formidable entry and exit - firms' output affects industry supply \\
    4. Price setter within a range - administered prices
\end{quote}
Oligopoly behaves strategically to have monopoly power over price, usually through merging and acquisition. This is to decrease rivalry and increase profits. \\
Both natural causes (high start-up costs) and artificial causes (government intervention) can lead to such a market structure.

\subsection{Game Theory, 博弈论}
A \textbf{game} is a decision-making process of $2+$ players who are interdependent and know the outcomes \\
A \textbf{simultaneous game} is a game where both players make decisions at the same time. \\
A \textbf{sequential game} is a game where one player makes a decision, then the other reacts. \\
\subsubsection{Simultaneous game, 同步博弈}
The \textbf{best response} of a player is the strategy taken when the player cannot gain utility by switching to a different strategy, given the strategy of the other player. \\
A \textbf{Nash equilibrium} is when each player is playing the best responses. An equilibrium is usually reached by rational non-cooperation. \\
A \textbf{dominant strategy} is a strategy that yields a higher payoff, regardless of the strategy of the other player.
A \textbf{prisoner's dilemma game}(囚徒困境)is a game where
\begin{quote}
    1. Each player has a dominant strategy \\
    2. that leads to a Nash equilibrium \\
    3. with a lower payoff \\
    4. than if they had not played their dominant strategy \\
    In general, they are better off cooperating but worse off following self-interest respectively.
\end{quote}
The \textbf{Pareto optimum} is a situation where one cannot make someone better off without making someone else worse off.
\begin{quote}
    1. Allocative Efficiency = Pareto optimality/efficiency \\
    2. Pareto Improvement is making someone better off without making someone else worse off \\
    3. Pareto Dis-improvement is making someone worse off without affecting others.
\end{quote}
\subsubsection{Sequential game, 序贯博弈}
A \textbf{decision tree} is a diagram that shows the possible decisions, in sequence, with the payoffs for each possible decision. \\
Player $2$ will have the best response according to Player $1$'s choice, thus, through backward induction, Player $1$ can have a certain best response. \\
An \textbf{ultimatum game} is when the first player has the power to impose a "take it or leave it" offer. A minimum acceptance threshold can be imposed by the Player $2$ in advance. \\
A \textbf{credible threat/promise} is a promise which is in the promisor's interest to perform.

\subsubsection{Problems in games and strategic role of preferences}
For a game, we may have the players:
\begin{quote}
    1. Not having full knowledge \\
    2. Not having all relevant information \\
    3. Have a commitment problem (player's inability to make a credible promise) \\
    4. Not have a commitment device (methods to create a credible promise by changing incentives) \\
    5. Motivated by self-interest \\
    6. Have an advantage because of first-mover ($2$ Nash equilibrium)
\end{quote}
The solution to a commitment problem can include:
\begin{quote}
    1. Alter the material incentives the players face \\
    2. Alter the psychological incentives the players face and ensure the other party knows of these incentives (Stockholm Syndrome)
\end{quote}

\subsection{Oligopoly in practice}
A \textbf{collusion} is an agreement to cooperate to restrict $Q$ and raise $P$:
\begin{quote}
    1. Explicit/Overt collusion - Can lead to cheating, Conflicts in determining market share of each member \\
    2. Implicit/Covert/Tacit collusion
    \begin{quote}
        May or may not cheat, \\
        1. Conscious parallelism: recognition of common interests without explicit agreement. \\
        2. Focal point pricing: MSRP.
    \end{quote}
\end{quote}
There can still be rivalry within collusion: market share competition, and 
innovation competition. \\
\textbf{Contestable markets} allow free entry and exit, and allow "hit and run entry". \\
Oligopoly structures are tradeoffs: monopoly power vs. cost saving and innovation.

\newpage
\section{Efficiency, 市场效率}
\textbf{Efficiency} is reached when costs are minimized, both to individuals (productive efficiency) and the society (allocative efficiency).

\subsection{Productive Efficiency}
This refers to minimizing input holding output constant (LRAC) or maximizing output holding input constant (PPC). \\
Two types of productive efficiency include:
\begin{quote}
    1. Firm productive efficiency: each firm on LRAC; \\
    2. Industry productive efficiency: each firm has the same MC (no reallocation of output can lower costs)
\end{quote}
If all firms and industries are producing efficiently, the economy is productively efficient, thus, on PPC (Economy Allocative Efficiency: at the optimal point of PPC).

\subsection{Allocative Efficiency}
This means Pareto efficiency/optimality, exchange efficiency, and market efficiency. \\
This is: minimizing the cost to society, referring to the mix of commodities produced, where consumer and producer coincide in preferences. \\
For allocative efficiency, we need
$$P = MB = MC$$
thus, firms must be perfectly competitive, there is no DWSL. \\
Any point on the PPC represents productive efficiency, but only one point on the PPC reaches allocative efficiency.

\subsection{Efficiency in Perfect Competition and Monopoly}
Perfect competition market structure can reach both productive efficiency and allocative efficiency. \\
Monopoly market structure can reach productive efficiency (on LRAC, just not minimum), but cannot reach allocative efficiency ($P > MC$, there is a DWSL). However, a perfectly price-discriminating monopoly can reach both. \\
Single-price monopolies may "offset" the increase in price by reducing costs, therefore, reducing certain DWSL. \\
It is important to note:
\begin{quote}
    1. Efficiency is not the only goal (equity) \\
    2. Private costs can omit social costs (lead to market failure) \\
    3. Costs are not independent of market structure (Monopolies needed for economies of scale) \\
    4. Monopolies may be needed for innovation.
\end{quote}

\newpage
\section{Government Intervention, 政府干预}
\subsection{Basic functions of government}
\textbf{Government} must have \textbf{monopoly of violence} through the military and the courts to deprive liberty. It must define and enforce "Property Rights".

\subsection{Case for the free market}
1. Free market can reach allocative efficiency
\begin{quote}
    Problem: Most markets are imperfectly competitive
\end{quote}
2. Free market coordinates automatically through the price system. \\
3. It stimulates innovation and economic growth. \\
4. It decentralizes power.

\subsection{Case for government intervention}
\subsubsection{Monopoly power}
It is inevitable because:
\begin{quote}
    Economies of scale, scope, networks \\
    Differentiated products \\
    Innovation
\end{quote}
It leads to allocative inefficiency, so the government intervenes with economic policy to balance it.
\subsubsection{Externalities, 外部性}
\textbf{Externality} is non-priced (excluded from the original D and S analysis) third-party cost or benefit. \\
A production externality affects supply, and a consumption externality affects demand. \\
A positive externality is an external economy (increase in S or D), and a negative externality is an external diseconomy (decrease in S or D). \\
\\
\textbf{Private cost} is an opportunity cost to the seller (seller: first-party, buyer: second-party). \\
\textbf{External cost} is the opportunity cost of third-party. \\
\textbf{Social cost} is private cost $+$ third-party opportunity cost to society. \\
\\
\textbf{Network externality} is the effect of an extra user on others. \\
\textbf{Pecurniary externality} is not an externality: I buy an iPhone, the price to third-party rises. \\
\\
To determine social underproduction or overproduction, we first assume perfect competition and no externalities. Then, apply the corresponding externalities to shift S or D. Compare the final $Q_E$ with the original $Q_E$. \\
\\
Some common applications of externalities include
\begin{quote}
    1. Environmental pollution \\
    2. Open access resources \\
    3. Highway congestions
\end{quote}

\subsubsection{Public goods}
\textbf{Public goods} are characterized by non-rivalry (consumption of one does not diminish consumption by another) and non-excludability (if produced, it must be consumed equally by all). \\
By such categorization, we have:
\begin{quote}
    1. Private good: rivalrous and excludable, e.g. chocolate \\
    2. Club good: non-rivalrous and excludable, e.g. art gallery, roads \\
    3. Common property good: rivalrous and non-excludable, e.g. fishery \\
    4. Public good: non-rivalrous and non-excludable, e.g. national defence
\end{quote}
The problem with club good:
\begin{quote}
    1. Non-rivalry: $MC = 0$ to produce, so $P = 0$ for efficiency; \\
    2. Excludability: allows market pricing, so $P > 0$ to cover fixed costs \\
    This leads to inefficiency
\end{quote}
The problem with public good:
\begin{quote}
    One can enjoy a positive production externality. If the good does not exist yet, this then leads to no production.
\end{quote}
The problem with common property goods:
\begin{quote}
    There is a negative consumption externality, leading to the Tragedy of the Commons, the market overuses.
\end{quote}
\subsubsection{Asymmetry of information, 信息不对称}
This means buyers and sellers have different relevant information about the goods. This leads to:
\begin{quote}
    1. Adverse selection \\
    2. Moral hazard \\
    3. Principal-agent problem \\
    4. Signaling
\end{quote}
Adverse selection is tilting the selection of goods towards poor quality before a contract is made.
\begin{quote}
    1. Seller knows more - The Lemons Problem
    \begin{quote}
        1. Increase buyer's ability to observe quality \\
        2. Incentives for truthful quality reporting \\
        3. Increasing average quality
    \end{quote}
    2. Buyer knows more - Insurance
    \begin{quote}
        1. Group policies (even out the risk) \\
        2. Screening
    \end{quote}
\end{quote}
A moral hazard is a party that is insulated from risk, after the contract is made, and therefore may behave adversely.
\begin{quote}
    1. Setting up deductible: portion of claim for which policyholder not compensated \\
    2. Co-payment: portion of the fee that the policyholder pays upfront \\
    3. Co-insurance: predetermined apportionment of claim
\end{quote}
The principal-agent problem is a type of moral hazard where the agent has the incentive to change behaviour after being hired, and the behaviour of the agent cannot be fully known. This is a sequential game.
\begin{quote}
    1. Flat rate \\
    2. Pay for performance \\
    3. Commission \\
    4. Bonuses \\
    5. Stock options \\
    6. "You're fired"
\end{quote}
A signal is a non-free message from the owner of the goods evincing unknown information.
\begin{quote}
    Education is a classic signalling example, as it signals higher productivity with a university degree. \\
    This can lead to a prisoner's dilemma effect where all students follow the dominant strategy to attend expensive schools to send better signals, but are all worse off because all spent more money for the same signal. \\
    These cases do not equate MSB (marginal social benefit) to MSC.
\end{quote}\

\subsubsection{Social goals}
These can include
\begin{quote}
    1. Income redistribution \\
    2. Merit goods: health, education \\
    3. Social obligations: military, jury \\
    4. Protecting individuals from others: minimum wage \\
    5. Protecting individuals from themselves: seat belt
\end{quote}
They are due to normative "value judgments"

\subsection{Case against government regulation}
\subsubsection{Methods}
1. Cost-benefit analysis to determine whether or not the government should provide the public good. \\
2. Problems arise
\begin{quote}
    1. Quantifying costs/benefits \\
    2. Forecasting \\
    3. Discounting future costs/benefits to present
\end{quote}
3. Methods include
\begin{quote}
    1. Public provision \\
    2. Redistribution \\
    3. Regulation
\end{quote}
4. Government can induce/impose behaviour, and change incentives.

\subsubsection{Costs of intervention}
1. Direct Resource Costs \\
2. Indirect resource costs
\begin{quote}
    Externalities: \\
    1. Costs of production: safety standards \\
    2. Costs of compliance: pay equity \\
    3. "Rent-seeking": lobbying for economic advantage
\end{quote}

\subsubsection{Causes of government failure}
1. Politicians' self-interests \\
2. "Public Choice Theory"
\begin{quote}
    1. Politicians (votes), bureaucrats (authority), and electorates (utility) do not care about the public interest \\
    2. Rational ignorance: no incentive to become informed, costs $>$ benefits
    \begin{quote}
        No need to be informed on tax lax change (cost), when a vote worth 1 in 30 mn (benefit)
    \end{quote}
    3. Democracy and inefficient public choice: "One person, one vote" fails to account for preferences, and Arrow's theory of social choice
\end{quote}
3. Government monopolies


\end{document}
