\documentclass{article}

\usepackage{amsmath, amsthm, amssymb, amsfonts, physics}
\usepackage{thmtools}
\usepackage{graphicx}
\usepackage{setspace}
\usepackage{geometry}
\usepackage{float}
\usepackage{hyperref}
\usepackage[utf8]{inputenc}
\usepackage[english]{babel}
\usepackage{framed}
\usepackage[dvipsnames]{xcolor}
\usepackage{tcolorbox}
\usepackage{ctex}
\usepackage{tikz-cd}
\usepackage{mathtools}

\colorlet{LightGray}{White!90!Periwinkle}
\colorlet{LightOrange}{Orange!15}
\colorlet{LightGreen}{Green!15}
\colorlet{LightBlue}{Blue!15}

\newcommand{\HRule}[1]{\rule{\linewidth}{#1}}
\newcommand{\R}{\mathbb{R}}
\newcommand{\Prob}{\mathbb{P}}
\newcommand{\E}{\mathbb{E}}
\newcommand{\Bern}{\text{Bern}}
\newcommand{\Bin}{\text{Bin}}
\newcommand{\Geom}{\text{Geom}}
\newcommand{\Unif}{\text{Unif}}
\newcommand{\Poisson}{\text{Poisson}}
\newcommand{\Exp}{\text{Exp}}

\declaretheoremstyle[name=Theorem,]{thmsty}
\declaretheorem[style=thmsty,numberwithin=section]{theorem}
\tcolorboxenvironment{theorem}{colback=LightGreen}

\declaretheoremstyle[name=Proposition,]{prosty}
\declaretheorem[style=prosty,numberwithin=section]{proposition}
\tcolorboxenvironment{proposition}{colback=LightGray}

\declaretheoremstyle[name=Lemma,]{lemsty}
\declaretheorem[style=lemsty,numberwithin=section]{lemma}
\tcolorboxenvironment{lemma}{colback=LightOrange}

\declaretheoremstyle[name=Definition,]{defnsty}
\declaretheorem[style=defnsty,numberwithin=section]{definition}
\tcolorboxenvironment{definition}{colback=LightBlue}

\setstretch{1.2}
\geometry{
    textheight=9in,
    textwidth=5.5in,
    top=1in,
    headheight=12pt,
    headsep=25pt,
    footskip=30pt
}

\begin{document}

% ------------------------------------------------------------------------------
% Cover Page and ToC
% ------------------------------------------------------------------------------

\title{ \normalsize \textsc{}
		\\ [2.0cm]
		\HRule{1.5pt} \\
		\LARGE \textbf{\uppercase{Basic Multivariable Calculus - MATH 226 Notes}
		\HRule{2.0pt} \\ [0.6cm] \LARGE{基本多元微积分——MATH 226 笔记} \vspace*{10\baselineskip}}
		}
\date{}
\author{\textbf{Author} \\ 
		Wenyou (Tobias) Tian \\ 田文友 \\
		University of British Columbia \\ 英属哥伦比亚大学 \\
		2023}

\maketitle
\newpage

\tableofcontents
\newpage
\section{Section 1}
\subsection{Sample Space and Probabilities}
\begin{definition}
    A \textbf{\textit{sample point}} is a possible outcome, denoted as $\omega$. \\
    A \textbf{\textit{sample space}} is the set of all sample points, denoted as $\Omega$.
\end{definition}
An \textbf{\textit{event}} is a subset of $\Omega$, with $F$ representing the set of all possible events, we then have
$$|F| = 2^{|\Omega|}$$
A probability measure is a function where:
$$\Prob: F \to [0, 1]$$
such that for an event $A \in F$, $\Prob(A)$ means the probability of event $A$ occuring. \\
It is trivial that $\Prob(\varnothing) = 0$ and $\Prob(\Omega) = 1$, where $\forall A \in F, \Prob(A) \in [0, 1]$.
\begin{theorem}
    If events $A_1, A_2, \dots$ are pairwise disjoint, that is, $\forall i \ne j, A_i \cap A_j = \varnothing$, we have
    $$\Prob(\bigcup_{i = 1}^{n} A_i) = \sum_{i = 1}^{n} \Prob(A_i)$$
\end{theorem}
We then define
\begin{definition}
    The triple $(\Omega, F, \Prob)$ is called a \textbf{\textit{probability space}}.
\end{definition}

\subsection{Random sampling}
\textbf{\textit{Sampling}} is choosing an object at random from a given set.
\begin{theorem}
    If all outcomes are equally likely, if $|\Omega| < \infty$, then
    $$\Prob(A) = \frac{|A|}{|\Omega|}$$
    where $A$ is an event.
\end{theorem}

\subsection{Infinitely many outcomes}
\begin{definition}
    Sample spaces that are finite or countably infinite are \textbf{\textit{discrete}}.
\end{definition}
When $\Omega$ is discrete, we have $\Prob(A) = \sum_{\omega \in A} \Prob(\{\omega\})$. \\
Uncountably infinite sample spaces can be the set $[0, 1]$. Notice that in this case, we \textbf{do not} have $\Prob(A) = \sum_{\omega \in A} \Prob(\{\omega\})$.

\subsection{Consequences of Rules of Probability}
\begin{definition}
    Complement of a set $A$:
    $$A^C = \{\omega \in \Omega \ | \ \omega \notin A\}$$
    Union of two sets $A$ and $B$:
    $$A \cup B = \{\omega \in \Omega \ | \ \omega \in A \lor \omega \in B\}$$
    Intersection of two sets $A$ and $B$:
    $$A \cap B = \{\omega \in \Omega \ | \ \omega \in A \land \omega \in B\}$$
    Difference of two sets $A$ and $B$:
    $$A - B = \{\omega \in \Omega \ | \ \omega \in A \land \omega \notin B\}$$
\end{definition}
We then have the following properties:
\begin{quote}
    1. $\Prob(A) = 1 - \Prob(A^C)$ \\
    2. If $A = \bigcup_{i = 1}^n A_i$, and $A_i$ are pairwise disjoint, $\Prob(A) = \sum_{i} A_i$. \\
    3. If $B \subset A$, $\Prob(B) \le \Prob(A)$. \\
    4. $\Prob(A \cup B) = \Prob(A) + \Prob(B) - \Prob(A \cap B)$ \\
    5. $\Prob(A \cup B \cup C) = \Prob(A) + \Prob(B) + \Prob(C) - \Prob(A \cap B) - \Prob(B \cap C) - \Prob(C \cap A) + \Prob(A \cap B \cap C)$
\end{quote}

\subsection{Random Variables}
\begin{definition}
    A \textbf{\textit{random variable}} (r.v.) is a function from $\Omega$ to $\R$. If we define a random variable $X$, then we know
    $$X: \Omega \to \R$$
\end{definition}
\begin{definition}
    The \textbf{\textit{probability distribution}} of a random variable $X$ is the collection of probabilities $\Prob(X \in B)$ for $B \subset \R$, where $X \in B$ is defined as
    $$\{\omega \in \Omega: X(\omega) \in B\}$$
\end{definition}
A discrete random variable takes value on a discrete set, then the \textbf{\textit{probability mass function}} (p.m.f) of a discrete random variable $X$ is defined to be the collection of probabilities
$$p(k) = \Prob(X = k)$$
for all $k$ values $X$ may take. This implies: $\Prob(X \in B) = \sum_{k \in B} p(k)$
\newpage
\section{Section 2 - Conditional Probability and Independence}
\subsection{Conditional Probability}
We write $\Prob(A | B)$ to be ``the probability that event $A$ occurs \textbf{given} event $B$ occurs".
\begin{definition}
    The probability that $A$ occurs given $B$ occurs is defined to be:
    $$\Prob(A|B) = \frac{\Prob(AB)}{\Prob(B)}$$
    where $\Prob(AB)$ represents that both $A$ and $B$ occurs.
\end{definition}
\begin{theorem}
    $\Prob(A_1A_2\dots A_n) = \Prob(A_1) \times \Prob(A_2 | A_1) \times \dots \times \Prob(A_n | A_1A_2\dots A_{n-1})$. \\
    We also have:
    $$\Prob(A) = \Prob(A|B) \cdot \Prob(B) + \Prob(A|B^C) \cdot \Prob(B^C)$$
    Furthermore, if $B_1, \dots, B_n$ is a partition of $\Omega$, then
    $$\Prob(A) = \sum_{i = 1}^n \Prob(A|B_i)\Prob(B_i)$$
\end{theorem}

\subsection{Bayes' Formula}
\begin{theorem}
    \textbf{\textit{Bayes' Formula}} is the following:
    $$\Prob(B|A) = \frac{\Prob(A|B) \cdot \Prob(B)}{\Prob(A)} = \frac{\Prob(A|B) \cdot \Prob(B)}{\Prob(A|B) \cdot \Prob(B) + \Prob(A|B^C) \cdot \Prob(B^C)}$$
\end{theorem}
This is often used to represent situations like ``false positives", where the table is the following: \par
{
\centering
\begin{tabular}{|l|l|l|} 
    \hline
    Test $\downarrow$ Actual $\rightarrow$ & Positive       & Negative        \\ 
    \hline
    Positive & True Positive  & False Positive  \\ 
    \hline
    Negative & False Negative & True Negative   \\
    \hline
\end{tabular}\par
}

\subsection{Independence}
\begin{definition}
    $A$ and $B$ are independent if and only if $\Prob(A) = \Prob(A|B)$, equivalently
    $$\Prob(AB) = \Prob(A) \times \Prob(B)$$
\end{definition}
If $A$ and $B$ are independent, then $A$ and $B^C$ are also independent. Mutually exclusive (disjoint) is not equivalent to independence. The independence between two events is essentially \textbf{proportional overlap}. For independence with more than $2$ events, say $A_1, \dots, A_n$, they are independent if and only if for any set of indices $1 \le i_1 < i_2 < \dots < i_k \le n$, we have
$$\Prob(A_{i_1}A_{i_2}\dots A_{i_k}) = \Prob(A_{i_1}) \times \Prob(A_{i_2}) \times \dots \times \Prob(A_{i_k})$$
If we have $A$, $B$ and $C$, and we only have
\begin{quote}
    1. $\Prob(AB) = \Prob(A) \times \Prob(B)$ \\
    2. $\Prob(BC) = \Prob(B) \times \Prob(C)$ \\
    3. $\Prob(CA) = \Prob(C) \times \Prob(A)$
\end{quote}
they are only \textbf{\textit{pairwise independent}}.
\begin{definition}
    Random variables $X_1, \dots, X_n$ are independent if and only if, $\forall B_1, \dots, B_n \subset \R$,
    $$\Prob(X_1 \in B_1, X_2 \in B_2, \dots, X_n \in B_n) = \prod_{i = 1}^{n} \Prob(X_i \in B_i)$$
\end{definition}
This implies that discrete random variables $X_1, \dots, X_n$ are independent if and only if, $\forall k_1, \dots, k_n \in \R$, we have
$$\Prob(X_1 = k_1, X_2 = k_2, \dots, X_n = k_n) = \prod_{i = 1}^{n} \Prob(X_i = k_i)$$

\subsection{Independent Trials}
A \textbf{\textit{trial}} is a run of an experiment, where we we consider an experiment with two outcomes: $1$ to denote success with probability $p$, and $0$ to denote failure with probability $1 - p$.
\begin{definition}
    A \textbf{\textit{Bernoulli random variable}} with parameter $p$ satisfies:
    $$X = \begin{cases}
        1 & \text{with probability } p \\
        0 & \text{with probability } 1-p
    \end{cases}$$
    This random variable represents $1$ independent trial. \\
    We denote this random variable to be $X \sim \text{Bern}(p)$
\end{definition}
If we have $n$ trials, and let random variable $X$ be the number of successes, we have
\begin{definition}
    A \textbf{\textit{binomial random variable}} with parameter $n$ and $p$ satisfies:
    $$\Prob(X = k) = \binom{n}{k}p^k(1-p)^{n-k}$$
    for $k = 0, 1, \dots, n$. \\
    We denote this random variable to be $X \sim \text{Bin}(n, p)$
\end{definition}
If $X_1, \dots, X_n \sim \text{Bern}(p)$, and they are all independent, then
$$X_1 + X_2 + \dots + X_n \sim \text{Bin}(n, p)$$
If we have unbounded trials, and let random variable $X$ be the number of trials until the $1$st success, we have
\begin{definition}
    A \textbf{\textit{geometric random variable}} with parameter $p$ satisfies:
    $$\Prob(X = k) = (1-p)^{k-1}p$$
    for $k = 1, 2, \dots$
    We denote this random variable to be $X \sim \text{Geom}(p)$
\end{definition}

\subsection{Conditional Independence}
\begin{definition}
    Events $A$ and $B$ are said to be \textbf{\textit{conditionally independent}} given event $D$ if
    $$\Prob(AB|D) = \Prob(A|D) \times \Prob(B|D)$$
\end{definition}

\newpage
\section{Section 3 - Random Variables}
\subsection{Probability Distribution}
Probability distribution for a random variable $X$ is the set of all probabilities $\{\Prob(X \in B)\}$, we use a p.m.f for discrete random variables to represent its probability distribution. What if the random variable is not discrete?
\begin{definition}
    A random variable $X$ has \textbf{\textit{probability density function}} (p.d.f) $f$ if
    $$\Prob(X \le a) = \int_{-\infty}^{a} f(x) \dd x$$
\end{definition}
If $X$ has a p.d.f, we call $X$ to be a \textbf{\textit{continuous random variable}}. A valid p.d.f must satisfy two conditions:
\begin{quote}
    1. $\displaystyle \int_{-\infty}^{\infty} f(x) \dd x = 1$ \\
    2. $f(x) \ge 0$
\end{quote}
We have the following properties of a p.d.f:
\begin{quote}
    1. $\Prob(X \in [a, b]) = \int_{a}^{b} f(x) \dd x$ \\
    2. $\Prob(X \in B) = \int_{B} f(x) \dd x$ \\
    3. $\Prob(X = k) = 0$
\end{quote}
\begin{definition}
    Let random variable $X$ have p.d.f:
    $$f(x) = \begin{cases}
        \frac{1}{b - a} & x \in [a, b] \\
        0 & x \notin [a, b]
    \end{cases}$$
    We say $X$ is a uniform variable with parameters $a, b$, where $X \sim \text{Unif}(a, b)$
\end{definition}
The intuitive meaning of $f(x)$ is the following:
$$f(a) \approx \frac{\Prob(X \in [a, a+\varepsilon])}{\varepsilon}$$

\subsection{Cumulative Distribution Function}
\begin{definition}
    The \textbf{\textit{cumulative distribution function}} (c.d.f), $F$, of a random variable $X$ satisfies:
    $$F(s) = \Prob(X \le s)$$
    for all $s \in \R$. \\
    The c.d.f completely characterizes the probability distribution of a random variable.
\end{definition}
Furthermore, if r.v. $X$ is continuous with p.d.f $f(x)$, we then have
$$F(s) = \int_{-\infty}^{s} f(x) \dd x$$
This also gives that $f(x) = F'(x)$, for where $F'$ is defined, if $F'$ is undefined, then we choose an arbitrary value.
\begin{theorem}
    If c.d.f is continuous and differentiable at all but finite number of points, then the underlying random variable is continuous.
\end{theorem}
The c.d.f of discrete r.v.s would have jumps at each available $X = k$.
\begin{theorem}
    Suppose r.v. $X$ has c.d.f $F$ which is piecewise constant, then $X$ is a discrete r.v. The values that $X$ can take are the places where $F$ has jumps. If $x$ is such a point, then $\Prob(X = x)$ is the size of the jump.
\end{theorem}
We also know that a c.d.f must be \textbf{non-negative}, \textbf{always increasing}, and the limit as it approaches $\infty$ is $1$.

\subsection{Expectation}
\begin{definition}
    The expected value of a r.v. $X$ is:
    \begin{quote}
        1. Discrete: $\E(X) = \sum_{k} kp(k)$ \\
        2. Continuous: $\E(X) = \int_{-\infty}^{\infty} xf(x)\dd x$
    \end{quote}
    It is essentially the weighted average of values that $X$ can take.
\end{definition}
Some expected values for common random variables are:
\begin{quote}
    1. $X \sim \text{Bern}(p): \E(X) = p$ \\
    2. $X \sim \text{Bin}(n, p): \E(X) = np$ \\
    3. $X \sim \text{Unif}(a, b): \E(X) = \frac{a+b}{2}$ \\
    4. $X \sim \text{Geom}(p): \E(X) = \frac{1}{p}$
\end{quote}
\begin{theorem}
    $$\E(X_1 + \dots + X_n) = \E(X_1) + \dots + \E(X_n)$$
\end{theorem}
Specifically for a geometric r.v., we have the property:
$$\E(X) = \sum_{k \ge 1} \Prob(X \ge k) = \sum_{k \ge 1} (1-p)^{k-1}$$
This is derived from expected values of non-negative r.v.s where:
\begin{quote}
    1. Discrete: $\E(X) = \sum_{k = 1}^{\infty} \Prob(X \ge k)$ \\
    2. Continuous: $\E(X) = \int_{0}^{\infty} \Prob(X \ge x) \dd x$
\end{quote}
\begin{theorem}
    Let the range of r.v. $X$ be contained in the domain of some real function $g$, then
    $$\E(g(X)) = \sum_{k} g(k)p(k)$$
    $$\E(g(X)) = \int_{-\infty}^{\infty} g(x)f(x) \dd x$$
\end{theorem}
\begin{definition}
    The $n$th \textbf{\textit{moment}} of r.v $X$ is defined to be $\E(X^n)$
\end{definition}

\subsection{Variance}
\begin{definition}
    Let $X$ be a r.v with mean $\mu$, then the \textbf{\textit{variance}} of $X$ is
    $$\text{Var}(X) = \E(X - \mu)^2$$
    We may also denote it as $\sigma^2(X)$, where $\sigma(X)$ denotes the \textbf{\textit{standard deviation}} of $X$ with $\sigma(X) = \sqrt{\text{Var}(X)}$.
\end{definition}
We can consider a function $g$, where $g(X) = (X - \mu)^2$. Then we may use Theorem 3.4 to yield corresponding formulas for discrete and continuous r.v.s. \\
Some variances of common r.v.s include:
\begin{quote}
    1. $X \sim \text{Bern}(p): \sigma^2(X) = p(1-p)$ \\
    2. $X \sim \text{Bin}(n, p): \sigma^2(X) = np(1-p)$ \\
    3. $X \sim \text{Geom}(p): \E(X) = \frac{1-p}{p^2}$
\end{quote}
\begin{theorem}
    If $X$ is constant, then $\sigma^2(X) = 0$. \\
    If $X_1, \dots, X_n$ are independent, then $\sigma^2(X_1 + \dots + X_n) = \sigma^2(X_1) + \dots + \sigma^2(X_n)$.
    $$\E(aX+b) = a\E(X) + b$$
    $$\sigma(aX+b) =a^2\sigma^2(X)$$
    $$\sigma^2(X) = \E(X^2) - (\E(X))^2$$
\end{theorem}

\subsection{Gaussian Distribution}
\begin{definition}
    R.v. $Z$ has \textbf{\textit{standard normal distribution}} if it has p.d.f:
    $$\phi(x) = \frac{1}{\sqrt{2\pi}}e^{-\frac{x^2}{2}}$$
    We denote $Z \sim N(0, 1)$
\end{definition}
The corresponding c.d.f is thus:
$$\Phi(s) = \Prob(Z \le s) = \int_{-\infty}^{s} \phi(x) \dd x$$
$Z$ is called standard normal because $\E(Z) = 0, \sigma^2(Z) = 1$. For the entire family of normal r.v.s, just consider $X = \sigma Z + \mu$, then we would have $X \sim N(\mu, \sigma^2)$, with $\E(X) = \mu$ and $\sigma^2(X) = \sigma^2$. \\
Then for such a normal r.v. $X$, we have
$$F(s) = \Prob(X \le s) = \Prob(Z \le \frac{s - \mu}{\sigma}) = \Phi(\frac{s-\mu}{\sigma})$$
$$f(s) = \dv{s}F(s) = \frac{1}{\sigma} \cdot \phi(\frac{s-\mu}{\sigma}) = \frac{1}{\sigma\sqrt{2\pi}}e^{-\frac{(s - \mu)^2}{2\sigma^2}}$$
If $X \sim N(\mu, \sigma^2)$ and $Y = aX+b$, then we have
$$Y \sim N(a\mu + b, a^2\sigma^2)$$
Furthermore, we have a special property for the standard normal c.d.f:
$$\Phi(t) = 1 - \Phi(-t)$$

\newpage
\section{Section 4}
\subsection{Law of Large Numbers (LLN), Binomial Version}
Consider $X_i \sim \text{Bern}(p)$, and $S_n = X_1 + \dots + X_n$ all of which are independent. We already know $S_n \sim \text{Bin}(n, p)$, but as $n$ gets large, it becomes very hard to calculate, thus we use binomial approximation:
\begin{quote}
    1. First order approximation: $S_n \approx \E(S_n) = np$ \\
    2. Second order approximation (deviation from mean is Gaussian):
    $$S_n \approx \E(S_n) + \sqrt{\sigma^2(S_n)}\cdot N(0, 1) = np + \sqrt{np(1-p)} \cdot N(0, 1)$$
\end{quote}
\begin{theorem}
    The \textbf{\textit{Law of Large Numbers}} claim:
    $$\forall \varepsilon > 0, \lim_{n \to \infty} \Prob(|\frac{S_n}{n} - p| < \varepsilon) = 1$$
    This is equivalent to saying:
    $$\forall \delta > 0, \exists N, n \ge N, 1 - \delta \le \Prob(|\frac{S_n}{n} - p| < \varepsilon) \le 1$$
    We can call this $\frac{S_n}{n}$ converges to $p$ in probability.
\end{theorem}
We define $\frac{S_n}{n}$ to be \textbf{\textit{proportion of success}}, with expected value $p$. \\
Note that for any r.v. $X$, $\{|X| \le a\} \subset \{X \le a\}$.

\subsection{Central Limit Theorem (CLT), Binomial Version}
Let r.v. $S_n \sim \text{Bin}(n, p)$ with $\E(S_n) = np$ and $\sigma^2(S_n) = np(1-p)$, we standardize it to another r.v. $Q_n$ with:
$$Q_n = \frac{S_n - np}{\sqrt{np(1-p)}}$$
so $\E(Q_n) = 0$ and $\sigma^2(Q_n) = 1$. \\
Then $Q_n$ converges to $N(0, 1)$ in distribution.
\begin{theorem}
    The \textbf{\textit{Central Limit Theorem}} (CLT)) claim:
    $$\lim_{n \to \infty} \Prob(Q_n \in [a, b]) = \Phi(b) - \Phi(a)$$
\end{theorem}
The normal approximation goes from CLT saying:
$$\frac{S_n - np}{\sqrt{np(1-p)}} \to_d N(0, 1)$$
This means for large $n$, $S_n$ is approximately $N(np, np(1-p))$, the approximation is accurate if $np(1-p)>10$. \\
Consider a random walk problem, where one takes 1 step left or right each with $\frac{1}{2}$ probability, about how far is the person from home after $n$ steps, assuming the person starts from the house. \\
Let
$$X_i = \begin{cases}
    1 & \frac{1}{2} \\
    -1 & \frac{1}{2}
\end{cases}$$
and $Y_i \sim \text{Bern}(\frac{1}{2})$, then $X_i = (Y_i - \frac{1}{2}) \times 2$. Then if we let $Z_n$ be the distance after $n$ steps, we have
$$Z_n = \sum_{i = 1}^{n} X_i = 2 \sum_{i = 1}^{n} (Y_i - \frac{1}{2}) = -n + 2\sum_{i = 1}^{n} Y_i = -n + 2S_n$$
Then by normal approximation, we have $Z_n \approx -n + 2 \cdot N(\frac{n}{2}, \frac{n}{4}) = -n + 2(\frac{n}{2} + \frac{\sqrt{n}}{2} \cdot N(0, 1)) = \sqrt{n} \cdot N(0, 1)$. \\
The concentration of standard normal r.v. is as follows:
\begin{quote}
    1. $\Prob(N(0, 1) \in [-1, 1]) \approx 0.68$ \\
    2. $\Prob(N(0, 1) \in [-2, 2]) \approx 0.95$ \\
    3. $\Prob(N(0, 1) \in [-3, 3]) \approx 0.997$
\end{quote}
We then know with normal approximation:
$$\text{Bin}(n, p) \in [np - 3\sqrt{np(1-p)}, np + 3\sqrt{np(1-p)}]$$

\subsection{Application of Normal Approximation and Confidence Intervals}
To find a 95\% confidence interval, we first start with an estimator $\hat{\mu}$. We express this estimator as some normal r.v. with information about $\mu$. We then find the distribution of $\hat{\mu} - \mu$ and build relationship with a standard normal distribution $N(0, 1)$. Finally, we may use the concentration of a standard normal variable to find the 95\% confidence interval. \\
Recall with normal approximation $\Prob(\text{Bin}(n, p) \in [a, b]) \approx \Prob(N(np, np(1-p)) \in [a, b])$, then for large $n$, we should have for $Z \sim N(0, 1)$,
$$S_n \approx np + \sqrt{np(1-p)} \cdot Z$$
and $\hat{p} = \frac{S_n}{n}$. So $\hat{p} - p \approx \sqrt{\frac{p(1-p)}{n}} \cdot Z$.
\begin{align*}
    \Prob(|\hat{p} - p| \le \varepsilon) &\approx \Prob(|Z| \le \frac{\varepsilon \sqrt{n}}{\sqrt{p(1-p)}}) \\
    &\ge \Prob(|Z| \le 2\varepsilon\sqrt{n}) \{\sqrt{p(1-p)} \le \frac{1}{2}\}
\end{align*}
For a 95\% confidence interval, we choose $\varepsilon = \frac{1}{\sqrt{n}}$, then we know:
$$p \in [\hat{p} - \frac{1}{\sqrt{n}}, \hat{p} + \frac{1}{\sqrt{n}}]$$

\subsection{Poisson Random Variable}
\begin{definition}
    A discrete r.v. $X$ has \textbf{\textit{Poisson}} distribution with parameter $\lambda > 0$ if:
    $$\Prob(X = k) = \frac{\lambda^k}{k!}e^{-\lambda}$$
    We denote $X \sim \text{Poisson}(\lambda)$.
\end{definition}
\begin{theorem}
    Let $\lambda > 0$, $k \in \mathbb{Z}^+$, then:
    $$\lim_{n \to \infty} \Prob(\text{Bin}(n, \frac{\lambda}{n}) = k) = \Prob(\text{Poisson}(\lambda) = k)$$
\end{theorem}
The p.m.f of $\text{Bin}(n, \frac{\lambda}{n})$ converges to the p.m.f of $\text{Poisson}(\lambda)$. \\
$\text{Bin}(n, p) \approx \text{Poisson}(np)$ for $n$ large and $p$ small.
\begin{theorem}
    Let $X \sim \text{Bin}(n, p)$, and $Y \sim \text{Poisson}(np)$, then for any $A \subset \mathbb{Z}^+$, we have
    $$|\Prob(X \in A) - \Prob(Y \in A)| \le np^2$$
\end{theorem}
The properties of a Poisson r.v. include:
\begin{quote}
    Let $X \sim \text{Poisson}(\lambda)$, $Y \sim \text{Poisson}(\alpha)$
    1. $\E(X) = \lambda, \sigma^2(X) = \lambda$ \\
    2. $X + Y \sim \text{Poisson}(\lambda + \alpha)$
\end{quote}
Often it is natural to model as Poisson even without knowing $n$ or $p$ for
underlying Binomial distribution. We only need to know $np = \lambda$.

\subsection{Exponential Distribution}
An exponential r.v. models continuous waiting time, analagous to geometric r.v. modelling discrete waiting time.
\begin{definition}
    A continuous r.v. $X$ has \textbf{\textit{exponential distribution}} with parameter $\lambda > 0$ if its p.d.f, $f$, is:
    $$f(x) = \begin{cases}
        \lambda e^{-\lambda x} & x \ge 0 \\
        0 & x < 0
    \end{cases}$$
    We write $X \sim \text{Exp}(\lambda)$
\end{definition}
Thus, we have $\Prob(X > t) = e^{-\lambda t}$, giving us the c.d.f, $F$, to be:
$$F(s) = 1 - e^{-\lambda s}$$
A useful property is that $X \sim \text{Exp}(\alpha), Y \sim \text{Exp}(\beta)$, and they are independent, if we define $Z = \min(X, Y)$, then $Z \sim \text{Exp}(\alpha + \beta)$. \\
Other general properties include:
\begin{quote}
    1. $\E(X) = \frac{1}{\lambda}, \sigma^2(X) = \frac{1}{\lambda^2}$ \\
    2. $\alpha X \sim \text{Exp}(\frac{\lambda}{\alpha})$ \\
    3. Memorylessness: $\Prob(X > s + t \ | \ X > t) = \Prob(X > s)$
\end{quote}

\newpage
\section{Section 5}
\subsection{Moment Generating Function}
\begin{definition}
    The \textbf{\textit{moment generating function}} of a r.v. $X$ is the function $M: \R \to \R^+$ defined by:
    $$M(t) = M_X(t) = \E(e^{tX})$$
\end{definition}
The m.g.f of some common r.v.s include:
\begin{quote}
    1. $X \sim \Bern(p)$: $M(t) = 1 - p + pe^t$ \\
    2. $X \sim \Unif(0, 1)$: $M(t) = \frac{1}{t}(e^t - 1)$ \\
    3. $X \sim N(0, 1)$: $M(t) = e^{\frac{1}{2}t^2}$ \\
    4. $X \sim \Exp(\lambda)$:
    $$M(t) = \begin{cases}
        \frac{\lambda}{\lambda - t} & t < \lambda \\
        \infty & t \ge \lambda
    \end{cases}$$
\end{quote}
\begin{theorem}
    M.g.f can generate moments in the following way:
    $$M^{(n)}(0) = \E(X^n)$$
    That is
    $$\dv[n]{}{t}M(t)|_{t = 0} = \E(X^n)$$
\end{theorem}
We say that $X$ and $Y$ are equal in distribution if $\forall B \subset \R$, $\Prob(X \in B) = \Prob(Y \in B)$.
\begin{theorem}
    If $M_X(t) = M_Y(t) \ne \infty$, we say $X \stackrel{\text{dist.}}{=} Y$. \\
    More generally, if $\exists \delta > 0$, $\forall t \in (-\delta, \delta)$, $M_X(t) = M_Y(t) \ne \infty$, we say $X \stackrel{\text{dist.}}{=} Y$.
\end{theorem}

\subsection{Distribution of a Function of a Random Variable}
\begin{theorem}
    Let $X$ be a discrete r.v. with p.m.f $p_X$, $g$ be an one-to-one function with $Y = g(X)$, then $Y$ has p.m.f satisfying $p_Y(g(k)) = p_X(k)$
\end{theorem}
If $X$ is still discrete, but $g$ is no longer $one-to-one$, then we have
$$p_Y(l) = \sum_{k: g(k) = l} p_X(k)$$
For continuous random variables, we derive its c.d.f for $g(X)$. \\
For affine transformations, that is $Y = aX + b$, we should have
$$f_Y(s) = \frac{1}{|a|} \times f_X(\frac{s - b}{a})$$
If $g$ is one-to-one and differentiable for continuous r.v. $X$, and $Y = g(X)$, we would have:
$$f_Y(y) = \frac{f_X(g^{-1}(y))}{|g'(g^{-1}(y))|}$$
We also need $g$ have derivative at only finite number of points and $g^{-1}$ existing. For other points, $f_Y(y) = 0$.
If $g$ is only differentiable and has derivative at finitely many number of points, with $Y = g(X)$, we have:
$$f_Y(y) = \sum_{x: g(x) = y, g'(x) \ne 0} \frac{f_X(x)}{|g'(x)|}$$

\newpage
\section{Section 6 - Joint Distribution of Random Variables}
\subsection{Joint Distribution of Discrete Random Variables}
\begin{definition}
    Let $X, Y$ be discrete r.v.s defined on the same sample space, then the \textbf{\textit{joint p.m.f}} is defined by:
    $$p(k, l) = \Prob(X = k, Y = l)$$
    for all values of $k$ and $l$.
\end{definition}
\begin{definition}
    Given a joint p.m.f of $p(k, l)$, the p.m.f of $X$ is defined by:
    $$p_X(k) = \sum_{l}p(k, l)$$
    where $p_X$ is the \textbf{\textit{marginal p.m.f}} of $X$
\end{definition}
\begin{theorem}
    Let $g: \R^2 \to \R$, and let $X, Y$ be discrete r.v. with joint p.m.f $p$, then:
    $$\E(g(X, Y)) = \sum_{k, l} g(k, l)p(k, l)$$
\end{theorem}
A famous example in game theory is the situation of Prisoner's Dilemma.

\subsection{Jointly Continuous Random Variables}
\begin{definition}
    R.v.s $X, Y$ are \textbf{\textit{jointly continuous}} if $\exists f: \R^2 \to \R^+$ such that $\forall B \subset \R^2$, we have
    $$\Prob((X, Y) \in B) = \iint_B f(x, y) \dd x \dd y$$
\end{definition}
We need:
\begin{quote}
    1. $f(x, y) \ge 0$ \\
    2. $\int_{-\infty}^{\infty}\int_{-\infty}^{\infty} f(x, y) \dd x \dd y = 1$
\end{quote}
\begin{definition}
    Let $g: \R^2 \to \R$, the expected value of $g$ can be characterized by:
    $$\E(g(X, Y)) = \int_{-\infty}^{\infty}\int_{-\infty}^{\infty} g(x, y) f(x, y) \dd x \dd y$$
\end{definition}
\textbf{If ($X, Y$) are jointly continuous, then $X, Y$ are continuous r.v.s}. However, if $X, Y$ are continuous, $(X, Y)$ need not be jointly continuous.
\begin{definition}
    Let $(X, Y)$ be jointly continuous with joint p.d.f $f$, then $X$ is continuous with p.d.f $f_X$ satisfying:
    $$f_X(x) = \int_{-\infty}^{\infty} f(x, y) \dd y$$
\end{definition}
We specifically analyze uniform r.v.s in higher dimensions. Consider $D \subset \R^2$ with $A_D < \infty$, then $(X, Y)$ has uniform distribution on $D$ if:
$$f(x, y) = \begin{cases}
    \frac{1}{A_D} & (x, y) \in D \\
    0 & (x, y) \notin D
\end{cases}$$
Let $(X_1, \dots, X_n) \sim \Unif(\sqrt{n}B_n)$, with $B_n = \{(x_1, \dots, x_n): x_1^2 + \dots + x_n^2 \le n\}$, then we first have:
$$f(x_1, \dots, x_n) = \begin{cases}
    \frac{1}{\text{Vol}(\sqrt{n}B_n)} & (x_1, \dots, x_n) \in \sqrt{n}B_n \\
    0 & (x_1, \dots, x_n) \notin \sqrt{n}B_n
\end{cases}$$
For $r > 0$, we also have $\text{Vol}(rB_n) = r^n\text{Vol}(B_n)$. \\
The marginal of $X_1$ is derived to be ($V_i = \text{Vol}(B_i)$):
$$f_{X_1}(x_1) = \frac{V_{n-1} \cdot n^{\frac{n}{2}}}{V_n \sqrt{n}^n \sqrt{n - x_1^2}} \cdot (\frac{n - x_1^2}{n})^{\frac{n}{2}}$$
when taking $n \to \infty$, it converges to $\phi(x_1)$

\subsection{Joint Distribution and Independence}
For $X, Y$ be discrete r.v.s with joint p.m.f $p$ and marginal $p_X$, $p_Y$, $X$ and $Y$ are independent if and only if:
$$p(k, l) = p_X(k) \times p_Y(l)$$
Analagously, for $X, Y$ be continuous r.v.s with joint p.d.f $f$ and marginal $f_X$, $f_Y$, $X$ and $Y$ are independent if and only if:
$$f(x, y) = f_X(x) \times f_Y(y)$$

\newpage
\section{Section 7}
\subsection{Sums of Independent Random Variables}
\begin{definition}
    Let $X, Y$ be independent r.v.s, we have $X + Y$ satisfying:
    \begin{quote}
        1. Discrete: $p_{X + Y}(n) = \sum_{k} p_X(k)p_Y(n-k) = p_X * p_Y(n) = p_Y * p_X(n)$ \\
        2. Continuous: $f_{X + Y}(z) = \int_{-\infty}^{\infty} f_X(x)f_Y(z-x)\dd x = f_X * f_Y(z)$
    \end{quote}
\end{definition}
Here $p_X * p_Y$ is the \textbf{\textit{convolution}} of $p_X$ and $p_Y$, $f_X * f_Y$ is the \textbf{\textit{convolution}} of $f_X$ and $f_Y$.

\newpage
\section{Section 8 - Expectation and Variance in Multivariate Setting}
\subsection{}
\subsection{Expectation and Variance for Sums of r.v.s}
\begin{theorem}
    Regardless of $X, Y$ being independent or not, we have:
    $$\E(X+Y) = \E(X) + \E(Y)$$
    $$\E(f(X) + g(Y)) = \E(f(X)) + \E(g(Y))$$
    For $X, Y$ being independent r.v.s, we have:
    $$\E(X \cdot Y) = \E(X) \cdot \E(Y)$$
    $$\sigma^2(X + Y) = \sigma^2(X) + \sigma^2(Y)$$
    $$\E(\frac{X}{Y}) = \E(X) \cdot \E(\frac{1}{Y})$$
\end{theorem}
\begin{theorem}
    Let $X_1, \dots, X_n$ be independently identically distributed r.v.s with $\E(X_i) = \mu$ and $\sigma^2(X_i) = \sigma^2$, if we define:
    $$\bar{X}_n = \frac{X_1 + \dots + X_n}{n}$$
    we have $\E(\bar{X}_n) = \mu$ and $\sigma^2(\bar{X}_n) = \frac{\sigma^2}{n}$
\end{theorem}
\begin{definition}
    Let $A$ be an event, the indicator of that event $1_A$ is a random variable satisfying:
    $$1_A = \begin{cases}
        1 & \text{A happens} \\
        0 & \text{A does not happen}
    \end{cases}$$
\end{definition}
We know that $1_A \sim \Bern(p)$, with $p = \Prob(A)$ and $\E(1_A) = \Prob(A)$

\subsection{Sums and Moment Generating Functions}
We know functions of $X, Y$ are independent if $X, Y$ are independent. Then, if $X, Y$ are independent, $M_{X+Y}(t) = M_X(t)M_Y(t)$. That is the m.g.f of sum is product of m.g.f. \\
For $X \sim N(0, \sigma^2)$, we have $M_X(t) = e^{\frac{t^2\sigma^2}{2}}$. \\
Since $X \sim \Bin(n, p)$ and we know $X \stackrel{\text{dist.}}{=} X_1 + \dots + X_n$ with $X_i \sim \Bern(p)$, so we know
$$M_X(t) = (1 - p + pe^t)^n$$
For Poisson variables $X \sim \Poisson(\alpha)$, we have
$$M_X(t) = \exp(\alpha(e^t - 1))$$
And in this case, if $X_1, \dots, X_n \sim \Poisson(1)$, we know $\sum_{i} X_i \sim \Poisson(n)$

\subsection{Covariance and Correlation}
To say something about the level of dependence between $X$ and $Y$, we can look into the covariance of $X$ and $Y$.
\begin{definition}
    Let $X, Y$ be r.v.s with $\mu_X, \mu_Y$, we have
    $$\text{Cov}(X, Y) = \E((X - \mu_X)(Y - \mu_Y)) = \E(XY) - \E(X)\E(Y)$$
\end{definition}
If $X$ and $Y$ are independent, then the covariance of $X$ and $Y$ is $0$. However, the other way may not stand, since if $f$ is an even function, $X \sim (-a, a)$, then $\text{Cov}(X, f(X)) = 0$. \\
The properties of covariances include:
\begin{quote}
    1. $|\text{Cov}(X,Y)| \le \sqrt{\sigma^2(X)\sigma^2(Y)}$ \\
    2. $\text{Cov}(X, Y) = \text{Cov}(Y, X)$ \\
    3. $\text{Cov}(aX + bZ, Y) = a \text{Cov}(X, Y) + b \text{Cov}(Z, Y)$ \\
    4. $\text{Cov}(X, X) = \sigma^2(X)$
\end{quote}
Consider two events $A, B$ and their corresponding indicator variables $1_A, 1_B$, then we have
$$\text{Cov}(1_A, 1_B) = \E(1_A1_B) - \E(1_A)\E(1_B) = \Prob(AB) - \Prob(A)\Prob(B)$$
Thus, $\text{Cov}(1_A, 1_B) = 0 \iff$ $A, B$ are independent.
\begin{theorem}
    For r.v.s $X_1, \dots, X_n$, we have
    $$\sigma^2(\sum_{i = 1}^{n} X_i) = \sum_{i = 1}^{n} \sigma^2(X_i) + 2 \sum_{1 \le i < j \le n} \text{Cov}(X_i, X_j)$$
\end{theorem}
\begin{definition}
    $$\text{Corr}(X, Y) = \frac{\text{Cov}(X, Y)}{\sqrt{\sigma^2(X)\sigma^2(Y)}}$$
    We also know it is $\in [-1, 1]$
\end{definition}
Then $Y = aX + b \iff \text{Corr}(X, Y) = 1$. \\
Furthermore, if $X_i$ and $X_j$ are pairwise uncorrelated, then we still have:
$$\sigma^2(\sum_{i = 1}^{n} X_i) = \sum_{i = 1}^{n} \sigma^2(X_i)$$
\newpage
\section{Section 9}
\subsection{Estimating Tail Probabilities}
\begin{theorem}
    \textbf{\textit{Markov Inequality}} states that, for a non-negative random variable $X$ and some $t > 0$, we have
    $$\Prob(X \ge t) \le \frac{\E(X)}{t}$$
\end{theorem}
The proof relies on $X \ge t \cdot 1_{[X \ge t]}$.
\begin{theorem}
    \textbf{\textit{Chebychev's Inequality}} states that, for a random variable $X$ with $\E(X) = \mu$ and $\sigma^2(X) = \sigma^2$, then for some $t > 0$, we have:
    $$\Prob(|X - \mu| \ge t) \le \frac{\sigma^2}{t^2}$$
\end{theorem}
For $X_i$ be uncorrelated (independence suffices) with $\E(X_i) = \mu$ and $\sigma^2(X_i) = \sigma^2$, then define $S_n = X_1 + \dots + X_n$, we have
$$\Prob(|S_n - n\mu| \ge \sqrt{n}t) \le \frac{\sigma^2}{t^2}$$
$$\Prob(|\frac{S_n}{n} - \mu| \ge \frac{t}{\sqrt{n}}) \le \frac{\sigma^2}{t^2}$$

\subsection{}
\subsection{Law of Large Numbers, Central Limit Theorem, General Version}
\begin{theorem}
    Let $X_1, \dots, X_n$ be iid with $\E(X_i) = \mu, \sigma^2(X_i) = \sigma^2 < \infty$, let
    $$\bar{X}_n = \frac{X_1 + \dots + X_n}{n}$$
    then $\forall \varepsilon > 0$,
    $$\lim_{n \to \infty} \Prob(|\bar{X}_n - \mu| \le \varepsilon) = 1$$
    $$\lim_{n \to \infty} \Prob(|\bar{X}_n - \mu| > \varepsilon) = 0$$
\end{theorem}
If we only focus on a small interval around $\mu$, we ask about the distribution of $\bar{X}_n$. We examine $Z_n = \frac{\sqrt{n}}{\sigma}(\bar{X}_n - \mu)$, so that $\E(Z_n) = 1$ and $\sigma^2(Z_n) = 1$.
\begin{theorem}
    Let $X_1, \dots, X_n$ be iid with $\E(X_i) = \mu, \sigma^2(X_i) = \sigma^2 < \infty$, let
    $$Z_n = \frac{1}{\sigma \sqrt{n}} \sum_{i = 1}^{n} (X_i - \mu)$$
    Then $Z_n \to_d N(0, 1)$, $\forall a \le b \in (-\infty, \infty)$, we have
    $$\lim_{n \to \infty} \Prob(Z_n \in [a, b]) = \Phi(b) - \Phi(a)$$
\end{theorem}
This means a normalized sum of many iid r.v.s is almost $N(0, 1)$, that is
$$\frac{1}{\sqrt{n}} \sum \pm 1 \approx N(0, 1)$$
The normal approximation start with $X_1, \dots, X_n$ being iid, with $\E(X_i) = \mu$ and $\sigma^2(X_i) = \sigma^2$, let $S_n = X_1 + \dots + X_n$, thus, we know $\E(S_n) = n\mu$ and $\sigma^2(S_n) = n\sigma^2$, so if $n$ is large
$$S_n \approx N(n\mu, n\sigma^2) = n \mu + \sqrt{n}\sigma N(0, 1)$$
\begin{theorem}
    $$|\Prob(Z_n \le x) - \Phi(x)| \le \frac{3\E|X - \mu|^3}{\sigma^3 \sqrt{n}}$$
\end{theorem}
The tail decay for normal random variables has:
$$\Prob(N(0, 1) \ge t) \le e^{-\frac{t^2}{2}}$$

\newpage
\section{Section 10}
\subsection{Conditional Distribution of Discrete Random Variables}
\begin{definition}
    Let $X$ be a discrete r.v., $B$ be an event with $\Prob(B) > 0$, the conditional p.m.f of $X$ given $B$ is defined as:
    $$p_{X|B}(k) = \Prob(X = k \ | \ B) = \frac{\Prob(X = k, B)}{\Prob(B)} = \frac{\Prob(B \ | \ X = k)\Prob(X = k)}{\Prob(B)}$$
\end{definition}
\begin{definition}
    Following the same setup, the expected value of $X$ given $B$ is
    $$\E(X \ | \ B) = \sum_{k} k \times p_{X | B}(k)$$
\end{definition}
\begin{theorem}
    The \textbf{\textit{Law of Total Probability}} states that, if $B_1, \dots, B_n$ is a partition of $\Omega$, then
    $$\Prob(A) = \sum_{i} \Prob(A \ | \ B_i)\Prob(B_i)$$
    This gives us
    $$p_X(k) = \sum_{i = 1}^{n} p_{X | B_i}(k) \cdot \Prob(B_i)$$
    bringing the expected value to be
    $$\E(X) = \sum_{i = 1}^{n} \E(X \ | \ B_i) \cdot \Prob(B_i)$$
\end{theorem}
We then transition from conditioning on events to conditioning on r.v.s. Then,
\begin{definition}
    Let $Y$ be another discrete r.v., so conditioning on $\{Y = y\}$, then
    $$p_{X|Y}(x|y) = \Prob(X = x \ | \ Y = y) = \frac{p_{X, Y}(x, y)}{p_Y(y)}$$
\end{definition}
So this gives the conditional expected value of $X$ conditional on $Y = y$, we have
$$\E(X \ | \ Y = y) = \sum_{x} x \cdot p_{X|Y}(x|y)$$
Since $\{Y = y\}$ form a partition, then we have:
$$p_X(x) = \sum_{y} p_{X|Y}(x|y) p_Y(y)$$
$$\E(X) = \sum_{y} \E(X \ | \ Y = y) \cdot p_Y(y)$$
By Bayes' Formula, we also have:
$$p_{X|Y}(x|y) = \frac{p_{Y|X}(y|x)p_X(x)}{p_Y(y)}$$

\subsection{Conditional Distribution for Continuous Random Variables}
Analagous to discrete cases, 
\begin{definition}
    Let $X, Y$ be jointly continuous, the conditional function of $X$ given $Y = y$ is defined as:
    $$f_{X|Y}(x|y) = \frac{f_{X, Y}(x, y)}{f_Y(y)}$$
    The \textbf{\textit{conditional expectation}} of $g(X)$ given $Y = y$ is defined as:
    $$\E(g(X) \ | \ Y = y) = \int_{-\infty}^{\infty} g(x) \cdot f_{X|Y}(x|y) \dd x$$
    The \textbf{\textit{conditional probability}} of $X \in A$ given $Y = y$ is defined as:
    $$\Prob(X \in A \ | \ Y = y) = \int_A f_{X|Y}(x|y) \dd x$$
\end{definition}
Then by definition, 
$$f_X(x) = \int_{-\infty}^{\infty} f_{X|Y}(x|y) \cdot f_Y(y) \dd y$$
$$\E(g(X)) = \int_{-\infty}^{\infty} \E(g(X) \ | \ Y = y) \cdot f_Y(y) \dd y$$

\subsection{Conditional Expectation}
From previous definitions, we can consider $\E(X \ | \ Y = y) = v(y)$ to be a function of $y$. Then we transform this to $v(Y)$ so that $\E(X \ | \ Y) = v(Y)$ and it becomes a r.v.
\begin{theorem}
    The \textbf{\textit{Law of Iterated Expectation}} states that:
    $$\E(X) = \E(\E(X | Y))$$
\end{theorem}
Conditional expectation respects linearity, thus, we have
$$\E(X_1 + X_2 \ | \ Y) = \E(X_1 \ | \ Y) + \E(X_2 \ | \ Y)$$
Independence means that conditioning has no effect, thus, we should have
$$p_{X|Y}(x|y) = p_X(x)$$
$$f_{X|Y}(x|y) = f_X(x)$$
if and only if $X, Y$ are independent. This also adds that:
$$\E(X \ | \ Y = y) = \E(X), \E(X \ | \ Y) = \E(X)$$
Some properties include:
\begin{quote}
    1. $\E(X \ | \ X = x) = x$ \\
    2. $\E(g(X) \ | \ X = x) = g(x)$ \\
    3. $\E(X \ | \ X) = X$ \\
    4. $\E(g(X) \ | \ X) = g(X)$ \\
    5. $\E(g(X)f(Z) \ | \ X) = g(X) \cdot \E(f(Z)\ | \ X)$
\end{quote}

\newpage

\end{document}