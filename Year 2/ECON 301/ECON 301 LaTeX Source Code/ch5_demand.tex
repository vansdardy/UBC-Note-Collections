\section{Demand(需求)}
\subsection{Properties of Demand Functions(需求函数的性质)}
\subsubsection{Change in Income(收入变化)}
\begin{definition}
    The \textbf{\textit{income expansion path(收入扩充曲线)}} is the locus of optimal buntdles traced out as $m$ changes while keeping $p_1, p_2$ constant.
\end{definition}
\begin{definition}
    The \textbf{\textit{Engel curves(恩格尔曲线)}} represent the relationship between $m$ and $q_1^{*}, q_2^{*}$ respectively.
\end{definition}
For Cobb-Douglas preference, we have
$$m = p_1q_1^{*} \cdot \frac{\alpha + \beta}{\alpha}$$
$$m = p_2q_2^{*} \cdot \frac{\alpha + \beta}{\beta}$$
where they are linear Engel curves. We also have
$$q_2^{*} = \frac{\beta}{\alpha} \cdot \frac{p_1}{p_2} \cdot q_1^{*}$$\
as the IEP. Notice that the slope of IEP represents the $MRS$ is constant. \\
The Leontief preference has the IEP connecting all the corner bundles. \\
The linear preference has the IEP aligned either along the $x$-axis, or the $y$-axis, such that the Engel curves would be combination of vertical line along the $y$-axis, and a linear line. \\
The Cobb-Douglas, Leontief, and linear preferences are considered \textbf{\textit{homothetic preferences(同源偏好)}}, where they all have
\begin{quote}
    1. Linear Engel curves (through the origin) \\
    2. Fixed proportion of income spent on each good \\
    3. Same $MRS$ along a ray through the origin, that is $MRS = f(\frac{q_2}{q_1})$
\end{quote}
For normal goods, if $q^{*}$ grows more rapidly than $m$, then it is a luxury good; if $q^{*}$ grows less rapidly than $m$, then it is a necessary good. \\
For quasilinear preference, the IEP would be a vertical line at $\bar{q_1}$, the threshold quantity.

\subsubsection{Change in Price(价格变化)}
\begin{definition}
    The \textbf{\textit{price consumption curve(价格消费曲线)}} is the locus of optimal bundles traced out as $p$ changes while keeping $m$ constant.
\end{definition}
In the case of a Cobb-Douglas preference, given a fixed $\alpha, \beta, m, p_2$, we would yield the demand curve for good $1$, where the ordinary demand function is already calculated to be $q_1^{*} = f(p_1) = \frac{\alpha}{\alpha + \beta} \cdot \frac{m}{p_1}$

\subsection{Income and Substitution Effects(收入效应和替换效应)}
When price changes, we consider it having two effects: the \textbf{\textit{income effect}} and the \textbf{\textit{substitution effect}}. \\
For example, if the price of good $1$ drops:
\begin{quote}
    1. The relative price of good $1$ drops (opportunity cost drops), so that consumers buy more of good $1$. This is the substitution effect ($SE$). \\
    2. The budget set expands such that the real income increases, so that consumer buy more of good $1$. This is the income effect ($IE$).
\end{quote}
Assume $m, p_2$ to be constant, as $p_1$ drops, $U^O \to U^F, q_1^O \to q_1^F, q_2^O \to q_2^F$, where $O$ represents "original", and $F$ represents "final", then the total effect ($TE$) is
$$TE = SE + IE = q_1^F - q_1^O$$

The substitution effect is described as: how would $q_1$ change if $m$ drops to just attain original utility after $p_1$ drops. This forms the \textbf{\textit{compensated budget line(补偿预算线)}}. This is the Hick's decomposition where we attain the original utility. If we label $C$ as the cost minimization point on the compensated budget line, then
$$SE = q_1^C - q_1^O$$
then
$$IE = q_1^F - q_1^C = TE - SE$$

For normal goods, $SE$ and $IE$ reinforce each other, that is they are always the same sign. \\
For inferior goods, $SE$ and $IE$ oppose each other, where $|SE| > |IE|: TE > 0$ if $p$ drops. \\
For Giffen goods (extremely inferior), we have $|SE| < |IE|: TE < 0$ if $p$ drops. \\
For Leontief preference, $SE = 0$, $TE = IE$. \\
For linear preference, it can be the case that $SE = IE = TE = 0$, or the case that $SE = TE, IE = 0$, or $TE = SE + IE$. \\
For quasilinear preference, we usually analyze from one interior bundle to another interior bundle, where $SE = TE, IE = 0$.

\subsection{Welfare Effect of a Price Change(价格变化的福利效应)}
There are three \textit{money measures} of a price change: compensating variation, equivalent variation, and change in consumer surplus. \\
\subsubsection{Compensating Variation(补偿变化)}
The compensating variation is the change in $m$, where if there is a change in $p$, the compensating variation will restore consumer's original utility level. \\
This is often considered when there is a price increase, we can think this as a compensation for a price increase, or a take-away for a price decrease.

\subsubsection{Equivalent Variation(等效变化)}
The equivalent variation is the change in $m$, where if there is a change in $p$, the equivalent variation will reach the consumer's final utility level.

\subsubsection{Change in Consumer Surplus(消费者效用盈余变化)}
The change in consumer surplus will be the area under the \textbf{Marshallian demand curve}, use integral to solve for the change.

\subsection{Relationship between $CV, EV, \Delta CS$}
If the $IE = 0$, then $|CV| = |\Delta CS| = |EV|$
If the $IE \approx 0$, then $|CV| \approx |\Delta CS| \approx |EV|$
The larger the income effectof the price change, the larger the difference between the three welfare measures. \\
The income effect is typically small for goods that account for goods that account for only a small fraction of total expenditure.

\newpage