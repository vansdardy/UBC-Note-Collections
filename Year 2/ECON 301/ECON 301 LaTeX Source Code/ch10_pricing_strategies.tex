\section{Pricing Strategies(定价策略)}
The pricing strategies describe how firm sets price(s) based on objectives and characteristics of the product and the market. \\
For advanced pricing strategies to be possible and profitable, the firm must
\begin{quote}
    1. possess some degree of market power. \\
    2. be able to cost-efficiently prevent resale and arbitrage.    
\end{quote}
\begin{definition}
    \textbf{\textit{Price discrimination, or differential pricing}} involves charging different consumers different prices for the same good, when price differences do not reflect differences in the cost of providing the good to the different consumer.
\end{definition}

\subsection{First-Degree Price Discrimination(一级价格分歧)}
Beyond the definition of price discrimination, the first-degree price discrimination has two more conditions:
\begin{quote}
    1. firm's customers have different willingness to pay (WTP) for the firm's product \\
    2. firm can identify customers' WTP before transaction occurs
\end{quote}
The final phenomena will be
\begin{quote}
    1. each unit is sold at a different price (varies across buyers) \\
    2. each unit is sold for consumer's maximum WTP
\end{quote}
In this case, $MR = D$, with $CS = 0, DWL = 0$.

\subsection{Third-Degree Price Discrimination(三级价格分歧)}
The additional conditions that a third-degree price discrimination must satisfy include:
\begin{quote}
    1. firm has several \textbf{types} of consumers with different WTP \\
    2. firm can identify customer's type before transaction happens
\end{quote}
The final phenomena will be
\begin{quote}
    1. different types of consumers will be charged different price \\
    2. buyers of the same type share the same price 
\end{quote}
Assume two markets, with $p_1, q_1; p_2, q_2$, uniform pricing within each market, to maximize profit, we must have
$$MR_1 = MR_2 = MC$$
And for the more inelastic market, say market $1$, the clearing price across markets will have the relationship that $p_1^{*} > p_2^{*}$, even when the marginal cost is the same for both markets.

\subsection{Second-Degree Price Discrimination(二级价格分歧)}
Second-degree price discrimination does not classify customers, thus it can be \textbf{possible} if the following condition is satisfied:
\begin{quote}
    1. firm has different types of customers with different WTP
\end{quote}
This is a type of \textbf{\textit{indirect price discrimination}}, which involves offering \textbf{\textit{a menu of options}}, where the customer is \textbf{free to choose preferred option}. Thus, the menu of options needs to be \textbf{\textit{incentive compatible}}. \\
Assume the firm offers 2 options and has two types of customer ($A$ and $B$), the firm wishes type-$A$ customers buy the first option and type-$B$ customers buy the second option. Then, to be incentive compatible, we must have
$$CS_A(\textbf{1}) \ge CS_A(\textbf{2}), CS_B(\textbf{2}) \ge CS_B(\textbf{1})$$
The four common forms of second-degree price discrimination are:
\begin{quote}
    1. Quantity discounts \\
    2. Versioning \\
    3. Discount coupons \\
    4. Bundling
\end{quote}
\subsubsection{Quantity Discounts}
This type offers lower per-unit price if a particular threshold quantity is met. \\
Take the example of two types of customers: occasional bakers, and serious bakers. \\
The firm that produces flour has two options: either $\$2.7/\text{kg}$ and buy however much flour, or $\$1.2/\text{kg}$ if buy $20+$ kg flour. \\
The firm wants occasional bakers to choose option 1, and serious bakers to choose option 2.

\subsubsection{Versioning}
This type offers a range of products that all varieties is of the same core product. \\
Each version of the product will have a \textbf{different markup} and a \textbf{different marginal cost}. \\
Take the example of NVDIA producing RTX4060 and RTX4070, where there are two types of consumers, casual gamers, and serious gamers. \\
NVDIA wishes the casual gamers to buy RTX4060, where the serious gamers to buy RTX4070.

\subsubsection{Discount coupons}
This type involves charging customers will a less elastic demand more. \\
Since the smaller the fraction icnome spent on an item, the less elastic the demand; thus wealthier consumers generally have less elastic demand. \\
Wealthier consumers also tend to place a higher value on their tiem and so are less likely to search for low prices or coupons. \\
We want the coupons to minimally cover the opportunity cost of finding the coupons for consumers with less income.

\subsubsection{Bundling}
This involves firm selling $2+$ products together as a package. The firm can adopt standalone pricing as usual, or use either pure-bundling (sell products only as part of a bundle) or mixed bundling (sell products both individually and as a bundle). \\
If there is a negative correlation for the same product for different consumers across products, that is if $A$ values good 1 more than 2, but $B$ values good 2 more than 1, then bundling can be considered. \\
Specifically, mixed bundling can be applied when the marginal cost of some goods is high enough that it makes sense to let some consumers to opt out the bundle. Thus, the firm needs to adjust the price to make sure the firm is not compensating for consumer, and the price change will be incentive compatible.

\newpage