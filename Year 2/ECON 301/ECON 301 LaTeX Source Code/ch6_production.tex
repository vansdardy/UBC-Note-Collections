\section{Production(生产)}
\begin{definition}
    A \textbf{\textit{firm(企业)}} is an organization that transforms inputs (resources) into outputs (goods and services). \\
    \textbf{\textit{Production}} is the process of transforming inputs into outputs, including other ancillary activities.
\end{definition}
Production theory analyzes how much output should be produced, and what combination of inputs should be used to produce. \\
We make the following assumptions about production:
\begin{quote}
    1. Produces single good. \\
    2. Has already selected the good to be produced. \\
    3. Uses only two inputs, labour ($L$) and capital ($K$). \\
    4. Can buy unlimited quantities of both inputs at fixed prices. \\
    5. Can access unlimited financial resources through bankers, investors.
\end{quote}
Thus, we can consider a \textbf{\textit{production function(生产函数)}}, such that it \textbf{shows the maximum amount of output given some input}, can be \textbf{represented by an equation, a graph, or a table}. \\
The inputs and outputs are measured in physical units per period of time, so if we assume two inputs, the general production can be 
$$Q = f(L, K)$$
Notice that \textbf{positive monotonic transformation} does not apply to production functions as the output is cardinal, not ordinal. \\
Firms can vary some inputs more easily and more quickly, so inputs can be classified as either fixed or variable, where
\begin{quote}
    1. Fixed input: an input that cannot be varied over the given time frame. \\
    2. Variable input: an input that can be varied over the given time frame.
\end{quote}
Thus, we can define
\begin{definition}
    The \textbf{\textit{long run}} is the time frame in which all inputs are variable; the \textbf{\textit{short run}} is any time frame short enough that at least $1$ input will not vary.
\end{definition}


\subsection{Production with One Variable Input(单变量生产输入)}
This is a short run production. \\
The \textbf{\textit{total product(所有产出)}} curve shows the how $Q$ changes by holding $1$ input constant, and $1$ input variable. \\
For most short-run TP curves:
\begin{quote}
    1. $TP = 0$ when $L = 0$ \\
    2. Initial: increasing at an increasing rate \\
    3. Then: increasing at a decreasing rate \\
    4. Finally: decreasing
\end{quote}
The \textbf{\textit{marginal product(边际产出)}} is defined to be
$$MP_L = \dv{TP_L}{L}$$
Notice that marginal product initially increase (labour specialization), then starts to decline at some point, finally becoming negative (crowding). This is the \textbf{\textit{Law of Diminishing Marginal Returns(边际产出递减)}}, which only applies in short run. \\
The \textbf{\textit{average product(平均产出)}} is defined to be
$$AP_L = \frac{TP_L}{L}$$
The relationship between $AP$ and $MP$ is as follows: 
\begin{quote}
    1. when $MP > AP$, $AP$ is increasing \\
    2. when $MP = AP$, $AP$ obtains its maximum \\
    3. when $MP < AP$, $AP$ is decreasing
\end{quote}

\subsection{Production with Two Variable Inputs(双变量生产输入)}
This could be a short-run or long-run production depending on the number of inputs required. \\
We use the \textbf{\textit{production isoquant(等产出线)}} to refer to all the combinations of inputs for some given output. We mainly focus on negative-sloped isoquants. \\
This introduces the concept of \textbf{\textit{marginal rate of technical substitution(技术替换边际速率)}}, that is the rate at which a small quantity of one input can be substituted by the other given a constant output. Mathematically,
$$MRTS = -\dv{K}{L}$$
$$MRTS = \frac{MP_L}{MP_K}$$

\subsubsection{Substitutability of Inputs(生产输入的可替换性)}
Consider a Leontief production function, $Q = \min{\{ax_1, bx_2\}}$, the $MRTS$ is undefined. \\
Consider a linear production function, $Q = \alpha x_1 + \beta x_2$, the $MRTS$ is $\frac{\alpha}{\beta}$. \\
Consider a Cobb-Douglas production function, $Q = Ax_1^\alpha x_2^\beta$, where $A$ measures the productive efficiency, the $MRTS$ will be a function of $x_1, x_2$.

\subsubsection{Isocost Lines(等成本线)}
Given two inputs $L, K$ with their unit price to be $w, r$, where $w$ represents wage and $r$ represents rent, then the firm's total cost is
$$C = wL + rK$$
If we consider $K = f(L)$, then from the above cost function, $\frac{C}{r}$ is the intercept and $-\frac{w}{r}$ is the slope. \\
For a fixed output target, if the firm wishes to maximize profit, it must minimize the costs. Given that
$$\pi = TR - TC$$
Cost-minimization occurs when the slope of the isoquant equals the slope of the isocost, that is
$$\dv{K}{L} = \frac{w}{r} \to \frac{\dd K}{w} = \frac{\dd L}{r} \to \frac{MP_L}{w} = \frac{MP_K}{r}$$
When the input prices change then there will be a change in the input mix and a change in the \textbf{optimal mix of input}.

\subsubsection{Firm's Long Run Problem(企业长期运营问题)}
Consider a firm's production function $Q = f(L, K)$, the firm aims to minimize the cost, in the form
$$\min_{L,K} wL + rK \text{ s.t. } Q = \bar{Q}$$
We use the Lagrangian to solve for $L^{*}, K^{*}$, and then we can substitute back into the cost function to find the minimum cost. \\
If the production function is linear, the minimum cost depends on which input has a larger marginal product per unit. \\
If the production function is Leontief, the minimum cost occurs when $\alpha x_1^{*} = \beta x_2^{*} = Q$.

\subsubsection{Returns to Scale(规模报酬)}
This describes what happens to the output when all inputs are scaled by the same factor. \\
If the inputs are scaled by a factor of $k$, then
\begin{quote}
    1. Constant returns to scale: output is also scaled by a factor of $k$. \\
    2. Increasing returns to scale: output is scaled by a factor of $> k$. \\
    3. Decreasing returns to scale: output is scaled by a factor of $< k$.
\end{quote}
The \textbf{\textit{long-run expansion path(长期扩大生产曲线)}} is the input combinations the firm would use at different output levels in the long run.

\newpage