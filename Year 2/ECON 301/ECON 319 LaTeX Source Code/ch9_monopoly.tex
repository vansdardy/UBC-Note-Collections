\section{Monopoly(完全市场垄断)}
A monopoly is a market served by only single seller of a product with no close substitutes. \\
There are barriers to entry in a monopoly, these include: 
\begin{quote}
    1. extreme economies of scale \\
    2. switching costs \\
    3. product differentiation \\
    4. control of key inputs \\
    5. absolute cost advantages \\
    6. government regulation    
\end{quote}
This means, the firm is a \textbf{\textit{price setter}} in the market, and the industry is the firm.

\subsection{Monopoly and Demand(垄断与需求)}
We first assume \textbf{\textit{uniform pricing(单一定价)}} for the firm, and the firm faces the market demand curve. \\
If the price increases, the firm sells less; if the price decreases, the firm sells more.

\subsection{Marginal Revenue and the Price Elasticity of Demand(边际营收与需求价格弹性)}
We know that $TR = P \times Q$, so $MR = \dv{TR}{Q}$, and by assuming that $P = f(Q)$, by knowing the price elasticity of demand is 
$$\epsilon_D = \frac{\frac{\dd Q}{Q}}{\frac{\dd P}{P}} = \dv{Q}{P} \cdot \frac{P}{Q}$$
then, the marginal revenue is
$$MR = P(1 - \frac{1}{|\epsilon_D|})$$
We take the absolute value of price elasticity of demand, as it is negative. \\
Note that:
\begin{quote}
    1. When demand is elastic $\to$ $|\epsilon_D| > 1$, we have $MR > 0$ \\
    2. When demand is inelastic $\to$ $|\epsilon_D| < 1$, we have $MR < 0$ \\
    2. When demand is unielastic $\to$ $|\epsilon_D| = 1$, we have $MR = 0$
\end{quote}

\subsection{Marginal Revenue with Linear Demand(线性需求的边际营收)}
Assume the linear demand function is characterized by
$$Q = \alpha - \beta P$$
Then, the inverse demand function is $P = \frac{\alpha}{\beta} - \frac{1}{\beta} Q$
The marginal revenue function is thus $MR = \frac{\alpha}{\beta} - \frac{2}{\beta} Q$

\subsection{Monopoly and Profit Maximization(垄断与其利润最大化)}
The goal of the firm is to
$$\max_Q \pi = TR(Q) - TC(Q)$$
This also indicates that profit maximization is achieved when $MR = MC$ and $\dv{MC}{Q} > 0$. The market thus clears at some $(Q_M, P_M)$.

\subsection{The Lerner Index(勒尔纳指数)}
The Lerner index is measures of a firm's markup and indicates the amount of market power the firm enjoys. It is defined as
$$LI = \frac{P - MC}{P}$$
And we have its relationship with the price elasticity of demand to be
$$LI = \frac{1}{|\epsilon_D|}$$
This means, the more inelastic a product, the larger the market power the firm has. This equation is obtained at profit maximization level.

\subsection{Profit under Monopoly(垄断的利润)}
As usual, $\pi = TR - TC = (P - ATC)Q$. \\
The producer surplus is thus $PS = TR - TVC = TR - \int MC \dd Q$. Here we have $PS > \pi$, since $FC > 0$.

\subsection{The Welfare Cost of Monopoly(垄断的福利成本)}
Since monopoly charges $P > MC$, thus, the $CS$ is lower under monopoly and the $PS$ is higher compared to a perfectly competitive market. \\
Therefore, monopoly generates a deadweight loss.

\newpage