\section{Costs(生产成本)}
\subsection{Costs in the Short Run(短期生产成本)}
In the short run, the total costs can be broken down into fixed costs and variable costs, where $TC = FC + VC$. \\
Consider a production function, $Q = f(L, \bar{K})$, when $Q = 0$, we have $TC = FC$, and we further have $MC = \dv{TC}{Q} = \dv{VC}{Q}$. \\
Other relevant parameters and relationships include:
$$AFC = \frac{FC}{Q}, AVC = \frac{VC}{Q}, ATC = \frac{TC}{Q}$$
$$VC = \int_0^Q MC \dd Q$$
At $Q = 1$, we also have $MC = AVC$. If we label $Q_V$ to be the quantity where $AVC$ reaches a minimum and $Q_T$ to be the quantity where $ATC$ reaches a minimum, then when $FC > 0$, $Q_T > Q_V$. $MC$ curve always intersects $AVC$ and $ATC$ at their minimums.

\subsection{Short-run and Long-run Costs(短期与长期生产成本关系)}
Consider a firm with only $L, K$ as its inputs, we can have one long-run expansion path, and many short-run expansion path given a certain level of $K$. \\
So for a fixed quantity of $K$, there is a unique output level where $TC^{SR} = TC^{LR}$, but for other output levels, $TC^{SR} > TC^{LR}$. This can be considered to be a cost penalty in SR compared to LR, except for that unique level of output. \\
Given these relationships, we also know $ATC^{SR} \ge ATC^{LR}$, and the equal sign is obtained at that unique output level. In the long run, a profit-maximizing firm would install the amount of capital that would minimize $ATC$ of production at its target output level, thus, the $ATC^{LR}$ is the \textbf{lower envelope} of the $ATC^{SR}$ curves. The $ATC^{LR}$ curves will be smooth if firm can use any quantity of capital it wants in the long run.
\subsubsection{Long-Run Average Costs}
If $ATC^{LR}$ decreases as $Q$ increases, the firm is experiencing \textbf{\textit{economies of scale(规模经济)}}; \\
If $ATC^{LR}$ remains the same as $Q$ increases, the firm is experiencing \textbf{\textit{constant economies of scale}}; \\
If $ATC^{LR}$ increases as $Q$ increases, the firm is experiencing \textbf{\textit{diseconomies of scale}}. \\
A firm either experiences EOS and DOS or EOS + CEOS + DOS, as output expands. When the firm is only experiencing EOS + DOS, the quantity at which $ATC^{LR}$ obtains its minimum is \textbf{\textit{efficient scale of operation}}, where if there is also CEOS, the lowest quantity at which $ATC^{LR}$ obtains its minimum is \textbf{\textit{minimum efficient scale of operation}}. \\
The relationship between returns to scale and economies of scale is as follows:
\begin{quote}
    1. If the firm is experiencing decreasing returns to scale: the firm can have either EOS, CEOS, or DOS. \\
    2. If the firm is experiencing constant returns to scale: the firm can have EOS, or CEOS. \\
    3. If the firm is experiencing increasing returns to scale: the firm can only have EOS.
\end{quote}

\newpage