\section{Position and Momentum Observables - 位置与动量可观测量}
\subsection{The Properties of and Transformations between the Observables - 可观测量的性质和互相变换}
In this section, we explore the relationship between the two important observables. \\
First, we revisit the 1D continuous system, where the quantum system is defined as a Hilbert space: $\hilbert = \Span{\ket{x}}_{x \in \R}$ with the condition $\braket{x}{x'} = \delta(x-x')$. A general quantum state can be expressed in the position basis:
$$\ketpsi = \intR \psi(x)\ketx \dd x, \intR \abs{\psi(x)}^2 \dd x = 1$$
where in particular we know $\psi(x) = \braket{x}{\psi}$. \\
We have the \impt{position observable} to be:
$$\oppos = \intR x \dyad{x} \dd x$$
However, we can also represent the same state in another basis, say $\ket{p}_{p \in \R}$. Here, a general state will be represented as:
$$\ketpsi = \intR \tilde{\psi}(p) \ket{p} \dd p, \tilde{\psi}(p) = \braket{p}{\psi}$$
We now wish to understand how the position wavefunction and the momentum wavefunction relate with each other. Here, we may substitute the state in different bases in the wavefunction calculation:
\begin{align*}
    \psi(x) &= \braket{x}{\psi} \\
    &= \brax \intR \tilde{\psi}(p) \ket{p} \dd p \\
    &= \intR \tilde{\psi}(p) \braket{x}{p} \dd p \\
    \tilde{\psi}(p) &= \intR \psi(x) \braket{p}{x} \dd x
\end{align*}
We observe that, knowing the inner product between these two bases is enough to properly build the relationship.
\begin{definition}
    Define $\ket{p}_{p \in \R}$ to be the \uimpt{momentum basis} in an 1D continuous system, where the inner product between the position basis and the momentum basis is:
    $$\forall x, p \in \R, \braket{x}{p} = \frac{1}{\sqrt{2\pi\hbar}}\exp(\frac{\imag p x}{\hbar})$$
\end{definition}
Therefore,
\begin{align*}
    \psi(x) &= \frac{1}{\sqrt{2\pi\hbar}} \intR \tilde{\psi}(p) \exp(\frac{\imag p x}{\hbar}) \dd p \\
    \tilde{\psi}(p) &= \frac{1}{\sqrt{2\pi\hbar}} \intR \psi(x) \exp(-\frac{\imag p x}{\hbar}) \dd x
\end{align*}
They represent the \impt{Fourier transform} between the two bases. \\
We can thus further express a basis state in the other basis as follows:
\begin{align*}
    \ketx &= \id \ketx \\
    &= \intR \dyad{p} \dd p \ketx \\
    &= \intR \ket{p}\braket{p}{x} \dd p \\
    &= \frac{1}{\sqrt{2\pi\hbar}} \intR \exp(-\frac{\imag p x}{\hbar}) \ket{p} \dd p \\
    \ket{p} &= \frac{1}{\sqrt{2\pi\hbar}} \intR \exp(\frac{\imag p x}{\hbar}) \ket{x} \dd x
\end{align*}
This can be confirmed by the above definition being consistent with the Dirac delta function $\braket{x}{x'} = \delta(x-x')$ even expressing $\ketx$ in the momentum basis. The deduction process is hereby not shown.
\begin{definition}
    Define the momentum observable to be:
    $$\hat{P} = \intR p \dyad{p} \dd p$$
    where it has the property of being Hermitian: $\conjt{\hat{P}} = \hat{P}$
\end{definition}
It is straightforward to understand that when the observables act on the respective wavefunction of respective basis, we have:
$$\begin{cases}
    \hat{X}: \psi(x) \mapsto x \cdot \psi(x) \\
    \hat{P}: \tilde{\psi}(p) \mapsto p \cdot \tilde{\psi}(p)
\end{cases}$$
However, we wish to further know how the observables act on each other's wavefunction. One way is to calculate $\psi(x)$ after $\hat{P}$ is applied:
\begin{align*}
    \mel{x}{\hat{P}}{\psi} &= \brax (\intR p \dyad{p} \dd p) \ketpsi \\
    &= \intR p \braket{x}{p}\braket{p}{\psi} \dd p \\
    &= \frac{1}{\sqrt{2 \pi \hbar}} \intR p \cdot \exp(\frac{\imag p x}{\hbar}) \tilde{\psi}(p) \dd p
\end{align*}
The key here is to notice that: $p \cdot \exp(\frac{\imag p x}{\hbar}) = (-\imag \hbar)\pdv{x}\exp(\frac{\imag p x}{\hbar})$. Notice that the partial derivative and the integral are both linear, thus, we can swap their positions and have:
\begin{align*}
    \mel{x}{\hat{P}}{\psi} &= -\imag \hbar \pdv{x} \intR \frac{1}{\sqrt{2 \pi \hbar}} \exp(\frac{\imag p x}{\hbar}) \tilde{\psi}(p) \dd p \\
    &= -\imag \hbar \pdv{x} \psi(x)
\end{align*}
Therefore, we yield:
$$\begin{cases}
    \hat{X}: \tilde{\psi}(p) \mapsto +\imag \hbar \pdv{p}\tilde{\psi}(p) \\
    \hat{P}: \psi(x) \mapsto -\imag \hbar \pdv{x}\psi(x)
\end{cases}$$
Finally, we wish to observe how the two observables commute. We first have:
\begin{align*}
    \mel{x}{\hat{X}\hat{P}}{\psi} &= (\brax \hat{X})(\hat{P} \ketpsi) \\
    &= x\mel{x}{\hat{P}}{\psi} \\
    &= x \cdot (-\imag \hbar \pdv{x} \psi(x))
\end{align*}
Then, we also have:
\begin{align*}
    \mel{x}{\hat{P}\hat{X}}{\psi} &= (\brax \hat{P})(\hat{X} \ketpsi) \\
    &= \mel{x}{\hat{P}}{(x \cdot \psi(x))} \\
    &= -\imag \hbar \pdv{x} (x \cdot \psi(x)) \\
    &= -\imag \hbar \psi(x) - \imag \hbar x \cdot \pdv{x}\psi(x)
\end{align*}
So, we must have:
$$\mel{x}{[\hat{X}, \hat{P}]}{\psi} = \mel{x}{\hat{X}\hat{P}}{\psi} - \mel{x}{\hat{P}\hat{X}}{\psi} = \imag \hbar \psi(x)$$
This eventually gives us
\begin{theorem}
    $$[\hat{X}, \hat{P}] = \imag \hbar \id$$
\end{theorem}

\subsection{Ehrenfest Relations - 埃伦费斯特原理}
\label{subsec:ehrenfest}
So far, we can understand the action of operators as one of the following:
\begin{quote}
    1. Representation of the action itself \\
    2. Provides eigenvectors and eigenvalues \\
    3. Its action on coefficients of vectors in a specific basis
\end{quote}
We have previously shown the canonical commutation relation between the position and momentum observables. Now we wish to utilize this knowledge to connect the classical representations and the quantum representations. \\
Consider a neutral particle in an 1D continuous space. \impt{Classically}, we may express the total energy $E$ in this classical system by the sum of the particle's kinetic energy $E_k = \frac{1}{2}\mu \cdot v^2 = \frac{p^2}{2\mu}$ and the particle's potential energy $E_p = V(x)$, where the position $x$ is a function of time $x(t)$ and the momentum is also a function of time $p(t)$. The equation of motion are derived to be:
$$\pdv{t}x(t) = \frac{p(t)}{\mu}, \pdv{t}p(t) = -\pdv{V(x(t))}{x}$$
In the quantum world, classical physical variables, as shown in the Stern-Gerlach experiment, are promoted to observables/operators. In this case:
\begin{quote}
    1. Total Energy: $\hamiltonian$. \\
    2. Position: $\hat{X}$. \\
    3. Momentum: $\hat{P}$.
\end{quote}
where the Hamiltonian is:
$$\hamiltonian = \frac{\opmtm^2}{2\mu} + V(\oppos)$$
and we assume that $V$ has a Taylor expansion, where
$$V(\oppos) = \Sum{n=0}{\infty}a_n \oppos^n$$
Recall that for a time-independent observable, we have
$$\pdv{t}\expval{\hat{A}}_t = \frac{\imag}{\hbar} \expval{[\hamiltonian, \hat{A}]}_t$$
This gives us:
\begin{align*}
    \pdv{t}\expval{\oppos}_t &= \frac{\imag}{\hbar} \expval{[\frac{\opmtm^2}{2\mu} + V(\oppos), \oppos]}_t \\
    &= \frac{\imag}{\hbar} \expval{[\frac{\opmtm^2}{2\mu}, \oppos] + [V(\oppos), \oppos]}_t \\
    &= \frac{\imag}{2\mu \hbar} \expval{[\opmtm^2, \oppos]}_t
\end{align*}
Specifically, we have
\begin{align*}
    [\opmtm^2, \oppos] &= \opmtm\opmtm\oppos - \oppos\opmtm\opmtm \\
    &= \opmtm(\oppos\opmtm - [\oppos, \opmtm]) - \oppos\opmtm\opmtm \\
    &= -\opmtm[\oppos, \opmtm] + \opmtm\oppos\opmtm - \oppos\opmtm\opmtm \\
    &= -\imag \hbar \opmtm + ([\opmtm, \oppos])\opmtm \\
    &= -2\imag\hbar\opmtm
\end{align*}
Therefore,
\begin{align*}
    \pdv{t}\expval{\oppos}_t &= \frac{\imag}{2\mu \hbar} \expval{[\opmtm^2, \oppos]}_t \\
    &= \frac{\imag}{2\mu \hbar} \expval{-2\imag\hbar\opmtm}_t \\
    &= \frac{\expval{\opmtm}_t}{\mu}
\end{align*}
In a similar fashion, we have:
\begin{align*}
    \pdv{t}\expval{\opmtm}_t &= \frac{\imag}{\hbar} \expval{[\frac{\opmtm^2}{2\mu} + V(\oppos), \opmtm]}_t \\
    &= \frac{\imag}{\hbar} \expval{[V(\oppos), \opmtm]}_t \\
    &= \frac{\imag}{\hbar} \expval{\imag\hbar \pdv{V(\oppos)}{x}}_t \\
    &= -\expval{\pdv{V(\oppos)}{x}}_t
\end{align*}
This concludes that:
\begin{theorem}
    The \uimpt{Ehrenfest Relations/Theorem} claims that for a neutral particle in an 1D continuous system, we have an analagous relationship between classical physical quantities and their promoted observables, specifically:
    $$\pdv{t}\expval{\oppos}_t = \frac{\expval{\opmtm}_t}{\mu}$$
    $$ \pdv{t}\expval{\opmtm}_t =  -\expval{\pdv{V(\oppos)}{x}}_t$$
\end{theorem}
In this case, we further explore how the Schr\"odinger's Equation is affected for such a particle. \\
For vectors, we already have the Schr\"odinger's Equation to be:
$$\hamiltonian \ket{\psi(t)} = \imag \hbar \pdv{t}\ket{\psi(t)}$$
The position wavefunction is thus transformed in the following way (by applying $\brax$ from the left):
\begin{align*}
    \forall x \in \R, \mel{x}{\hamiltonian}{\psi(t)} &= \mel{x}{\imag \hbar \pdv{t}}{\psi(t)} \\
    &= \imag\hbar \pdv{t}\braket{x}{\psi(t)} \\
    &= \imag \hbar \pdv{t}\psi(x, t)
\end{align*}
If we further separate the Hamiltonian in its respective parts, we first have:
\begin{align*}
    \frac{\opmtm^2}{2\mu}: \psi(x, t) &\mapsto \frac{1}{2\mu}(-\imag\hbar\pdv{x})^2 = -\frac{\hbar^2}{2\mu}\pdv[2]{x} \psi(x, t) \\
    V(\oppos): \psi(x, t) &\mapsto V(x)\psi(x, t)
\end{align*}
Therefore, 
\begin{theorem}
    For a neutral particle in 1D-continous space, we have two Schr\"odinger's Equation, one for position, and one for momentum:
    \begin{align*}
        -\frac{\hbar^2}{2\mu}\pdv[2]{x} \psi(x, t) + V(x)\psi(x, t) &= \imag \hbar \pdv{t}\psi(x, t) \\
        \frac{p^2}{2\mu}\tilde{\psi}(p, t) + V(\imag \hbar \pdv{p})\tilde{\psi}(p, t) &= \imag \hbar \pdv{t}\tilde{\psi}(p, t)
    \end{align*}
\end{theorem}
If we have solutions for $\psi(x, t)$ and $\tilde{\psi}(p, t)$, then we know that
$$\ket{\psi(t)} = \intR \psi(x, t) \ket{x} \dd x = \intR \tilde{\psi}(p,t) \ket{p} \dd p$$
is a solution for the Schr\"odinger's Equation for $\ket{\psi(t)}$. \\
To simply demonstrate, consider a free particle where $V(x) = 0$, so the Hamiltonian is directly $\hamiltonian = \frac{\opmtm^2}{2\mu}$. We know that $\opmtm^2 = \intR p^2 \dyad{p} \dd p$ is diagonal in the momentum basis, so it is easier for us to solve the equation for $\tilde{\psi}(p, t)$.
\begin{align*}
    \frac{p^2}{2\mu}\tilde{\psi}(p, t) &= \imag \hbar \pdv{t}\tilde{\psi}(p, t) \\
    -\frac{\imag p^2}{2\mu\hbar} \tilde{\psi}(p, t) &= \pdv{t} \tilde{\psi}(p, t) \\
    \tilde{\psi}(p, t) &= \tilde{\psi}(p, 0) \cdot \exp(-\frac{\imag p^2t}{2\mu\hbar})
\end{align*}
Thus, the equation of motion for a general state $\ket{\psi(t)}$ is:
$$\ket{\psi(t)} = \intR \tilde{\psi}(p, 0) \cdot \exp(-\frac{\imag p^2t}{2\mu\hbar}) \ket{p} \dd p$$
The consistency can be checked with the propagator:
$$\opuni(t) = \exp(-\frac{\imag \hamiltonian t}{\hbar})$$
applied to $\ket{\psi(0)} = \intR \tilde{\psi}(p, 0) \ket{p} \dd p$. \\
By the Ehrenfest relations, we also know
$$\pdv{t} \expval{\opmtm}_t = - \expval{\pdv{V(\oppos)}{x}}_t = 0$$
This indicates the momentum is conserved, which further implies:
$$\pdv{t} \expval{\oppos}_t = \frac{\expval{\opmtm}_t}{\mu} = \frac{\expval{\opmtm}_0}{\mu}$$
which is equivalent to $\expval{\oppos}_t = \expval{\oppos}_0 + t \cdot \frac{\expval{\opmtm}_0}{\mu}$. \\
In this case, $(\Delta \oppos)_t \propto t^2$. On a similar note, we yielded that the momentum wavefunction is transformed in the following way:
$$\tilde{\psi}(p, t) = \tilde{\psi}(p, 0) \cdot \exp(-\frac{\imag p^2t}{2\mu\hbar})$$
We can observe it only evolves with a complex phase. This indicates that every moment of $\opmtm$, which is $\expval{\opmtm^n}_t$ is time-independent. In momentum space, this does not change much, but in position space, it changes the wavefunction significantly, more specifically, the probability spreads out as time evolves.

\newpage